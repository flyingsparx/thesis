\chapter{Critical Assessment and Conclusions}


\section{Critical Analysis of Results}
\subsection{Analysis of Initial Research}
\begin{itemize}
\item What was the use of initial research?
\item Are the results sensible?
\item How have the results shaped the further research?
\end{itemize}

\subsection{Analysis of Final Results}
\begin{itemize}
\item Have methods been able to sensibly predict retweet volumes?
\item Have methods sensibly inferred Tweet interestingness?
\item What might have worked better?
\item Which parts were useless?
\item Which parts helped develop other areas of research which may have provided further avenues of research ideas?
\end{itemize}


\section{Further and Future Work}
How can this research be taken further in the future?

\begin{itemize}
\item Use previous results to predict how far a tweet is likely to be retweeted (for advertising purposes)
\item Useful for detecting the kind of messages that are likely to travel further
\item As well as providing an interest level, the systems also predict sensible estimations on retweet volumes.
\item Perhaps useful for measuring the spread of rumours.
\end{itemize}


\section{Conclusions}
\subsection{Summary}
Summarise events and processes covered, reiterate what the point of the work was and how each part of the work covered relates to that.

\subsection{Contributions}
Restate the original contributions (from Introduction section). Explain the ways in which the work done relates to the projected contributions, that it is novel and useful.