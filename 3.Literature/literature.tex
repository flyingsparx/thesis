Either overall lit-review here, individual ones for each chapter, or both (i.e. generall overall and specific ones too).

\section{Twitter and Retweeting}
References for work that helped with the introduction and initial understanding of Twitter. Include papers on:
\begin{itemize}
\item Twitter communities and network
\item Influential users (`hubs' and `authorities')
\end{itemize}

\section{Information Quality and Retrieval}
References for work regarding quality and relevance of information and retrieval techniques
\begin{itemize}
\item Recognition heuristic
\item Google search `bubble'
\item How Twitter handles information quality (Twitter noise, relevance) (evangelists and detractors - Bighona)
\item Document retrieval and recommendation systems (Gavalas)
\end{itemize}

\section{Tweet Interestingness}
Retweeting has become the focus of much research, particularly in defining the interest or credibility levels of information \cite{castillo11}. The phenomenon can be seen as a way in which users can `vote' for information in Twitter. If a user views a Tweet that he/she finds interesting, or expects his/her followers to, the Tweet can be retweeted so that the audience of the information is extended. It's unlikely that a user will retweet something uninteresting or inane, since they realise that their followers are probably not going to find it very engaging. Thus a user's friends (i.e. \emph{followees}) become filters of information by only forwarding on interesting information to their followers.\\
In \cite{java07}, the idea of Twitter communities is discussed, and how clusters of users with similar interests can build up over time in the social graph. This means that it's likely that several friends and followers of a user have a similar set of interests to the user, and are therefore likely to find similar Tweets of interest.\\
\cite{uysal11} discusses an idea similar to that of predicting the interestingness of a Tweet, but focuses largely on predicting the users most likely to find the Tweet interesting enough to retweet. Similarly, \cite{hong11} looks at the same issue but at the other way around by predicting the \textit{type} of Tweets that are likely to be retweeted many times. Discussions on the retweet decision in relation to a user's recognition of its features take place in \cite{chorley12} as well as conclusions about the effect of features such as the number of followers of a user or any pre-existing metadata on the interestingness of the Tweet.\\
A model for making predictions on retweet volume probability is introduced by \cite{zhu11} (and \cite{peng11}), which forms the basis for some of the ideas behind this paper. Their work involves the use of a logistic regression, trained by an extensive set of features, to predict retweet outcomes for a Tweet. We try and simplify parts of their methodology for efficiency, so that interestingness inferences can be made quickly and efficiently as the retweets occur in real time.\\
\cite{naveed11} and \cite{uysal11} also use machine learning to train models to predict retweet outcomes as part of their work, with the former using a logistic regression and employing their prediction forecasts to infer \textit{what} makes a Tweet interesting and the latter using a decision tree classifier to draw links between Tweets that are interesting to a user and that user's retweet likelihood on those Tweets.

\section{Scoring Tweets}
https://sites.google.com/site/learningtweetvalue/home - these guys came up with their own method of predicting retweet outcome (do not relate to interestigness per se, and instead look at the past history of user's tweet segments (such as certain features used, what hashtags contained, etc.)
