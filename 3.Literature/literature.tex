\chapter{Related Literature}

- RE-CHECK EACH CITED PAPER IN THIS SECTION (especially /

- MAYBE DO NOT USE SUB-HEADINGS SINCE MANY PAPERS ON INTERESTINGNESS

Either overall lit-review here, individual ones for each chapter, or both (i.e. generall overall and specific ones too).

\section{Twitter and Retweeting}
References for work that helped with the introduction and initial understanding of Twitter. Include papers on:
\begin{itemize}
\item Twitter communities and network
\item Influential users (`hubs' and `authorities')
\end{itemize}

\section{Information Quality and Retrieval}
References for work regarding quality and relevance of information and retrieval techniques
\begin{itemize}
\item Recognition heuristic
\item Google search `bubble'
\item How Twitter handles information quality (Twitter noise, relevance) (evangelists and detractors - Bighona)
\item Document retrieval and recommendation systems (Gavalas)
\end{itemize}

\section{Tweet Interestingness}

\cite{zadeh13} - Twitter already has high precision?

https://sites.google.com/site/learningtweetvalue/home - these guys came up with their own method of predicting retweet outcome (do not relate to interestigness per se, and instead look at the past history of user's tweet segments (such as certain features used, what hashtags contained, etc.)
