%%%%% STYLES AND STUFF 
\documentclass[a4paper,oneside,onecolumn,openright,12pt]{book}


\makeatletter

%
\usepackage{caption}
\usepackage{subcaption}
\usepackage{footmisc}
\usepackage{paralist}
\usepackage{amsthm}
\usepackage{subfig}

% Fonts, encoding, etc.
\usepackage{type1cm}
\usepackage[latin1]{inputenc}
\usepackage[british]{babel}
\usepackage[T1]{fontenc}
\usepackage{times}
\usepackage{hyperref}
\usepackage[all]{hypcap}
\usepackage{longtable}
\usepackage{algorithm}
\usepackage{algpseudocode}
\usepackage{multirow}
\usepackage{float}


% Line spacing
\def\baselinestretch{1.5}
\parindent0cm
\parskip1.5ex\@plus.7ex\@minus.1ex\relax

%Acronyms
 \usepackage{acronym}
 %Inline lists
 \usepackage{paralist}

%Glossary
\usepackage[toc]{glossaries}

% Page dimensions
\usepackage{vmargin}
\setpapersize{A4}
\setmarginsrb{40mm}{20mm}{25mm}{30mm}{14.5pt}{8mm}{0pt}{11mm}

% Footer and header
\usepackage{afterpage}
\usepackage{fancyhdr}
\pagestyle{fancy}
\fancyhead{}
\fancyhead[LE,RO]{\thepage}
\fancyhead[LO,RE]{\slshape \leftmark}
\fancyfoot{}
\renewcommand{\chaptermark}[1]{}
\renewcommand{\sectionmark}[1]%
             {\markboth{\thesection\ #1}{\thesection\ #1}}
\renewcommand{\subsectionmark}[1]{}
\fancypagestyle{plain}{%
  \fancyhead{}
  \fancyhead[LE,RO]{\thepage}
  \fancyfoot{}
  \renewcommand{\headrulewidth}{.6pt}
}

% Chapter
\def\@makechapterhead#1{%
  \ \\[-35.5pt]\hbox to \textwidth {%
    \hfill {\vbox{\hbox{\rule[5pt]{140pt}{4pt}}%
        \hbox to 140pt {\hfill\huge\bfseries\slshape \@chapapp\space\thechapter\/}}}}%
  \vskip55\p@%
  {\parindent \z@ \raggedright \normalfont%
    \interlinepenalty\@M%
    \Huge \bfseries #1\par\nobreak%
    \vskip 45\p@%
  }}
\def\@chapter[#1]#2{\ifnum \c@secnumdepth >\m@ne
                       \if@mainmatter
                         \refstepcounter{chapter}%
                         \typeout{\@chapapp\space\thechapter.}%
                         \addcontentsline{toc}{chapter}%
                                   {\protect\numberline{\thechapter}#1}%
                       \else
                         \addcontentsline{toc}{chapter}{#1}%
                       \fi
                    \else
                      \addcontentsline{toc}{chapter}{#1}%
                    \fi
                    \chaptermark{#1}%
                    \addtocontents{lof}{\protect\addvspace{10\p@}}%
                    \addtocontents{lot}{\protect\addvspace{10\p@}}%
                    \addtocontents{loa}{\protect\addvspace{10\p@}}%
                    \if@twocolumn
                      \@topnewpage[\@makechapterhead{#2}]%
                    \else
                      \@makechapterhead{#2}%
                      \@afterheading
                    \fi}
\def\@schapter#1{\addcontentsline{toc}{chapter}{#1}%
                 \markboth{#1}{#1}%
                 \addtocontents{lof}{\protect\addvspace{10\p@}}%
                 \addtocontents{lot}{\protect\addvspace{10\p@}}%
                 \addtocontents{loa}{\protect\addvspace{10\p@}}%
                 \if@twocolumn%
                    \@topnewpage[\@makeschapterhead{#1}]%
                 \else%
                    \@makeschapterhead{#1}%
                    \@afterheading%
                 \fi}
\def\@makeschapterhead#1{%
  \ \\[-35.5pt]\hbox to \textwidth {%
    \hfill {\vbox{\hbox{\rule[5pt]{140pt}{0pt}\rule[5pt]{0pt}{4pt}}%
        \hbox to 140pt {\hfill\huge\bfseries\slshape \ \/}}}}%
  \vskip55\p@%
  {\parindent \z@ \raggedright \normalfont%
    \interlinepenalty\@M%
    \Huge \bfseries  #1\par\nobreak%
    \vskip 45\p@%
  }}

% Table of contents
\def\contentsname{Contents}
\renewcommand\tableofcontents{%
    \if@twocolumn%
      \@restonecoltrue\onecolumn%
    \else%
      \@restonecolfalse%
    \fi%
    \chapter*{\contentsname}%
    \@starttoc{toc}%
    \if@restonecol\twocolumn\fi%
    }

% Bibliography
\renewenvironment{thebibliography}[1]
     {\chapter*{\bibname}%
      \list{\@biblabel{\@arabic\c@enumiv}}%
           {\settowidth\labelwidth{\@biblabel{#1}}%
            \leftmargin\labelwidth
            \advance\leftmargin\labelsep
            \@openbib@code
            \usecounter{enumiv}%
            \let\p@enumiv\@empty
            \renewcommand\theenumiv{\@arabic\c@enumiv}}%
      \sloppy
      \clubpenalty4000
      \@clubpenalty \clubpenalty
      \widowpenalty4000%
      \sfcode`\.\@m}
     {\def\@noitemerr
       {\@latex@warning{Empty `thebibliography' environment}}%
      \endlist}

% Floats
\long\def\@makecaption#1#2{%
  \vskip\abovecaptionskip
  \sbox\@tempboxa{\textbf{#1: #2}}%
  \ifdim \wd\@tempboxa >\hsize
    \textbf{#1: #2.}\par
  \else
    \global \@minipagefalse
    \hb@xt@\hsize{\hfil\box\@tempboxa\hfil}%
  \fi
  \vskip\belowcaptionskip}
\renewcommand{\topfraction}{0.9}
\renewcommand{\textfraction}{0.1}
\renewcommand{\floatpagefraction}{0.9}

% Tables
\usepackage{dcolumn}
\usepackage{hhline}

% Graphics
\usepackage[dvips]{graphicx}
\usepackage[usenames,dvipsnames]{color}
\usepackage{rotating}
\usepackage{psfrag}
\usepackage{epic}
\usepackage{eepic}

% Algorithm environment
%\newcounter{algorithm}[chapter]
%\renewcommand{\thealgorithm}{\thechapter.\@arabic\c@algorithm}
%\def\fps@algorithm{t}
%\def\ftype@algorithm{1}
%\def\ext@algorithm{loa}
%\def\fnum@algorithm{Algorithm~\thealgorithm}
%\newenvironment{algorithm}{\@float{algorithm}}{\end@float}
%\newenvironment{algorithm*}{\@dblfloat{algorithm}}{\end@dblfloat}
%\newenvironment{alevel}%
%   {\begin{list}{}{%
%      \setlength{\topsep}{0pt}%
%      \setlength{\parskip}{0pt}%
%      \setlength{\partopsep}{0pt}%
%      \setlength{\parsep}{0pt}%
%      \setlength{\itemsep}{0pt}}}%
%   {\end{list}}
%\newcommand\listalgorithmsname{List of Algorithms}
%\newcommand\listofalgorithms{%
%    \if@twocolumn
%      \@restonecoltrue\onecolumn
%    \else
%      \@restonecolfalse
%    \fi
%    \chapter*{\listalgorithmsname}%
%      \@mkboth{\listalgorithmsname}{\listalgorithmsname}%
%    \@starttoc{loa}%
%    \if@restonecol\twocolumn\fi
%    }
%\newcommand*\l@algorithm{\@dottedtocline{1}{1.5em}{2.3em}}

% List of figures and tables
\renewcommand\listoffigures{%
    \if@twocolumn
      \@restonecoltrue\onecolumn
    \else
      \@restonecolfalse
    \fi
    \chapter*{\listfigurename}%
      \@mkboth{\listfigurename}{\listfigurename}%
    \@starttoc{lof}%
    \if@restonecol\twocolumn\fi
    }
\renewcommand\listoftables{%
    \if@twocolumn
      \@restonecoltrue\onecolumn
    \else
      \@restonecolfalse
    \fi
    \chapter*{\listtablename}%
      \@mkboth{\listtablename}{\listtablename}%
    \@starttoc{lot}%
    \if@restonecol\twocolumn\fi
    }

% Math symbols, fonts, etc.
\usepackage{amsmath}
\usepackage{amsfonts}
\usepackage{amssymb}

\newcommand{\N}{\mathbb{N}}
\newcommand{\Z}{\mathbb{Z}}
\newcommand{\Q}{\mathbb{Q}}
\newcommand{\R}{\mathbb{R}}
\newcommand{\C}{\mathbb{C}}
\renewcommand{\S}{\mathbb{S}}
\renewcommand{\P}{\mathbb{P}}
\newcommand{\E}{\mathbb{E}}

\newcommand{\Cf}{\mathfrak{C}}
\newcommand{\Pf}{\mathfrak{P}}

\DeclareMathOperator{\sign}{sign}
\DeclareMathOperator{\avg}{avg}
\DeclareMathOperator{\floor}{floor}
\DeclareMathOperator{\ceil}{ceil}
\DeclareMathOperator{\round}{round}

\providecommand{\abs}[1]{\lvert#1\rvert}
\providecommand{\absd}[1]{\left\lvert#1\right\rvert}
\providecommand{\card}[1]{\lvert#1\rvert}
\providecommand{\norm}[1]{\lVert#1\rVert}

% URLs
\usepackage{url}
%% Define a new 'leo' style for the package that will use a smaller font.
\makeatletter
\def\url@leostyle{%
  \@ifundefined{selectfont}{\def\UrlFont{\sf}}{\def\UrlFont{\footnotesize\ttfamily}}}
\makeatother
%% Now actually use the newly defined style.
\urlstyle{leo}

\usepackage{thmbox}
\newtheorem[S,leftmargin=18pt,thickness=0.9pt,bodystyle=\noindent]{mydefinition}{Definition}[chapter]

% STUFF I ADDED:

\usepackage{amssymb}
\usepackage{graphicx}
\usepackage{float}
\usepackage{pgfplots}
\usepackage{pgfplotstable}
\pgfplotsset{
    % #1: index in the group(0,1,2,...)
    % #2: number of plots of that group
    bar group size/.style 2 args={
        /pgf/bar shift={%
                % total width = n*w + (n-1)*skip
                % -> subtract half for centering
                -0.5*(#2*\pgfplotbarwidth + (#2-1)*\pgfkeysvalueof{/pgfplots/bar group skip})  + 
                % the '0.5*w' is for centering
                (.5+#1)*\pgfplotbarwidth + #1*\pgfkeysvalueof{/pgfplots/bar group skip}},%
    },
    bar group skip/.initial=2pt,
    plot 0/.style={blue,fill=blue!30!white,mark=none},%
    plot 1/.style={red,fill=red!30!white,mark=none},%
    plot 2/.style={brown!60!black,fill=brown!30!white,mark=none},%
}
\usepackage{listings}
\usepackage{color}
\lstset{ %
  language=Octave,                % the language of the code
  basicstyle=\footnotesize\ttfamily,           % the size of the fonts that are used for the code
  numbers=left,                   % where to put the line-numbers
  numberstyle=\tiny\color{black},  % the style that is used for the line-numbers
  stepnumber=2,                   % the step between two line-numbers. If it's 1, each line 
                                  % will be numbered
  numbersep=5pt,                  % how far the line-numbers are from the code
  backgroundcolor=\color{white},      % choose the background color. You must add \usepackage{color}
  showspaces=false,               % show spaces adding particular underscores
  showstringspaces=false,         % underline spaces within strings
  showtabs=false,                 % show tabs within strings adding particular underscores
  rulecolor=\color{black},        % if not set, the frame-color may be changed on line-breaks within not-black text (e.g. commens (green here))
  tabsize=3,                      % sets default tabsize to 3 spaces
  captionpos=b,                   % sets the caption-position to bottom
  breaklines=true,                % sets automatic line breaking
  breakatwhitespace=false,        % sets if automatic breaks should only happen at whitespace
                                  % also try caption instead of title
  keywordstyle=\color{blue},          % keyword style
  commentstyle=\color{dkgreen},       % comment style
  stringstyle=\color{mauve}      % string literal style
}


% FUNCTIONS:

% retweet group
\newcommand{\rg}[1]{RG(#1)}

% retweets 
\newcommand{\rt}[1]{RT(#1)}

% retweet count
\newcommand{\rc}[1]{#1.\textrm{count}_R}

% expected retweet count
\newcommand{\ec}[1]{#1.\textrm{count}_E}

% follower count
\newcommand{\foc}[1]{\textrm{deg}^+(#1)}

% friend count
\newcommand{\frc}[1]{\textrm{deg}^-(#1)}

% author
\newcommand{\aut}[2]{#1.\textrm{author}_#2}

% set of followers
\newcommand{\fos}[1]{N^+(#1)}

% set of friends
\newcommand{\frs}[1]{N^-(#1)}

% immediate audience
\newcommand{\ima}[1]{\textrm{audience}(#1)}

% distinct audience
\newcommand{\dia}[1]{\textrm{distinct audienct}(#1)}

% interestingness score
\newcommand{\score}[1]{s(#1)}
\newcommand{\gscore}[1]{s_G(#1)}
\newcommand{\uscore}[1]{s_U(#1)}
\newcommand{\ascore}[1]{s_{\textrm{avg}}(#1)}

% score disparity
\newcommand{\disparity}[1]{d(#1)}
\newcommand{\gdisparity}[1]{d_G(#1)}
\newcommand{\udisparity}[1]{d_U(#1)}
\newcommand{\sdisparity}[1]{d_{\textrm{sel}}(#1)} % disparity between SELECTED tweets in question #1

% END OF STUFF I ADDED

\makeatother


\begin{document}


%%%%% FRONT STUFF	
\frontmatter

\begin{titlepage}

\begin{center}
\vspace*{3ex}
\textbf{\Huge Inferring Interestingness in Online Social Networks}\\[2ex]
\textbf{\large A thesis submitted in partial fulfilment}\\[1ex]
\textbf{\large of the requirement for the degree of Doctor of Philosophy}\\[16ex]
\textbf{\LARGE William M. Webberley}\\
\vfill
\textbf{\LARGE April 2014}\\
\vfill
\textbf{\LARGE Cardiff University}\\[1ex]
\textbf{\LARGE School of Computer Science \& Informatics}\\[4ex]
\end{center}

\end{titlepage}

\newpage\thispagestyle{empty}\cleardoublepage


\thispagestyle{plain}

\vspace*{6ex}

\textbf{\large Declaration}

This work has not previously been accepted in substance for any degree and is not concurrently submitted in candidature for any degree.\\[2ex]
Signed \dotfill \ (candidate) \hspace*{10em}\\[1ex]
Date\ \ \ \ \ \dotfill \hspace*{18em}

\vfill

\textbf{\large Statement 1}

This thesis is being submitted in partial fulfillment of the requirements for the degree of PhD.\\[2ex]
Signed \dotfill \ (candidate) \hspace*{10em}\\[1ex]
Date\ \ \ \ \ \dotfill \hspace*{18em}

\textbf{\large Statement 2}

This thesis is the result of my own independent work/investigation,
except where otherwise stated. Other sources are acknowledged by
explicit references.\\[2ex]
Signed \dotfill \ (candidate) \hspace*{10em}\\[1ex]
Date\ \ \ \ \ \dotfill \hspace*{18em}

\vfill

\textbf{\large Statement 3}

I hereby give consent for my thesis, if accepted, to be available for photocopying and for inter-library loan,
 and for the title and summary to be made available to outside organisations.\\[2ex]
Signed \dotfill \ (candidate) \hspace*{10em}\\[1ex]
Date\ \ \ \ \ \dotfill \hspace*{18em}

\vfill

\cleardoublepage


\thispagestyle{plain}
\ \vfill{\small
Copyright \copyright\ 2014 Will Webberley.\\
Permission is granted to copy, distribute and/or modify this document
under the terms of the GNU Free Documentation License, Version 1.3\footnote{http://www.gnu.org/copyleft/fdl.html} or
any later version published by the Free Software Foundation; with no
Invariant Sections, no Front-Cover Texts, and no Back-Cover Texts}\\[3.5ex]
\cleardoublepage


\ \vspace*{1.11cm}
\markboth{Dedication}{}
\begin{flushright}
%\textbf{\large To my parents, family and friends;}\\
%\textbf{\large To George, Pete,}\\
%\textbf{\large Jack, Max, Ella,}\\
%\textbf{\large Tom, and Charlie.}\\
%\large This is for their patience and support.
\end{flushright}
\newpage
\markboth{}{}
\cleardoublepage

\chapter*{Acknowledgements}
% Finish this section and tidy up, etc.

%This thesis and the research it contains would not exist if it weren't for the unfailing support of my supervisors, Roger Whitaker and, particularly, Stuart Allen. Their guidance and input have provided drive and shaped my research and have given me the confidence required in producing and defending ideas both within and outside of research. My utmost gratitude is extended to them for this and for their continuing encouragement at all stages of the PhD, always with fresh ideas and clarification for research pathways and for keeping me on track. I feel confident in saying that I was very lucky in being in their supervision and I would not be the same person, professionally and otherwise, today if it wasn't for their combined inputs over the past few years.

%I'd also like to lend my thanks to Martin Chorley for his continuously helpful and enthusiastic crowdsourcing expertise, without which the research in this thesis could not have been validated in the way it has. Matt Williams has provided insight throughout my time as a PhD student, from long sessions discussing the tiniest details to recommendations on research direction and scope. His ideas have helped shape many of the notions in this thesis. I am also deeply grateful to Chris Gwilliams for his inputs and support for everything; from where one might find the most effective M.O.T. deals through to how best to structure complex and fiddly data queries. He deserves far more than a simple acknowledgdment.

%Finally, my time as a research student would not have been the same without the constant presence and support from the other members of the MobiSoc group, the `superteam' and my non-academic friends. Gualtiero Colombo, Ian Cooper, Matt John, Liam Turner, Nick Sharp, and Ross Taylor have all provided assistance and I am very theankful to them for putting up with my worries and fusses without complaint.

%Thanks, all.



\chapter*{Abstract}
Information sharing on the Internet has given rise to the increased presence of non-interesting and `noisy' information in media streams on many online social networks. Although there is, however, a lot of `interesting' information also shared amongst users, the noise increases the cognitive burden in terms of the users' abilities to identify the interesting information and may affect the chance of the user missing important or interesting information. 
Additionally, users on such platforms are generally limited to receiving information only from those that they are directly linked to on the social graph, meaning that users exist within distinct information `bubbles', further limiting the chance of receiving interesting and relevant information from outside of the immediate social circle. 
In this thesis, Twitter is used as a platform for researching methods for deriving information ``interestingness'' through information popularity as given by the mechanism of \textit{retweeting}, which allows information to be propagated further between users on Twitter's social graph. Retweet behaviours are studied, and various properties are uncovered which relate retweet action to the social graph. This culminates in research into a methodology for assigning scores to Tweets based on their `quality', which is validated to perform well in various situations, and is analysed to demonstrate its potential in highlighting interesting information in mixed information streams.


\tableofcontents
\newpage
\chapter*{List of Publications}

Some of the work produced towards this thesis has also been published separately as follows.

\begin{itemize} 

\item \cite{webberley13} - W. Webberley, S. M. Allen, R. M. Whitaker. \textit{Inferring the Interesting Tweets in Your Network}, 
			in \textit{Workshop on Analyzing Social Media for the Benefit of Society (SOCIETY 2.0), 3\textsuperscript{rd} International Conference on Social Computing and its Applications (SCA)},
			Karlsruhe, Germany. \textit{IEEE 2013} 

\item \cite{webberley11} - W. Webberley, S. Allen, R. Whitaker. \textit{Retweeting: A Study of Message-Forwarding in Twitter}, 
			in \textit{Workshop on Mobile and Online Social Networks (MOSN'11), 5\textsuperscript{th} International Conference on Network and System Security (NSS)},
			Milan, Italy. \textit{IEEE 2011}

\end{itemize}

\listoffigures
\listoftables
\chapter*{List of Acronyms} 
\begin{acronym}
% to use in the thesis, use: \ac{OSN}
\acro{OSN}{Online Social Network}
\acro{MTW}{Mechanical Turk Worker}
\acro{URL}{Uniform Resource Locator}

\end{acronym}

\chapter*{Glossary} 

{\bf Audience}\\
The number of users that receive a given Tweet, either directly or as the result of retweets of that Tweet.

{\bf Author}\\
A user that has written a Tweet. The original author of Tweet $t$ is denoted as $\aut{t}{O}$. 

{\bf Follower}\\
A type of user. A user, \textit{x}, is a follower of user \textit{y} if user \textit{x} follows user \textit{y}. Other users who follow a particular user will receive all of the user's Tweets and retweets to their home timeline. A user can elect to follow another user. The subset of users that are followers of user $u$ is denoted as $\fos{u}$.

{\bf Friend}\\
The inverse of follower. User \textit{x} is a friend to user \textit{y} if \textit{y} follows \textit{x}. The subset of users that are friends of user $u$ is denoted as $\frs{u}$.

{\bf Local Graph}\\
The local graph of a user, $u$, is the subgraph of the full social graph (see below) representing the users and edges existing within $n$ hops of $u$.

{\bf Path-length}\\
The penetration of a Tweet - i.e. the number of times a Tweet is retweeted down one chain. The final retweeter in the chain indicates the number of hops the Tweet has taken from its author.
			
{\bf Retweet}\\
\textit{n.} - A replica of a Tweet, which has been forwarded on by a user (who is not the Tweet's original author) to their own followers. The set of retweets of a given Tweet, $t$, is denoted as $\rt{t}$.\\
\textit{v.} - The act of replicating a Tweet. A user who finds a Tweet interesting may retweet it so that it gains more exposure through an increase in the audience size (see above).
									
{\bf Retweet Group}\\
Set of Twitter users responsible for the propagation of a Tweet. Comprises the original author of the Tweet and the users which have since retweeted it. The set of members of a retweet group of Tweet $t$ is denoted as $\rg{t}$.

{\bf Retweet Count}\\
The number of times a particular Tweet has been retweeted. The retweet count of Tweet $t$ is denoted as $\rc{t}$.

{\bf Score Disparity}\\
The \textit{range} of scores assigned to a particular set of Tweets. Typically this is calculated by calculating the difference between the highest and lowest score of the set, and is denoted by $\disparity{T}$, where $T$ represents a set of Tweets. 

{\bf Social Graph}\\
The representation of users and the links illustrating relationships between them in real-world and online social networks.

{\bf Timeline}\\
A set of Tweets in Twitter in reverse-chronological order. A \textbf{user} timeline consists of that user's Tweets and retweets created by the user. A user's \textbf{home} timeline consists of that user's Tweets and retweets, the Tweets of each friend of the user, and retweets created by friends of the user. The home timeline contains all of the information that the user directly receives. 

{\bf Tweet}\\
\textit{n.} - A piece of information in Twitter; a piece of text, less than 140 characters long, which is written by a user. When sent, the Tweet is pre-pended to its author's user timeline and also to the home timelines of each of the followers of the Tweet's author. A Tweet is denoted as $t$.\\
\textit{v.} - The act of writing and sending a Tweet.\\ \\
\textit{Note -} A Tweet, in the context of Twitter, is treated as a proper noun and as such has its first letter capitalised\footnote{https://twitter.com/logo}.

{\bf User}\\
An account on Twitter. Each user (usually representing a real-life person or organisation) can Tweet, retweet, follow other users and be followed by other users. In this thesis, the terms \textit{user} and \textit{person} are occasionally used interchangeably.


%%%%% MAIN STUFF 
\mainmatter 

\chapter{Introduction}

Online social networks have exploded into the lives of millions of people worldwide over the last decade, and their use has dominated the communication highways and facilitated the interconnection of the world in ways never before perceived possible.

These social networks imitate real-world social networks. Although most such platforms each provide a different service to collaboratively satisfy an array of different use-cases, they tend to all be based around the idea of `friendships' (i.e. links between the user nodes in the social graph) and the sharing of information amongst friends.

Social networks like these have been available for around ten years now (with MySpace\footnote{http://myspace.com} launching in 2003 and Bebo\footnote{http://bebo.com} in 2005), but it wasn't really until Facebook's\footnote{http://facebook.com} worldwide launch in 2006 that social networks became the staple, ubiquitous norm that they are today. More recently, we have seen the introductions of Google's social network grown from its Buzz service, Google Plus\footnote{http://plus.google.com}, Pinterest\footnote{http://pinterest.com}, App.net\footnote{http://app.net}, and many more. They make up a large part of the basis and meaning behind the ideas of Web 2.0, which describes the web as being primarily formed from user-generated content and encourages the sharing of such content.

Another component that helped in the dawn of Web 2.0 was the rise of \textit{blogging}. A blog (`web-log') is a time-based series of posts consisting of continuous pieces of text, photos, or other media, and is generally contributed to by a single author. Blogs are often based around one or a set of topics and are usually public - meaning that they are written with the intention of being read by others. Despite this, they are often a way in which the author can look back at their history of posts, acting more as a diary recording snapshots of the author's life.\\
Various blogging services exist on the web today, such as Medium\footnote{http://medium.com}, Wordpress\footnote{http://wordpress.com}, and Tumblr\footnote{http://tumblr.com}.


\section{Twitter as a Social Network}
Twitter\footnote{http://twitter.com} is an online social network, which launched in the summer of 2006 \cite{krishnamurthy08}. Since then, it has rapidly gained in popularity amongst several different user groups - teens and young people, casual users, celebrities, reporters, and so on - and within eight months had around 94,000 registered users \cite{java07}. Although Twitter has never been a direct competitor with Facebook, users tend to use the two sites concurrently for different purposes: whilst Facebook's focus is on providing many services at once (such as photo-sharing, commenting/endorsing of information, messaging, pages for businesses, groups, events, etc.), Twitter's is more on simplicity.

More specifically than just being an online social network, Twitter is a microblogging website. Whilst a blog, as mentioned, typically contains long posts, Twitter only allows its users to post short pieces of text, up to 140 characters in length \cite{krishnamurthy08} \cite{huberman08}, called `Tweets'. Thus, Twitter is a hybrid social network and blogging service and whilst each Tweet may only realistically be able to hold a couple of sentences, this system facilitates quick, timely, and `real-time' \textit{live} information-sharing amongst its millions of users \cite{zhao09}. Its idea is that short pieces of news will `travel' faster and will be seen by more people more quickly than traditional news stories.

Although Tweets are limited to 140 characters in length, the inclusio n of URLs is allowed. This enables further extension of Tweets through external websites, and supports the inclusion of links to images and videos. Twitter has encouraged this use-case by providing `share' buttons for developers to embed in websites, and direct support for photo and video applications, such as TwitPic\footnote{http://twitpic.com} and Vine\footnote{http://vine.com}.\\
Its simplicity has also helped its growth into the mobile domain, in which smartphone users are able to very quickly post updates about their lives, a piece information they want to share, or a photo or video, and be able to post it \textit{as it happens} directly from the news source or geographical location \cite{castillo11}. This has been especially useful in emergency situations worldwide, including the Haiti earthquake in 2010 \cite{muralidharan11}, and 2011's Egyptian protests \cite{wilson11}, and Thai flood \cite{kongthon12}.

Use of Twitter is based around `timelines' of Tweets, to which new Tweets are pre-pended as they arrive. The \textit{home} timeline is the default view, in which Tweets from all of a user's subscribed users are placed. Timelines of an individual user contain only Tweets from that user, and are known as a `user' timeline. Customisation of timelines is also possible through the usage of Twitter lists, which can have users placed within to categorise different streams of Tweets.


\section{Twitter's Social Graph and Information Subscription}
As with many social networks, the structure of Twitter lies within the users and their connectivity within its social graph. However, unlike Facebook, whose social structure is made up of bi-directional `friendships' between users, Twitter's primary social graph is made up of \textit{directional} links between its users \cite{edwards13}. A person using Twitter can elect to \textit{follow} another user, which \textit{subscribes} the person to receive \textit{all} of that user's Tweets to their home timeline. The set of users that follow a person are known as that person's \textit{followers}, and the set of users that the person follows are the person's \textit{friends}.

Whilst bi-directional links are common amongst communities of similar interests, friends, colleagues, and so on, single-directional links are found more in situations in which less-influential users follow more-influential users, such as celebrities.


\section{The Problem}
A user who follows a set of other users can \textit{generally} be said to find that set of users to produce more interesting information than those users that the user does not follow. However, despite that, not \textit{all} information produced by an `interesting' user is likely to be interesting, and yet \textit{all} information produced by a Twitter friend will be received onto the home timeline.\\
Noise is a common problem in Twitter, and is the uninteresting information one might receive that conveys little interest. It is likely that most of the information received on Twitter \textit{is} noise, and this makes it very hard to distinguish the interesting information from the uninteresting.\\
Since people tend to use Twitter for short sporadic moments and do not have time to filter out noisy information, the presence of noise can dampen the experience of the user, making it much more difficult to find interesting information.

In addition, Twitter users typically exist within an information `bubble'. This is similar to the notion of the Google search bubble, in which Google uses previous results and searches to only return information to a user based on what \textit{it thinks} the user would find the most interesting and useful. This results in the users not knowing which information exists beyond the confines of their bubble, and if they do not know it exists, they cannot know if it is of interest to them. Similarly, a Twitter user cannot follow all of the users he/she may find interesting, since he/she will not \textit{know} of all the interesting users existing on the social graph.

How can users be exposed to \textit{interesting} and \textit{relevant} information, but without them having to know about it or look for it first?


\section{Contributions}
This thesis focuses on understanding information propagation, and how this combined with knowledge of the social structure of Twitter can assist towards solving the problem of identifying interesting and relevant information on Twitter. 
Whilst other work in the area has also looked into the notions of relevance and interest in online social networks, and Twitter in particular, none has addressed the problem in such a way as this.

Part of the outcome of this research are methods for effectively inferring interesting information and, indeed, ranking information by interestingness. The methods are validated in various ways to help highlight their strengths and weaknesses in detecting interestingness in different ways.

The work addresses the problem area in that it helps towards solving the goal of identifying \textit{globally} interesting information in Twitter. In addition, certain measures are taken in an attempt to address the idea of information relevance, which denotes how information interestingness is subjective, and thus different from user to user.


\section{Thesis Structure}
The rest of this thesis is structured as follows.

Background information to provide an introduction to some of the ideas behind the main research immediately follows this chapter, which in turn is followed by a review and assessment of relevant literature in the field of electronic information propagation and information interestingness.\\
The next three chapters contain research on Twitter's information propagation characteristics and its interesting and useful behaviours, the social structure of Twitter and the ways in which this is important for understanding the spread of information, and then on the research of the methodologies themselves, including validation and analysis of the results of this work.\\
The thesis ends with a general analysis and conclusion, and a discussion of potential future work in this area.
\chapter{Background and Research Domain}


One of the most widely-used features of Twitter is its inbuilt function for easily facilitating the spread of information through its social structure. This phenomenon is the basis for much of the research in this thesis and, when combined with the characteristics of Twitter's user graph, has many interesting attributes and behaviours associated with it.

\section{Key Concepts}
Although the main important concepts are described briefly in the Glossary of this thesis, in order to more clearly understand the research and components of Twitter they are explained in more depth in this section.

\subsubsection{Tweet}
A Tweet is a singular message on Twitter that cannot exceed 140 characters in length. It is written and sent by a user, who is the Tweet's author. Figure \ref{retweet_button} shows a Tweet written by a user, which is pre-pended to the author user's user timeline and to all of the author's followers' home timelines. Tweets can be retweeted, as explained later in this chapter, in order to create a retweet. The term `Tweet' can also be a verb defining the act of sending a Tweet.

\begin{figure}[h]
    \begin{subfigure}{.5\textwidth}
        \centering
        \includegraphics[scale=0.4]{2.Background/Media/user_timeline.png} 
        \caption{The user timeline for user \texttt{@BBCBreaking}.}
        \label{fig:user_timeline}
    \end{subfigure}
    \quad
    \begin{subfigure}{.5\textwidth}
         \centering
        \includegraphics[scale=0.42]{2.Background/Media/home_timeline.png} 
        \caption{A home timeline.}
        \label{fig:home_timeline}
    \end{subfigure}
    \label{fig:timelines}
    \caption{Examples of user and home timelines.}
\end{figure}

\subsubsection{User Timeline}
The user timeline of user $u$ is a set containing the Tweets written by $u$ and any retweets made by $u$ and is ordered by time with the most recent Tweets at the top.  Figure \ref{fig:user_timeline} illustrates this through the example of BBC Breaking News' user timeline. Since user timelines also contain retweets, not all Tweets on $u$'s user timeline are necessarily authored by $u$..

\subsubsection{Home Timelime}
The home timeline of user $u$ is a set containing the Tweets written and retweeted by $u$ as well as all of the Tweets and retweets made by his/her friends, where a friend of $u$ is another user that $u$ follows. \ref{fig:home_timeline} shows an example of a home timeline. As such, a home timeline is the union of the user timelines of $u$ and all of $u$'s friends, again with the most recent Tweets nearer the top.

\begin{figure}[h]
\centering
\includegraphics[scale=0.6]{2.Background/Media/notifications_page.png} 
\caption{A notifications page.}
\label{fig:notifications_page}
\end{figure}

\subsubsection{Notifications Page}
The notifications page for a user $u$ is a page that only $u$ can access, and which lists events involving him/her. For example, this may contain new Tweet mentions (Tweets that contain $u$'s username) and new followers, as demonstrated by Figure \ref{fig:notifications_page}.

\begin{figure}[h]
    \begin{subfigure}{.5\textwidth}
        \centering
        \includegraphics[scale=0.4]{2.Background/Media/following_list.png} 
        \caption{A friends or `following' list.}
        \label{fig:following_list}
    \end{subfigure}
    \quad
    \begin{subfigure}{.5\textwidth}
         \centering
        \includegraphics[scale=0.4]{2.Background/Media/followers_list.png} 
        \caption{A followers list.}
        \label{fig:followers_list}
    \end{subfigure}
    \caption{Examples of friends and followers lists.}
\end{figure}

\subsubsection{Friends List}
The friends list of user $u$ is the set of users that $u$ follows ordered with the most recently-made friendships nearer the top. Therefore, the first user $u$ followed is at the bottom of his/her friends list. Figure \ref{fig:following_list} shows an example of a friends (or `following') list.

\subsubsection{Followers List}
The followers list of user $u$ is the set of users that follow $u$. Again, this is ordered with the most recent follower at the top. An example of a followers list is given by Figure \ref{fig:followers_list}.


\section{Domain Introduction and Literature Survey}
This section provides an understanding of retweeting in Twitter and its effects on the users and the followships between them, which represent the underlying social graph of Twitter, and includes a survey of some of the most relevant works of the area.

\subsection{Information Propagation through Retweeting}
The function of propagation in Twitter is known as \textit{retweeting}, and is carried out by the Twitter users themselves. When a user views a Tweet that they believe to be particularly interesting, and believe it to also be interesting to his/her followers, then he/she can elect to retweet it, and thus pass it further through the social graph to that user's followers also. A Tweet that has been retweeted is known as a \textit{retweet}, and it is clear that a Tweet which is retweeted will be made available to significantly more users than a Tweet that isn't retweeted \cite{webberley11, kwak10}. Since Twitter's social graph is decentralised and retweeting occurs between individual groups of users, its properties are similar to information dissemination in other types of decentralised graphs, such as content-forwarding in opportunistic networking \cite{allen10}.

A retweet can be carried out in one of two ways: either through the use of Twitter's native retweet `button', or manually. The button is displayed along with each Tweet (see Figure \ref{fig:retweet_button}) in a timeline which, when clicked, immediately creates a new retweet containing the verbatim content of the original Tweet and automatically sends it on to the retweeting user's followers. The user who created the original Tweet is credited as the author on the recipients' timelines, with an indication of who carried out the retweet itself. Thus, users who follow the retweeter will see a Tweet appear in their home timeline from someone that they may not directly follow. Figure \ref{fig:northern_lights_tweet} illustrates an example; the user receiving the depicted Tweet does not follow the original author, \texttt{@aldakaida}, but \textit{does} follow \texttt{@DTW\_Holidays}, who was responsible for carrying out the retweet. 

\begin{figure}[h]
\centering
\includegraphics[scale=0.3]{2.Background/Media/northern_lights_tweet.png} 
\caption{A retweeted Tweet.}
\label{fig:northern_lights_tweet}
\end{figure}

The manual approach involves physically copying the content of the Tweet to be retweeted and pasting it into a new Tweet, usually with the text `\texttt{RT @<username>:}' pre-pended, where \texttt{RT} stands for \textbf{r}e\textbf{t}weet and \texttt{<username>} is the username of the author of the original Tweet. This method allows for annotating the original content of the Tweet (for example, to provide an opinion on the Tweet contents), producing a \textit{modified} Tweet, which can sometimes be pre-pended with \texttt{MT} rather than \texttt{RT}.

Historically, this manual approach was informally community-driven by the users of Twitter and was the only method available for carrying out retweets. However, in 2009, Twitter realised the popularity of this user convention and introduced the retweet button\footnote{https://blog.twitter.com/2009/project-retweet-phase-one} in order to assist in this trend, and through which Tweets could be retweeted much more quickly and accessibly. The button is implemented on Twitter's website and mobile device applications, yet even today the original and manual retweet approach remains popular amongst many of Twitter's users and communities.

\begin{figure}[h]
\centering
\includegraphics[scale=0.3]{2.Background/Media/retweet_button.png} 
\caption{The retweet `button' in context.}
\label{fig:retweet_button}
\end{figure}

Each Tweet has a retweet count associated with it, which is the raw representation of the number of times that the Tweet has been retweeted using the retweet button method. Since the manual retweet technique remains more community-driven, there is no official way to include these as part of the retweet count of the original Tweet. However, since the manual method is typically only used with the aim to annotate or modify the Tweet in some way, the resultant `retweet'  is no longer a real representation of the content of the original Tweet, and so should not be counted as such.

It should be noted that Twitter users may choose to make their account `protected'. A person who has a protected account will still have a publicly-visible profile (displaying a name, username, bio, and so on), but their Tweets and other information (such as the followers and friends lists) are hidden from users that aren't followers of the person. Potential followers of a protected account must \textit{request} a followship, which can then be accepted or rejected by the protected account holder. Since Tweets from a protected account are only visible to approved followers, the retweet button is unavailable for them to disseminate the Tweet any further than the author's immediate local follower network. However, since the manual retweet method does not rely on the button and isn't governed by Twitter, a protected account's Tweets can still be retweeted in this way.

As Facebook supports the endorsement of information found on its site by inviting users to `like' a piece of content, retweeting is effectively a \textit{vote} or endorsement for a Tweet on Twitter. In both cases, the number of likes and number of retweets is available to the platforms' respective users (Figure \ref{fig:tweet_comparison}), and so this provides some insight into the \textit{popularity} of the information. Whilst Facebook `likes' are immediately visible to users in feeds, the retweet count becomes visible once a user clicks a Tweet to expand its metadata.

Some Twitter users declare that their `retweets are not endorsements' in their profile's bio. This particular behaviour is largely associated with journalists who use retweets to highlight potentially controversial Tweets and to ignite discussion over certain topics. This declaration is unwelcomed by some users, who then argue the point by asking what retweets \textit{are} meant to imply\footnote{http://www.poynter.org/latest-news/media-lab/social-media/152448/the-problem-with-retweets-how-journalists-can-solve-it/} and that a user's bio is not immediately available for recipients to view in Tweet streams. Generally, this is not very widespread, and even Tweets retweeted with the aim of not being an endorsement imply some level of \textit{popularity} of the Tweet. 


\subsection{Retweets and the Social Graph}
The social graph of Twitter can be represented, like other online social networks, by edges between users, partially emulating real-life social interactions between humans. The growth of social media has encouraged more dense communication between users all over the world, who would not previously be able to be in direct contact with one another in this way.

Stanley Milgram's ``six degrees of separation'' \cite{milgram67} experiments are highly relevant to and useful for OSN research today. The results of the experiments demonstrated that people are usually no more than six hops away from each other on the world's social graph, yet this value was found to be an overestimate when it comes to the analysis of the structure of OSNs by \citet{backstrom11}, who found that the average `distance' observed in Facebook's entire 721 million-node graph in 2011 was only around 4.7 hops. This implies that denser links between users and larger communities that apparently manifest themselves in OSNs create a smaller `world' than that experienced in reality.

In each of Milgram's experiments participants passed a message to one another, at each stage only passing to other people that they actually \textit{know}, in the hope of it reaching a single intended recipient. This meant that people could use acquaintances in other geographic locations to transfer the message from community to community. Twitter supports a similar propagation mechanism in the fact that retweets can themselves be retweeted; this is a focus of some of the earlier research in this thesis.

This behaviour provides further penetrative `depth' of the information through the social network away from the source user in addition to the spread `width' made by the initial retweets. Although retweeting is not carried out with the aim of information reaching any particular final user (or set of users), as with Milgram's experiment, this phenomenon allows retweets to `travel' between `online communities' of users.

As with real-life social networks, communities of users in OSNs are also a common feature \cite{ugander11}. Social communities are identified by clusters of people sharing dense links with each other, and can vary in size, location, and geographic spread. They often grow over time and are formed between people living near to each other or between groups of users with shared interests.

In Twitter, these communities are typically small to begin with and are based on a topic of interest or around a more influential user. As Tweets are produced from within the community, further links are made to interconnect the community's users, producing a growing `swarm' of interest around the initial topic or user \cite{java07}. As further users begin associating themselves with this community, its audience becomes widespread and the community grows. This concept is discussed in greater length by \citet{java07}, who described them as compact groups of users connected by dense follower links after further experimentation.

\begin{figure}[h]
\centering
\includegraphics[scale=0.75]{2.Background/Media/tesco_mobile.png} 
\caption{Phatic Tweets from Tesco Mobile's official Twitter account.}
\label{fig:tesco_mobile}
\end{figure}

Some users try and build communities through \textit{phatic} communication, in which the Tweets being broadcast are `non-dialogic' or `non-informational', as defined by \citet{miller08}. This is often done with the aim of providing a relaxed environment, and could be useful for cases similar to an internal Twitter account for an organisation's employees. Additionally, some commercial organisations may use phatic language in order to engage more with the community for the purposes of appearing more approachable, such as Tweets from Tesco Mobile's official Twitter account, as demonstrated by Figure \ref{fig:tesco_mobile}. 

\begin{figure}[h]
\centering
\includegraphics[scale=0.7]{2.Background/Media/communities.png} 
\caption{A hypothetical group of user communities.}
\label{fig:communities}
\end{figure}

In more dense communities, Tweets can be made available to many users immediately after they are published, since many of the links between users are shared. This means that any retweets that occur within communities are likely to have a lot of \textit{redundancy}, in that many of the retweets will be sent to users who have already seen the original Tweet. Twitter prevents this information duplication by not displaying the retweets of Tweets that have already appeared on a user's timeline. However, this does increase the chance of the Tweet being propagated to users external to the community.

Retweets amongst users within a community are likely to be common, due to the shared-interest nature of communities, and some users can provide `bridges' by being active in more than one group. In these cases, Tweets can be passed between the communities through retweets by the bridging user. For example, Figure \ref{fig:communities} illustrates three hypothetical communities `bridged' by one user in community B. If this user was to retweet a Tweet from communities A or C, then it is clear how the Tweet could be propagated from one group of users to another. If there are many users sharing communities then there are many more avenues available for dissemination through the graph, causing a high level of information throughput. If there are fewer bridges, then there is more of a bottleneck between the communities, hindering the information spread.

\citet{java07} also find that communities can be formed from different types of people, such as those who Tweet frequently and have many followers, and those who contribute very little and have few followers. Those with many followers and many friends receive lots of information and have the potential to spread information further than those with fewer inward and outward edges. Studies into the behaviour of different types of user in Twitter is done more thoroughly by \citet{krishnamurthy08}, who define `broadcasters' (users with many followers and few friends) and `miscreants' (users with few followers but many friends) and their roles in information propagation. The authors desribe broadcasters as users who post many Tweets, which are then received by large numbers of users, and miscreants as those who receive lots of information but are unable to achieve a large Tweet audience. It is therefore assumed that broadcasters are likely to be retweeted more than many other types of user.

Users that retweet the interesting information from a source user to others, who do not follow the source user and so would not naturally receive the information, are effectively acting as information \textit{filters}. By not following the source user, a person might still receive the interesting information through these filters, but will not receive any of the `noise'. Thus retweeting means that friends of a user become useful filters of information for users further `downstream'  and retweeted information can be said to have a higher \textit{credibility} than Tweets that aren't retweeted \cite{castillo11}.


\subsection{User Influence}
Just as there are different types of user \textit{behaviours} on Twitter, there are also users of different \textit{influence} levels \cite{quercia11}. Much research has gone into user influence, including on how this might be detected \cite{yu11}, and influential users are generally found to be those who have a greater impact on Twitter's social network \cite{bakshy11} and who usually have significantly more followers than an average user. Influential users tend to have a high persuasion over other users, relating \textit{influtentials} in Twitter to those who are also influential in the real world as part of traditional communication theory \cite{cha10}, and therefore many Twitter influentials are the accounts belonging to real-world celebrities.

As with real-world celebrities, Twitter influentials are those with many `influenced' followers, or fans, which are the users who have the strongest agreeable opinions of the influential. As a result, an influential user has a greater number of followers who are interested in the information produced by the user, and is therefore more likely to receive more retweets than less influential users.

Although influence level is partly derived from the follower count of the user, it should be noted that a user with high in-degree on the social graph\footnote{In-degree: many followers} does not necessarily imply a high level of influence. An `active' audience of users who reply, retweet, and interact are more indicative of an influential user \cite{bigonha10}. This is especially true since a user can gain more followers through campaigns such as `\#teamfollowback'\footnote{Users associate themselves with \#teamfollowback to imply they will return all followships.} or by following `out of politeness', in which a user will follow another user back as an act of politeness, but these users tend to have \textit{both} high in- and out-degree and invoke less interactivity amongst their followers, which are not necessarily characteristics of an influential user \cite{cha10}.

Klout\footnote{http://klout.com} is a web service that attempts to review a user's social media influence by assigning users a Klout Score. Their website declares that this score, which ranges from 0 to a maximum of 100 and whose generation algorithm is kept private and unpublished \cite{edwards13}, is determined from a variety of over 400 `signals' taken from eight different social media platforms. These signals are derived from various attributes including the volume of information shared, the reaction to the shared information, and the relative scores of the users who interact with the information. They \textit{also} take interactivity between users as one of the primary indicators \cite{anger11}. Additionally, the service indicates the topics a user is influential about, with the general idea being for organisations to check up on which users are influential for marketing purposes, but also to highlight the users that should be replied-to at a higher priority.


\subsection{Twitter as an Information-Retrieval System}
At a high level, Twitter can be considered a type of information-retrieval system, which people can utilise to produce and consume information when required. In traditional information-retrieval systems, such as search engines and library systems, keywords and search terms are common ways for describing the type of information the user would like to receive. The system would then search a database or archive for what it believes is relevant information, \textit{based} on these `retrieval parameters', and return results to the user ordered usually by the estimated relevance of the articles \cite{arvola10}.

Information quality is also reliant on the expected reading effort of the returned documents. The character precision-recall metric was introduced by \citet{arvola10} by way of demonstrating the tolerance-to-irrelevance ratio. The general mechanism for this ratio is to do with users reading a document passage; the point at which this ratio is reached is when the user stops reading the particular passage and moves to the next whole document, since they assume the rest of the document is also irrelevant to them. Therefore, the more effective the information retrieval system is in displaying high-quality information, the lower the chance that this ratio is reached by the user. This is comparable to the event in which a Twitter user viewing Tweets from someone they are following gets to the point where he or she reaches this ratio (i.e. is beginning to get bored or find the Tweets irrelevant) and decides to unfollow the friend. Similarly, the more effective the user is when selecting people to follow in the hope of receiving interesting information, the less likely it is that the user will remove these friends.

Whilst Twitter does not support the use of keyword searching for its primary information delivery method, it does lend its users some control over the type of information they wish to receive. As mentioned previously, users receive all of the Tweets from everyone that they follow onto their home timelines. Thus, by selecting users to follow, a person is effectively describing and implicitly indicating the type of information he/she would like to receive, and by editing their friends list (either by adding new followers or pruning existing ones) he/she can alter this indication. In addition, Twitter moves towards an information-\textit{recommendation} system, in that the friends of a user can endorse and imply a Tweet's quality by retweeing it.

Despite this control, it is still unlikely that users will achieve a perfect Twitter experience due to the presence of \textit{noise} \cite{alonso10}. As discussed in the Introduction, this problem stems from that although a person follows users they consider to be interesting, it is often the case that not \textit{all} information produced by interesting users will be interesting itself. For example, a user may follow a news source in order to receive breaking news but is not interested in viewing the information about politics or celebrity updates that the news source also Tweets about.

\begin{figure}[h]
\centering
\includegraphics[scale=0.75]{2.Background/Media/eso.png} 
\caption{Example of a Tweet from a `dynamic' Twitter account.}
\label{fig:eso}
\end{figure}

Twitter's information-retrieval characteristics are also very valuable in highly dynamic scenarios, where status updates regarding rapidly-changing circumstances are very useful. There are some Twitter users that might use the service in this way very often, for example a user who may post updates on server availability for an online game, as shown in Figure \ref{fig:eso}, or a user who broadcasts Tweets on the state of transport links, such as the London Tube.

In addition to `permanent' dynamic behaviour, environments of uprising and political tensions, such as the `Green Revolution' during the 2009 Iranian elections \cite{burns09}, clearly demonstrate Twitter's important role in the live updating of information and news in difficult situations. The examples nearer the start of the Introduction chapter also help illustrate this.



\subsection{Information Quality, Popularity and `Interestingness'}
Generally, interestingness is almost impossible to define, since it is affective, subjective, depends on relevance, and often relies on the context of surrounding information. In addition, `interesting' content is not simply that which someone enjoys reading, since an article that conveys anger or frustration is also of interest.

As such, in this thesis, interestingness is the term used to describe information that is `not noise', and contains some amount of generally (or `globally') interesting content in that it is \textit{generally} interesting, relevant, and conveys `affective stimulation' to people and is identifiable from the noise surrounding it. More detail on affective stimulation within the context of information searching is described in Section \ref{section:affective_stimulation}. 

Information-retrieval systems typically use some measure of information quality when determining which documents to return to a user and also when deciding on the order the documents should be displayed in. This `quality' is subjective in that different systems use a variety of different algorithms for deducing quality, usually based on the level of interest in each of the available documents (such as Google's Page Rank algorithm and Amazon's recommendation algorithms), but also in that the level of quality itself depends on the user him/herself requesting the information. 

In the case of Google's Page-Rank, the algorithm uses multiple cues to determine who the user is, their interests, past searching habits, links clicked, and so on, to return \textit{relevant} information, which is incidentally one of the causes of the aforementioned Google search bubble. Amazon's recommendation algorithms analyse a user's past item views and purchases and cross-matches these against trends based from users who also looked or bought similar items. Amazon is then able to accurately determine the type of items a customer are interested in purchasing, and can send emails to that customer with personalised recommendations. In these cases, information quality is essentially a function of information interestingness and information relevance.

Twitter uses no such metrics to deliver information to its users, relying on the users themselves to implicitly `choose' the information they want to receive. Additionally, information is always displayed in chronologically-ordered timelines, with new Tweets being continuously inserted at the top as they occur. Twitter does not try to indicate interesting Tweets on the timeline which means that the interesting information is shown at equal value alongside the `noisy' Tweets, causing the difficulties in identifying the interesting information as has been mentioned previously. Indeed, the recent TechCrunch article from October 2013, ``Twitter Quitters And The Unfiltered Feed Problem''\footnote{http://techcrunch.com/2013/10/05/sorry-my-feed-is-full} talks at more length about this particular phenomenon, and helps highlight the problem area of this work more clearly from a layperson's perspective.

The retweet count of a given Tweet is a useful metric in inferring its \textit{popularity}. If a Tweet is retweeted 10 times, then ten people have taken the time to read that Tweet, decide it is worth sharing, and then actually retweet it \cite{uysal11}. This user (and the other nine retweeters) may have found the Tweet interesting, yet it should be noted that although the count can be used as a measure of popularity, as a function of the influence of the Tweet's author, the retweet count alone cannot be used as a measure of how interesting the Tweet actually is \cite{naveed11}. For example, it is inappropriate to say that the first Tweet in Figure \ref{fig:tweet_comparison} is so significantly more \textit{interesting} than the second, although it is clearly more popular since Justin Bieber is an extremely influential Twitter user. Indeed, although Justin Bieber's Tweet is clearly of `interest' to his fans, it is not necessarily the general case. 

\begin{figure}[h]
\centering
\includegraphics[scale=0.55]{2.Background/Media/compared_tweets.png} 
\caption{Example of Tweets with significantly different retweet counts.}
\label{fig:tweet_comparison}
\end{figure}

Whilst the work in this thesis does not primarily aim to build an accurate retweet-predictor, this does become more important in some of the work in later chapters, since it forms part of the basis of the methods of interestingness inference as a function of many retweet \textit{decisions}. \citet{uysal11} also identify the problem of `noisy' Twitter timelines and discusses methods for predicting \textit{popular} Tweets using a J48 decision tree classifier, based on the likelihood of the Tweet being retweeted by a particular user. Although the authors address information relevance from a user-centric point of view, the validation of whether a prediction of a retweet occurring for a given Tweet is actually indicative of the \textit{interestingness} of said Tweet do not perform particularly well.

A retweet-prediction model based on a factor graph model is introduced by \citet{yang10} to determine how retweetable a Tweet is on a global scale. A precision of just under 29\% is achieved in predicting if a Tweet will be retweeted, but no mention is made of how this relates to how \textit{interesting} the information is. Another study into retweet prediction was carried out by \citet{zaman10}, in which a trained probabilistic collaborative filter model (named `Matchbox') was used to determine the useful features in making the predictions. As with the previous study, the research focuses on a retweet \textit{probability}, which is a binary decision made by one particular user. The methodology is not aimed at the inference of interestingness, and simply determines that the most relevant features for accurate decision predictions are the author of the original Tweet and the retweeter.

Inversely, \citet{suh10} and \citet{hong11} predict the \textit{type} of messages that are likely to be retweeted further, the latter using a logistic regression to both predict an individual retweet decision and a retweet \textit{volume}. The methods do not apply these notions to how interesting the information actually is to a particular user, achieve low recall and the multi-classifications seems only to perform well on very unpopular or very popular Tweets. It is made clear, however, that the retweet volume of a Tweet is useful in denoting Tweet \textit{popularity}.

\citet{petrovic11} use a passive-aggressive machine-learning algorithm to make binary predictions on retweet decisions and cited that social features - for example, number of followers of the author, frequency of Tweeting, etc. - were the largest factors in the performance, and \citet{naveed11} use a logistic regression, partly using a dataset published as part of another paper by the same authors as \citet{petrovic11}, to predict retweet decisions in order to address information interestingness. However, little effort is made to define interestingness or, indeed, validate that the inferences towards this are accurate and correct. A logistic regression is again used by \citet{zhu11} for predicting binary retweet behaviours with the focus on information propagation in disaster scenarios, and \citet{peng11} showed that conditional random fields can perform better than logistic regressions when modelling retweet behaviour in the same way.

Since the above papers only effectively consider a prediction of retweet outcome, which is a binary decision, it is hard to relate this more to the notion of global interestingness, aside from stating that a retweet implies the retweeter's relative interest in the Tweet. However, a retweet count, as mentioned above, is inappropriate as an indicator of \textit{magnitude} of interest, and so the research into predicting individual retweet decisions cannot be used as a basis for this. Additionally, not much emphasis is placed on how well the techniques work on a more `on demand' basis; many of the methodologies discussed require several features that may take a long time to collect and compute, making them unsuitable for use as part of quick and useful interestingness evaluations.

The idea of Tweet scoring (through more semantic analysis) and retweet \textit{count} predictions is introduced by \citet{gransee12}, who used their methodologies to produce a system\footnote{https://sites.google.com/site/learningtweetvalue/home} enabling users to compile Tweets in ways that are predicted to achieve the most retweets. The predictions are based on averaging the score, derived through a linear regression, of different components of a user's Tweets (such as the inclusion of a particular hashtag), so that when a Tweet by the same author is next constructed, the various components of the new Tweet can be compared against the scores of the counterparts seen in previous Tweets. The value produced through this method is then used to generate an expected retweet count as part of a comparison to the user's average (`baseline') achieved retweet count at this point in time, and was shown to perform well on influential Twitter users.

However, the methods described do not take into account fluctuations in the social graph, particularly in the case of less-influential Twitter users, who's local networks are prone to more frequent changes. Additionally, they rely on a significant amount of previous Tweet and temporal information on the user to be evaluated, and do not relate the resultant score to any type of interestingness metric in the context of highlighting it from amongst noise.

\citet{alonso10} also use `scoring' to address interestingness, again focusing more on semantics through determining \textit{uninteresting} content. In this work, Tweets are assigned an integer score out of five. Although the authors initially attempted to train a decision tree classifier on a set of 14 features, they began classifying a Tweet as `possibly interesting' if it contains a URL, and otherwise classify it as `not interesting'. Although the authors did then further classify the possibly interesting Tweets, by studying the magnitude of the crowdsourcees used to evaluate the Tweets that found them interesting, and then classifying Tweets based on them containing a particular type of named entity (for example, a person's name, a place or brand name, and so on) the categorisation system is too coarse and is not capable of representing the many different types of Tweets seen on Twitter. Additionally, despite achieving relatively high accuracy in this particular area, the methods are not suitable for assessing Tweets on a general or user-specific level, especially since Tweets that don't contain URLs might still contain interesting content.

An interesting study is described by \citet{lauw10}, in which a clustering algorithm is used, taking into account the retweet count of a Tweet and how this is related to the popularity of the source user, to determine information quality. Although this work is more similar to the research discussed later in this thesis than others, the scoring is quite simple and the author's use-case seems limited to that of identifying the most important Tweets surrounding a particular event (such as the death of Michael Jackson). Additionally, the authors do not make any effort to verify their results in any way, aside from comparing the Tweets determined to have a high quality by each of their two assessed methodologies.



\subsection{Twitter is a `Memepool'}
A large amount of researcht in this field, particularly in the case of the work involving machine learning and classification, as discussed above, reles on feature selection and extraction. By choosing an appropriate set of features that are able to represent the entity more accurately, then this enables the model produced from the features to have a greater classification performance.  

In 1976, Richard Dawkins coined the term `meme' to be defined as a ``unit of cultural transmission'' \cite{dawkins76}. The general idea behind memetics is as an analogy to biological genetics except, unlike genes, memes are entirely non-physical and represent a cultural idea or aspect or another human-based behaviour. The rise of social networks on the Internet has allowed the spread of memes to grow to the extent that they are sometimes now even represented by `physical' constructs, such as images.

In genetics, a gene is a physical entity containing information and instructions. It is a unit of genetic inheritance, in that they are passed from parent to offspring through the act of reproduction, and the result of an organism having a gene is that the organism will express the features represented by that particular gene. These genes contain instructions that make up the features of an individual, such as physical characteristics like eye colour and height, and non-physical characteristics, including various aspects of personality.

Organisms exist in an evironment that also has features, such as humidity, altitude, temperature, relationships to other organisms, and so on. If the genes of an organism are such that they cause the individual to be well-suited to its environment, then that organism has a better chance of survival and, therefore, a better chance of achieving reproduction.

Memes are similar in that they are effectively made up of a set of features, or a `memome', such as the wordings of a particular phrase, or their relevance to other cultural aspects. These enable the meme to be less or more likely to be replicated in different environments, which is made up of the humans exposed to it and the interactions between them. For example, an Internet meme relating to the Star Wars movies would likely have a greater chance of being reproduced, through discussion and reposting, in an environment comprising a set of science-fiction fans than when amongst more mixed-interest groups.

The meme is also a useful analogy in this thesis when describing the way in which Tweets undergo replication within Twitter and for feature selection. Like a meme, a Tweet has a specific set of features, such as the text it contains, the inclusion of any mentions or a URL, and so on, and it exists within an environment consisting of a set of interconnected users on the Twitter social graph. A particular Tweet would generally have a greater chance of `surviving' and being replicated, through the act of retweeting, amongst a certain subset of users intereconnected in a particular way than in other environments.

As such, the Tweet features are analogous to the \textit{genes} of a genome, and the arrangement and type of users on the social graph that receive the Tweet and have an opportunity to assist in its propagation comprise the Tweet's \textit{environment}. Both of these aspects are of importance and are considered as part of feature selection in the relevant parts of this thesis.


\subsection{Precision and Recall}
Precision and recall are two metrics that are often used simultaneously to verify the performance of a method or procedure for information retrieval, with the usual goal being to maximise both. The metrics are used for validating \textit{accuracy} in different ways, yet they can be applied to other purposes also and are useful in describing the notion of interestingness in Twitter.

``Classic'' precision and recall are derived from the ratios of relevant documents to non-relevant documents and consider also the relevant documents that \textit{aren't} retrieved by the system for a given search query. In particular, precision is the proportion of documents retrieved that are relevant, and recall is the proportion of relevant documents that were retrieved;

\[	
	Precision = \frac{\text{Number of relevant documents retrieved}}{\text{Total number of retrieved documents}}
\]

\[
	Recall = \frac{\text{Number of relevant documents retrieved}}{\text{Total number of relevant documents available}}
\]

The precision and recall measures have been useful tools in Twitter- and retweet-based literature. These pieces tend to only analyse the measures on their own work when applied to Twitter rather than on any more global scale. Certainly, there is less in the literature on the subjects of precision and recall with regards to retweeting in general.

The idea of assessing the credibility of information is introduced by \citet{castillo11}, who demonstrate methods of measuring the credibility of `news' and `chat' Tweets. In this case, retweeting is seen as a possible measure of a Tweet's credibility, since users typically only retweet information they see as interesting or useful. The authors use a logistic regression on a set of features derived from each Tweet in order to classify its credibility. 

The precision and recall metrics are used to verify the different aspects of the paper's results. In particular, they are applied to the classification of assessing credible information (and users) in order to calculate how well classified the information is. A higher precision, therefore, shows that their model has accurately classified most of the total information classified as either credible or non-credible.
\[	
	Precision = \frac{\text{Number of correct classifications}}{\text{Number of total classifications made}}
\]

\[
	Recall = \frac{\text{Number of correct classifications}}{\text{Total number of potential classifications}}
\]

On a similar note, \citet{hong11} discuss the notions of precision and recall more generally. The authors discuss the problem regarding the balance of information received by Twitter users. Having too few friends reduces the number or interesting posts received (i.e. low recall); having too many friends may cause information overload and is likely to include a lot of noise (i.e. low precision). This issue is used, instead of for the purpose of validating results, as a basis for the work; predicting the Tweets that are most popular and will be retweeted the most.

In addition, precision and recall are used to compare the method to two other baselines; the TF-IDF score, which in this case is used to indicate how important the terms are in each Tweet; and \emph{Retweet Before}, which uses the fact that if a Tweet in the training data has been previously retweeted, then it's likely to be retweeted again. The two metrics are also used to compare results when certain features are removed from the classifier. For example, showing that without using a `user retweet' feature, the precision and recall remain significantly higher than when removing other features, meaning that this feature does not contribute highly to the performance. More specifically, precision and recall are used in a similar way by \citet{castillo11}; except rather than looking at the number of classifications made, the authors use the number of predicted retweets.

\cite{bigonha10} discusses a proof of concept for detecting influential users in one of two categories; evangelists or detractors. Precision and recall, in this case, are used slightly differently:
\[	
	Precision = \frac{\text{Number of influential users retrieved}}{\text{Number of users retrieved}}
\]

\[
	Recall = \frac{\text{Number of influential users retrieved}}{\text{Total number of users}}
\]

The concept is taken further through the use of another metric, the \emph{Mean Average Precision}, which is used to denote an influential user as being a detractor or an evangelist. A high precision, in this case, would imply a large proportion of influential users are classified correctly and a high recall means that most of the influential users existing in the entire dataset have been classified. The final results then show the precision and recall values for detecting evangelists and detractors in both follower/following networks and interaction networks. Both precision and recall improved when the size of the set of highest classified influentials increased (i.e. the top set of influential users).

\citet{pak10} present a method for the automatic classification of Twitter information to determine if a document (or Tweet) is positive, negative or neutral in sentiment. In this case, the authors replace precision with \emph{accuracy} and recall with \emph{decision}, since they are using many classes instead of a binary classification, and define them as the following:
\[	
	Accuracy = \frac{\text{Number of correct classifications}}{\text{Number of all classifications}}
\]

\[
	Decision = \frac{\text{Number of retrieved documents}}{\text{Number of all documents}}
\]

The accuracy is measured across the classifier's decision, and the $ F_{0.5}-measure $  is then calculated based on these values instead in order to show that the classifier works well when the dataset size is increased.

As well as a news source, Twitter is also used as an informational, user-contributed source on world events. \citet{marcus11} introduce a system, TwitInfo, which can be used for detecting, summarising and visualising events from Tweets. The authors looked at football match footage, web content, and earthquake survey data, and manually annotated major events in each to produce ground truth sets. These would be used to compare and contrast the results produced by their event detector using the following definitions of precision and recall:
\[	
	Precision = \frac{\text{Number of events detected that are in ground truth set}}{\text{Total number of events}}
\]

\[
	Recall = \frac{\text{Number of events detected}}{\text{Number of events in ground truth set}}
\]

With these definitions set, the authors were then able to easily calculate precision and recall for their algorithm.

For the work in this thesis, interestingness of information is the performance metric used to describe information quality, and thus precision and recall for any particular user in the scope of this thesis can be defined as follows:
\[
	Precision = \frac{\text{Number of interesting Tweets received}}{\text{Total number of Tweets received}}
\]

\[
	Recall = \frac{\text{Number of interesting Tweets received}}{\text{Total number of all interesting Tweets}}
\]
where \textit{received} means that the Tweet has arrived on the user's home timeline, but does not imply that the user has \textit{read} the Tweet.

Therefore, a user following many other users will receive lots of interesting information onto their home timeline in amongst lots of noise; resulting in a reduced precision and higher recall. Another user might follow a very select few other users who are of direct interest, and thus will experience high precision, but low recall. These metrics are therefore useful in describing the concepts of noise and interestingness, and are consistent with their respective definitions in that users will achieve an optimum Twitter experience if both precision and recall are maximised.

\citet{zadeh13} defined bespoke definitions of precision and recall, yet also in the domain of interesting information on Twitter. Although the authors identify the need for users to be able to discover other users of interest and declare that Twitter does, in fact, have a `high precision' of interesting information, they admit to using a very coarse set of possible interest categories and is only based on \textit{overlapping} interests rather than addressing the interest-noise ratio more concerning the research in this thesis. Additionally, clicks on URLs by users are the only means by which to measure this interestingness, and Tweets with URLs are usually the most interesting type of information \cite{alonso10}.



\section{Collecting Twitter Data}
Most of the analytical work in this thesis relies on various data being collected from Twitter. Twitter provides an API for developers in order to facilitate the production of applications for its platform, but also for research purposes. It permits interfacing with many components of Twitter's service, such as posting and retrieving Tweets, interacting with other users (e.g. creating new friendships), and most of the features that Twitter's service itself provides to its users. The API encourages use of the OAuth\footnote{http://oauth.net} authorisation framework to handle access\footnote{https://dev.twitter.com/docs/auth}, allowing Twitter to keep track of applications and each application's access privileges and rate limits\footnote{https://dev.twitter.com/docs/rate-limiting/1.1}.

Twitter's traditional REST API, v1\footnote{https://dev.twitter.com/docs/api/1}, provided many useful endpoints for data collection and allowed each OAuth-authenticated application 350 hourly POST and GET requests\footnote{https://dev.twitter.com/docs/rate-limiting/1}. In June 2013 Twitter officially deprecated v1 of its REST API, forcing use of its new v1.1 API\footnote{https://dev.twitter.com/blog/api-v1-retirement-date-extended-to-june-11}. The new version contains many of the same resources\footnote{https://dev.twitter.com/docs/api/1.1} as the original, but workarounds are required to get the same results as some of the endpoint requests possible through v1. Additionally, new rate-limit policies were introduced, allowing more limited and controlled access to most of the available resources.

Since the work in this thesis was ongoing over this switch-over date, the initial work utilised API v1, and the latter work API v1.1, causing some changes to some of the data-collection methodologies as the thesis progresses. Descriptions of the data-collection in each relevant part of the thesis reflect this change, where appropriate.


\section{Motivation}
The motivation for the research questions declared in the previous chapter lies in the need to distinguish interesting information from noisy Tweets in Twitter, the latter of which is the problem area identified over the previous sections of this thesis. It has been made clear that the retweet count of a Tweet cannot reliably be used as a measure of interestingness, especially in the context of influential users, who naturally achieve significantly more retweets than average users, but which does not imply that the information they produce is of a higher quality or interest level. 

As a result, the retweet count alone cannot be useful in distinguishing interesting information from noise in a timeline of mixed Tweets from different users with different levels of influence - some further metric is required to make this distinction. Thus, questions \textbf{RQ1} and \textbf{RQ2}, together forming part of the hypothesis in Section \ref{section:research_questions}, have been partially addressed and have helped show why a method for estimating interestingness is useful. The remaining questions pose curious research pathways, which are introduced and explored further over the following chapters.

This thesis covers the research and development into a methodology for determining and ranking information on Twitter by inferred interestingness through considerations of user influence and the structure of users on the social graph.

\chapter{Understanding The Behaviour of Retweeting in Twitter}


It has been discussed in Chapter 2 that the popularity of information in Twitter can be related to the propagation characteristics of that information through Twitter's social structure. That is to say, that the more times a Tweet is retweeted by users, the more people have found the information contained within it to be interesting enough to be worth sharing.

It has also been shown that this retweet count metric alone cannot be a direct implication of the actual interestingness level of a Tweet. Reasoning for this is related to the notion of user influence, which dictates that some Tweets are naturally immediately available to more people and thus have a higher chance of achieving a retweet. Indeed, the authors of \cite{suh10} demonstrated that a user's Tweets' retweet rates increase as the user's follower count increases.

The strength of Twitter lies is in its social structure, where users can elect to follow and unfollow others as they desire and with immediate effect. Followers of a user receive all of that user's posts onto their individual (or `home') timelines. As a result, people are likely to follow users who create more `interesting' posts; whether the follower is a big fan of the friend and simply wants to know everything going on in their life, or if the follower is simply interested in the topical area of most of the friend's posts. 

Just as Twitter users will post Tweets with subjects that are of interest to them - possibly related to the user's work, a hobby, or a mixture of multiple areas - and these Tweets are generally posted with the idea that they will be useful or interesting for some of their followers as well as an attempt to attract more followers, retweets are generated with the same motives in mind. This means that if a Tweet is retweeted, it is not only allowed to disseminate further through the social structure, but also that a higher Tweet quality is implied.

Thus, this describes how a user's friends, who carry out most of the retweets of the user, effectively become filters of interesting information for that user and for the followers of those friends. The \textit{audience} of the original Tweet is therefore significantly increased. Since retweets are usually always attributed to the original author then you, a Twitter user, may gain more attention by means of followers by posting \textit{interesting} Tweets, which will; 
\begin{enumerate}
\item increase the chances that users reading your Tweets will choose to follow you, and;
\item increase the chances that users will decide to retweet your Tweet, thus broadcasting it to a larger audience. People viewing this \textit{retweet} then may decide to follow you. 
\end{enumerate}

Since a Tweet can be retweeted multiple times, and, as mentioned, a retweet itself can also be retweeted, the much larger the effective audience (both directly and through retweets) of a Tweet's original author has the potential to become if they choose to post interesting information. In this chapter, an understanding of the behaviours and properties of retweets is provided, along with discussions into how these are relevant in researching useful metrics for determining which retweeted information is interesting.


\section{Tweet and Retweet Properties}
In this section, a more formal overview on retweet properties is provided in addition to an introduction and definition of concepts frequently referred to in the thesis.

\subsection{Retweet Groups}
A Tweet has various attributes associated with it, which make up the features that describe it and its author. These properties relate the to the content, the author, and other metadata, such as its creation time, geographical location coordinates, language, and so on. However, not all of these properties are relevant to this research and, as such, a particular Tweet, $t$, has its relevant properties declared and defined as follows;
\[
	t = (\mathrm{text}, \rc{}, \aut{}{O}, \aut{}{R}, \mathrm{orig}, \mathrm{prev})
\]

Respectively, this represents the Tweet's text, its retweet count, and the \textit{original} author of the Tweet. The final three values depend on whether $t$ is a retweet or not and represent the author of the retweet and a reference to the original Tweet respectively and are \texttt{null} when $t$ is not a retweet. Since a retweet is simply an extension of a class of Tweet, then the same properties can be assigned to retweets as to Tweets, except that in the case of retweets the values $\mathrm{prev}$, $\mathrm{orig}$ and $\mathrm{author}_R$ will be non-\texttt{null}.

For example, let the Tweet shown in Figure \ref{fig:retweet_button} be $t^1$. This is not a retweet and has the following properties; 
\begin{flalign*}
t^1.\mathrm{prev} & = \textrm{\texttt{null}}\\
t^1.\mathrm{orig} & = \textrm{\texttt{null}}\\
\aut{t^1}{R} & = \textrm{\texttt{null}}\\
\aut{t^1}{O} & = \textrm{\texttt{Adrian Bradley}}
\end{flalign*}
and $\rc{t^1}$ is not known in this context.

Inversely, the Tweet displayed in Figure \ref{fig:northern_lights_tweet} ($r^1$) \textit{is} a retweet of another Tweet, $t^2$, where;
\begin{flalign*}
r^1.\mathrm{prev} & = t^2 \quad \textrm{(assumed)}\\
r^1.\mathrm{orig} & = t^2\\
\aut{r^1}{R} & = \textrm{\texttt{Discover The World}}\\
\aut{r^1}{O} & =  \textrm{\texttt{Alda Sigmundsd\'{o}ttir}}\\
\rc{t^2} & \geq 1 
\end{flalign*}
and $\rc{r^1}$ is also unknown. $r^1.\mathrm{prev}$ is assumed since $r^1$ was created using the button retweet method, which does not cite any intermediate retweeters, even if they exist.

A Twitter user, $u$, is represented by a Twitter account, and also has a set of properties. In relevance to the work in this thesis, these largely relate to the user's position in the social graph. 

The full social graph, denoted by $G$, comprises $V(G)$, the vertices representing the full set of users on Twitter; and $E(G)$, which is the set of edges connecting these vertices. In Twitter's case, the edges denote the followships between users, and are therefore \textit{directional}. Thus, the set of followers and the set of friends of user $u \in G(V)$ are denoted by $\fos{u}$ and $\frs{u}$ respectively, where;
\[
    \fos{u} = \left\{u \in V(G) :  \overrightarrow{u v} \in E(G)\right\}
    %\quad \forall \quad 0 \leq i \leq |G| \quad : \quad \exists \quad \overrightarrow{u_i u} \right\}
\]
\[
    \frs{u} = \left\{u \in V(G) :  \overleftarrow{u v} \in E(G)\right\}
    %\left\{u_i \quad \forall \quad 0 \leq i \leq |G| \quad : \quad \exists \quad \overleftarrow{u_i u} \right\}
\]
In other friendship-based social networks, such as Facebook, relationships are mutual and are therefore represented by non-directional edges in the relevant social graphs.

The terms $\foc{u}$ and $\frc{u}$ respectively refer to the in-degree and out-degree of a user $u$, where $u \in V(G)$. These in turn represent the cardinality of each of the set of followers and friends of $u$, and therefore the author of Tweet $t$ has a follower count of $\foc{\aut{t}{O}}$ and a friend count of $\frc{\aut{t}{O}}$.

Let $T$ represent the set of \textit{all} Tweets. Since a Tweet can be retweeted more than once, and have its retweets also retweeted, the set of retweets of Tweet $t \in T$ is defined as;
\[
    \rt{t} = \left\{ r \in T : r.\textrm{orig} = t \right\}
\]

Hence, the retweet count of $t$ is given by $ \rc{t} = \left\vert{\rt{t}}\right\vert $.


\begin{mydefinition}
\label{definition:retweet_group}
A \textbf{retweet group}, denoted by $\rg{t}$, describes the original Tweet, $t$, along with the set of the retweets of $t$, $\rt{t}$. Thus;
\[
    \rg{t} = \left\{t\right\} \cup \rt{t}
\]
\end{mydefinition}


Retweet groups are useful for classifying a Tweet and the users who have retweeted it, and is appropriate when discussing the audience reach of a particular Tweet. Therefore, since $t$ is also a member of this set, the size of $t$'s retweet group is; 
\[
	\left\vert{
\rg{t}}\right\vert = \rc{t} + 1 
\] 
which can have a minimum cardinality of one - $\rg{t} = \left\{\aut{t}{O}\right\}$ - in cases where there are no retweets of $t$.
 

\subsection{Retweet Trees}
As a Tweet gains popularity and is retweeted more, and since its retweets themselves can \textit{also} be retweeted, then this results in the generation of a retweet \textit{tree}, which represents the retweet group of a particular Tweet. This tree is represented by the original Tweet and the various propagation pathways it takes as it is retweeted through the social graph. 

The tree is not a representation of the actual social ties between the authors of the tree's nodes, as users are able to retweet Tweets and retweets sent from others that they do not follow. However, as is mentioned later in this chapter, most retweeting does generally occur between directly-linked users. \cite{kwak10} also uses retweet trees to assist in illustrating information dissemination in Twitter, particularly in observing the Twitter reactions to the 2009 Air France airline crash.

The root of the tree representing every $\rg{t} \forall t \in T$ is $t$ and, if $t$ has been retweeted, each of the other nodes are made up of the set of retweets in $\rt{t}$. Each non-root member of the tree refers to its parent through its own `$\textrm{prev}$' attribute, as illustrated in Figure \ref{fig:retweet_tree}. Retweet trees are useful for this purpose as they help demonstrate the temporal `paths' down which the retweets occur and the chains they produce. A similar illustrative device is used by \cite{galuba10} in describing URL cascades in Twitter.

\begin{figure}[h]
\centering
\includegraphics[scale=0.6]{3.Chapter1/Media/tree.png} 
\caption{A hypothetical retweet pathway tree.}
\label{fig:retweet_tree}
\end{figure}

In very rare cases, more than one node in a retweet tree may share an author user. This only occurs when a particular user retweets a Tweet more than once, and would only generally happen in scenarios where the user is using the manual method to modify the content as part of a conversation with others or for expressing multiple opinions. For example, a user may receive a Tweet relating to a particular news story, and then decide to retweet it with a small annotation. Upon feedback from followers, the user then retweets the Tweet again, yet with a different annotation. Each of these new retweets could then become the root of two branches in the complete tree of the Tweet.

Retweeting the same Tweet multiple times is not supported through the button method. Once a Tweet has been retweeted by a user in this way, all members of its retweet group are displayed equally to the user and there is no provision for the functionality to retweet a different member of the same group or, indeed, the original Tweet. A retweet can be `un-done' by clicking the button again on any member of the retweet group.


\subsection{Path-Length}
In addition to retweet groups having a size property, a retweet groups's branch's \textit{path-length} refers to the length of a particular retweet chain. 

\begin{mydefinition}
\label{definition:path_length}
The \textbf{path-length} of a single retweet chain in a retweet group is defined as the number of hops between a Tweet, $t$, and the retweet represented by the leaf node of the chain's branch in $\rg{t}$'s tree.\\
The \textbf{maximum path-length} of a retweet group is the greatest path-length observed in the retweet group.
\end{mydefinition}

Figure \ref{fig:retweet_tree} represents the users in the retweet group of a hypothetical Tweet. This retweet group has a size of 8 and has 3 distinct retweet chains, the longest of which are the two involving [$t, r_1, r_3, r_6$] and [$t, r_2, r_4, r_7$]. The \textit{maximum} path-length of this retweet group is therefore 3, as the leaf node of both of these branches is three hops away from the original author at the root.

Although the tree does not illustrate the edges between users on the social graph, is is possible for the underlying graph to connect the authors of the tree in various ways. For example, it is likely that user 3 follows user 1, but it's also possible that user 3 follows user 2. More on this topic is discussed in the audience analysis in Section \ref{section:audience} and in the social graph analyses in Section  \ref{section:retweets_graph}.

As has been mentioned previously, when a user retweets a Tweet or retweet through the manual approach, it involves pre-pending the current state of the Tweet with the text \texttt{RT @<username>:}. Therefore, a Tweet with the content;\newline
\texttt{RT @user2: RT @user1: This is the body of the Tweet}\newline
was originally authored by \texttt{user1}, then retweeted by \texttt{user2}, and then finally retweeted by the author of this current retweet (a Tweet or retweet's author's username is not credited in the body of the text in this way).

It should be noted that this phenomenon can only be observed through retweets by the manual approach, since the button method always simply credits the original author, and not any of the internal members of the retweet group. Although a significant number of retweets today are carried out using the button method, the manual approach still remains popular currently and even more so at the time the research in this chapter was carried out in the spring of 2011. This allowed for making useful observations of retweet patterns that could not be as successful later on.


\section{Information Searching and Affective Stimulation}
A Twitter user electing to follow another user cannot, in most cases, predict precisely what the new friend will Tweet about in the future. The user has some \textit{expectation} of the type of information they are likely to receive based on the previous Tweets of the new friend, which is generally the main cue the user can use to base the follow decision on.

Part of the follow decision is based on the notion of relevance judgement, which is an idea discussed at more length by \cite{xu07} and is partly made up of the goal of achieving \textit{affective stimulation} through \textit{hedonic} searching as opposed to the use of \textit{epistemic} searching.

\subsection{Epistemic Search}
An epistemic information search is one that involves carrying out a search with the purpose of finding out information on a particular topic (or set of) to satisfy a \textit{desire for knowledge} \cite{xu07}, yet without an actual aim to solve any particular problem.

An example of this type of search is a `crawl' through Wikipedia, in which a searcher may start at one particular page of interest and then follow links within that page to other related pages of interest that stem away from the source topic. In this case, the search `parameter' is simply the name or title of the article the searcher wants to view. As mentioned previously, a followship between users is effectively a search parameter in Twitter, since the following user has elected to follow the new friend to receive information from him/her. It is clear that this type of `searching' cannot be epistemic as the following user cannot know exactly the \textit{type} of information they are going to receive.

\subsection{Hedonic Search and Affective Stimulation}
\label{section:affective_stimulation}
Hedonic searching is similar to epistemic searching in that it is also not carried out with the aim to solve an immediate problem, but is different in that it is done to search for fun or `affective stimulation' \cite{xu07}.

A person can be said to be affectively stimulated if they view a piece of information that has some effect on the person, such as something that conveys emotion, something that is of particular interest to the person, or something that is capable of provoking some further thought.

With hedonic searching, users are not aware of the information that they are going to receive prior to searching and thus cannot really predict any level of affective stimulation. This aligns more with Twitter usage, since users receive information that they cannot accurately predict. Any Tweets received that do provide interesting information can convey affective stimulation to the user. This is the type of information that becomes harder to identify amongst lots of noise, yet is also the type of information a user is more likely to retweet.


\subsection{The Recognition Heuristic} 
A further metric for measuring information relevance in information retrieval is the recognition heuristic. This heuristic takes advantage of a person's memory and declares that if a person is able to recognise only one of two (or more) items, then s/he is more likely to judge the recognised item to be `greater' or more important \cite{oppenheimer03} \cite{goldstein99}.

Relating this to information received on Twitter, \cite{chorley12} found that a user recognising a Tweet's author significantly increases the chance that the user will decide to read the Tweet. Since a user must read a Tweet in order to make a decision on whether, or not, to retweet it, then the recognition heuristic transiently plays a part in a user's retweet decision also.

The authors also find that information about the Tweet itself, such as its text content and its retweet count, has much more of an effect on a user's read decision than information about the author, such as the followers count or Tweet rate. This also contributes to the declaration that information interest goes beyond the features surrounding a particular user and that user influence does not dictate interestingness of information.


\section{Twitter Propagation Analysis}
Understanding information propagation in Twitter is the key to also understanding how interesting information might be detected. Whilst it is known that the retweet count of a Tweet cannot be used alone in inferring interestingness, since this is simply a level of popularity tied in with the author user's influence, it is still a factor in that users are more likely to retweet interesting information than noise.

Of particular interest is to achieve an overview of propagation behaviours in Twitter; the patterns in the properties of retweet groups, such as their sizes and penetration depth, temporal aspects of retweets and information on the social structure of Twitter itself with regards to propagation within it.

The remainder of this chapter involves an exploratory study of the retweet characteristics in Twitter to provide a further background, and which demonstrates the area's relevance towards the goal of inferring interesting information.


\section{Retweet and Retweet Group Analysis}
To assist in providing a further grounding in this area of research, a series of analyses were carried out into retweets and retweet groups. This section describes the processes and purposes of the analyses.


\subsection{Data Collection Methodology}
The analyses involve the examination of Tweets extracted from Twitter's REST API v1, which was used between 26\textsuperscript{th} January and 24\textsuperscript{th} May 2011 to collect Tweets and retweets from the public timeline.

The data collection involved a mixture of using Twitter's timelines and its search capabilities. Version 1 of the REST API supported retrieval of Tweets, 20 at a time, from the Twitter \textit{public} timeline. Historically, this timeline contained the 20 most recent Tweets published by all the authors that have non-protected Twitter accounts, and it used to be visible on their website's homepage\footnote{http://twitter.com} to non-logged-in users.

In particular, for the data-collection periods, the public timeline endpoint was queried every ten seconds to retrieve the current set of the most recent public Tweets. Millions of Tweets are posted each hour, and ten seconds was a granular-enough frequency to ensure that there was no duplication in the data returned. From all of the retrieved Tweets, the ones that were retweets were filtered out and stored. Retweets, as mentioned earlier, are distinguishable since they start with the characters `RT' followed by a username. It should be noted that when retrieving Tweets from Twitter's API that even retweets that were created using the button method begin with the same character sequence, allowing detection of these also.

Following storage, the content of the retweets were parsed in order to extract the text that the original Tweet contained. Sometimes, retweets using the manual approach are used to provide additional annotation to the Tweet. Although this can often be distinguished by the fact that the original Tweet content is inside quotation marks (`` ''), it is not true in all cases, meaning that sometimes the original text could not be reliably extracted programmatically by a machine. In these cases additional queries were made to Twitter's search API in an attempt to resolve the problem, yet, failing that, the retweet was discarded.

Once the original text had been successfully extracted, this was used along with other metadata as query parameters to Twitter's search API in order to try and find the original Tweet and any other retweets of this Tweet. The search API uses approximate (or `fuzzy') string matching, but quotation marks can be used to retrieve search results based on an exact string pattern\footnote{https://dev.twitter.com/docs/using-search}.

Once the API search was complete (in some cases, with Tweets achieving many retweets, many API calls were required in order to page through results), the original Tweet could easily be identified as the only one of the set \textit{not} starting with the sequence ``RT''. This provided a retweet group comprising the original Tweet and all available retweets of this Tweet.

On some occasions, more than one Tweet were each identified as the original Tweet and so no data was stored for this group. This could occur, for example, if many users may Tweet exactly the same text if it comes external sources, such as a news webpage, and means that the entire set of retrieved Tweets are not likely to be part of the same retweet group. In cases where no results were returned, the retweet was not stored and was assumed to be an orphan retweet (perhaps as a result of a retweet of a Tweet posted by a protected Twitter account). Where no original Tweet could be identified it was sometimes possible to calculate it through cross-matching against other retweets in the retrieved retweet group, but were discarded if unsuccessful.

The retweet groups were finally stored along with relevant metadata in order to carry out the studies described in the following sections. The data consisted of a set of around 4,400 retweet groups, which comprised of 26,000 Tweets and retweets. The relatively limited size of the dataset is acknowledged, yet it should be emphasised that these analyses are simply exploratory and are not used to answer or solve any specific problem.


\subsection{Exploring Retweet Group Path-Lengths}
The path-lengths of each chain in a retweet group can be calculated by identifying the members involved in retweet activity down that chain; from the original Tweet to the final retweet. The \textit{maximum} path-length of a particular retweet group is the longest path-length observed in the group's tree.

Identification of path-lengths can be carried out through parsing the text of a retweet, and following the citations. Although it cannot be guaranteed that all users will be properly cited in a chain, and there is no realistic method to verify this, it is felt that correct citations will be made enough times to make these cases relatively insignificant.

On average, the maximum path-length observed across the retweet groups was around 1.8, with the vast majority of retweet chains being between one and two edges in length. When one considers that many retweets are made through the button method, which removes citations of internal users in the chain and simply credits the original author and would therefore produce many single-length retweet chains, this average will theoretically be an underestimate. \cite{kwak10}'s similar obsevations in the area also indicate a large number of groups with maximum path-lengths of one and two.

The longest observed maximum path-length was nine, which is a huge depth of penetration through the social structure since the total number of users involved in propagating the Tweet was ten. This, combined with the knowledge that social networks can represent a `closer' social graph than the real world's six degrees of separation (see Introduction), shows how retweeting can have a huge impact in information spread amongst millions of people worldwide very quickly.

\begin{myobservation}
\label{observation:path-length}
The mean maximum path-length observed across the retweet groups analysed was around 1.8.\\
The longest maximum path-length observed was 9.
\end{myobservation}

\begin{figure}[h]
\centering
    \begin{tikzpicture}
     \begin{axis}[
            xlabel=Maximum path-length of $\rg{t}$,
            ylabel=Frequency of occurrence,
            grid = major,
            width=8cm,
            xmin = 1,
            ymin = 1
            ]
        \addplot[mark=+,blue] plot coordinates {
            (1,2437) (2, 1049) (3, 461) (4, 191) (5,96) (6,65) (7,30) (8,7) (9,1)
            };
            % Data from pathlength-distribution.csv
    \end{axis}
    \end{tikzpicture}
    \caption{Distribution of maximum path-lengths observed in $\rg{t} \forall t \in T'$, where $T'$ is the set of Tweets analysed}
    \label{fig:pathlength-distribution}
\end{figure}

Also of interest is the relationship in terms of the social ties between the different authors of the Tweets in a retweet group. In cases where a retweet group's maximum path-length is precisely one, i.e. the situation where a user (or set of) has retweeted a particular Tweet only once, the retweeting authors of the leaf Tweets of this group's retweet tree follow the original author around 90\% of the time.

This implies, therefore, that in the remaining 10\% of cases, a retweeter has retweeted a Tweet from outside of their home timeline and has instead seen a Tweet whilst browsing through another user, who isn't a friend, timeline that the retweeter regards as sufficiently interesting. This helps to demonstrate that the more followers a particular user has, the greater the chance that another user somewhere has of viewing the user's Tweets and then having the opportunity to retweet them. The fact that 90\% of retweets of a particular user are created by direct followers reinforces this further.

This particular property could also be partly due to use of the button method of retweeting, which does not cite intermediate retweeters, and thus always imply that the final retweeter directly retweeted the Tweet from the original author. However, there may, in fact, have been other retweeters in between the final retweeters and original author, each of which following the immediately upstream retweeter. As such, this 90\% follow probability between the retweeter and source user in 1-hop retweet chains is also likely to be an underestimate.

\begin{figure}[h]
\centering
    \begin{tikzpicture}
    \begin{axis}[
        symbolic x coords={1,2,3,4,5,6},
            ylabel=Likelihood of citation of $\aut{t}{O}$,
            xlabel=Maximum path-length of $\rg{t}$,
            ymin=0,
            ymax=100,
            ybar,
            bar width=7pt
            ]
       \addplot plot coordinates{
            (1,72) (2,64) (3,60) (4,64) (5,65.5) (6, 83.5)    
        };    
    \end{axis}
    \end{tikzpicture}
    \caption{Proportion of cases where the original author is cited with varying maximum path-length of retweet group}
\label{fig:citation-pathlength}
\end{figure}

Further to this, in situations in which the maximum path-length of a retweet group is \textit{greater} than one, retweeting authors in the group follow the author of the original Tweet about 40\% of the time. It is clear from Figure \ref{fig:totalretweets-pathlength} that retweet groups with a longer maximum path-length tend to have a larger size themselves. This increases the likelihood that the Tweet has been able to spread both further around the original Tweet's author's community, and also the potential for the Tweet to `travel' to other communities. Since users from outside the source user's community are less likely to follow the source user, this explains the reduction in the followship likelihood between further downstream retweeters in the retweet chains and the original author.


\subsection{Size of Retweet Groups}
The distribution of retweet group sizes across all of the original Tweets in the set collected from Twitter was found to follow a power-law type distribution, with a relatively large $p$-value of around $0.87$. Figure \ref{fig:retweet-distribution} represents the complementary distribution function demonstrating the changing probability of a randomly generated $X$ being greater than or equal to $x$, the `current' value of $|\rg{t}|$, at each stage. The techniques used in this analysis are adapted from the methods and code provided by \cite{clauset07}.

\begin{figure}[h]
\centering
    \begin{tikzpicture}
     \begin{loglogaxis}[
            xlabel=$x$,
            ylabel=$P(X \geq x)$,
            grid = major,
            width=8cm,
            ]
        \addplot[mark=+,blue, only marks] plot coordinates {
(1,1) (2,0.6) (3,0.45) (4,0.4) (5,0.31) (6,0.29) (7,0.26) (10,0.21) (15,0.20) (16,0.19) (17,0.16) (18,0.145) (19,0.14) (20,0.13) (31,0.125) (32,0.118) (33,0.10) (39,0.09) (57,0.077) (82,0.068) (157,0.044) (310,0.035) (700,0.025) (1100,0.012)
            };
        \addplot[mark=none,red] table[mark=none,blue,row sep=\\,
            y={create col/linear regression={y=Y}}]{
            X Y\\
            1 1\\
            2 0.6\\
            3 0.45\\
            4 4\\
            5 0.31\\
            6 0.29\\
            7 0.26\\
            10 0.21\\
            15 0.20\\
            16 0.19\\
            17 0.16\\
            18 0.145\\
            19 0.14\\
            20 0.13\\
            31 0.125\\
            32 0.118\\
            33 0.10\\
            39 0.09\\
            57 0.077\\
            82 0.068\\
            157 0.044\\
            310 0.035\\
            700 0.025\\
            1100 0.012\\
        };
            % Data from pathlength-distribution.csv
    \end{loglogaxis}
    \end{tikzpicture}
    \caption{Maximum likelihood power-law fit for the cumulative distribution of retweet group sizes}
    \label{fig:retweet-distribution}
\end{figure}

The mean group size from this dataset was found to be just below six, and the largest size was 284. The smallest $|\rg{t}|$ were the cases in which $\rc{t} = 1$, and which were significantly the most common occurrences.

Of interest also is the relationship between a group's size and its maximum path-length. Generally, the maximum path-length of a group, $\rg{t}$, increases with $|\rg{t}|$, indicating a mostly uniform growth in the retweet trees representing these groups - as might be expected. Figure \ref{fig:totalretweets-pathlength} demonstrates this trend, which this illustrates that as the retweet count of $t$ increases, then the longer the retweet chains in $\rg{t}$ are likely to become. This would increase its penetrative dissemination away from the source and further facilitate its spread between communities, increasing its potential \textit{audience size}.

\begin{figure}[h]
\centering
    \begin{tikzpicture}
    \begin{axis}[
            ylabel=Mean $|\rg{t}|$,
            xlabel=Maximum path-length of $\rg{t}$,
            grid=major,
            xmin=1,
            width=8cm]
       \addplot[mark=+,only marks,blue] plot coordinates{
            (1,2) (2,6) (3,11) (4,12) (5,16) (6, 16) (7,29) (8,19) (9,78)    
        };    
    \end{axis}
    \end{tikzpicture}
    \caption{Relationship between the maximum path-length and size of a retweet group. The greatest path-length was included for context, but had a sample size of only one.}
\label{fig:totalretweets-pathlength}
\end{figure}

\subsection{A Tweet's Audience - How Many Users Can be Reached?}
\label{section:audience}
$\rg{t}$'s (immediate) audience size refers to the number of Twitter users that have received $t$, either in its original form or as a retweet, $r$, such that $r.\textrm{orig} = t$. Users in the audience are not guaranteed to have read the Tweets, but they are the users who will see the Tweet on their home timelines.

The term `immediate' is used to signify the distinction between those users who passively receive the Tweet, due to following the original author or a retweeter, and those who see the Tweet whilst actively browsing through other user timelines or the public timeline. Users in the latter group are therefore not direct followers of $\aut{t}{O}$ or $\aut{r}{R} \forall r \in \rt{t}$ and thus cannot be tracked as members of $t$'s audience.

Let $ r_1,...,r_n $ be the members of $\rt{t}$. The size of the audience of $\rg{t}$ can then be calculated thus (assuming $\rc{t} \geq 1$);
\[
	|\raa{\rg{t}}| = |\fos{\aut{t}{O}}| + |\fos{\aut{t}{r^1}}| + ... + |\fos{\aut{t}{r^n}}| 
\]
where $\aut{t}{r^1} ... \aut{t}{r^n}$ are the users who retweeted $t$.

Despite this, properties of Twitter's social graph dictates that this audience size calculation is na{\"i}ve in that, particularly in the case of more tightly-knit communities, users who are authors of $t$ or $r \in \rt{t}$ are likely to share a subset of each of their followers. The more dense the communities, the more followers are likely to be shared between the authors in $\rg{t}$ and, as such, the aforementioned audience calculation is likely to be an overestimate in nearly all cases. As such, $\raa{\rg{t}}$ is a list and not a formal \textit{set} of users, since it is likely to have some non-distinct members.

The following analyses of retweet group audience sizes relies on a dataset which began collecting at a later date than the general set used in this chapter, and thus the data represented in the rest of this section contains 2860 of the total 4400 groups originally collected. The longest maximum path-length of retweet groups observed in this subset was eight.

The \textit{overhead} of a group, $\rg{t}$, which attempts to address this problem, is related to the level of redundancy of received information by the audience.

\begin{mydefinition}
    The $\rg{t}$'s \textbf{overhead} is a value equal to the number of users in $\raa{\rg{t}}$ that receive $t$ or any $r \in \rt{t}$ more than once and, if received more than once, the number of times $t$ is received by each user.
\end{mydefinition}

The audience overhead was found to exist (be greater than 0) in 71\% of all observed retweet groups, further reinforcing that retweets often occur within communities containing users sharing many edges.

Therefore, the actual audience of a Tweet is given by the \textit{set} of users that can be found by modifying the existing definition to take the overhead into account;
\[
	\dia{\rg{t}} = \fos{\aut{t}{O}} \cup \fos{\aut{t}{r^1}} \cup ... \cup \fos{\aut{t}{r^n}}
\]
where $\aut{t}{r^1} ... \aut{t}{r^n}$ are the users who retweeted $t$. This definition (`distinct') of audience is used in preference over the previous (`raw') definition for the remainder of the thesis.

 The \textit{proportionate} overhead is defined as the ratio of the overhead to the audience size, and is sometimes more useful for analysing the size of the overhead compared to the popularity of the original Tweet.

For example, a Tweet $t$ has been received onto the home timelines of 800 users as a result of a single retweet by user $r_1$. 400 of those users received the Tweet twice due to the presence of shared followers between $\aut{t}{O}$ and $\aut{r_1}{R}$. In this case, the overhead is 400, the proportionate overhead is 0.5, the old method for calculating the audience gives a size of 1200, and the new method gives a size of 800.

\begin{figure}[h]
\begin{subfigure}{.5\textwidth}
    \centering
    \begin{tikzpicture}[scale=0.8]
    \begin{axis}[
            ylabel=Mean distinct audience size of $\rg{t}$,
            xlabel=Maximum path-length of $\rg{t}$,
            grid=major,
            xmin=1]
       \addplot[mark=+,only marks,blue] plot coordinates{
            (1,1430) (2,2294) (3,2427) (4,3421) (5,4333) (6,5087) (7,4015) (8,2862)
        };    % audienceDISTINCT_pathlength.csv
    \end{axis}
    \end{tikzpicture}
    \caption{Varying $\rg{t}$'s \textit{distinct} audience size with its longest path-length.}
    \label{fig:pathlength-audience}
\end{subfigure}
\quad
\begin{subfigure}{.5\textwidth}
    \centering
    \begin{tikzpicture}[scale=0.8]
    \begin{semilogyaxis}[
            ylabel=Mean immediate  audience size of $\rg{t}$,
            xlabel=Maximum path-length of $\rg{t}$,
            grid=major,
            xmin=1]
       \addplot[mark=+,only marks,blue] plot coordinates{
            (1,5030) (2,8660) (3,8175) (4,7474) (5,8313) (6,17977) (7,9351) (8,40705)
        };    % audienceDISTINCT_pathlength.csv
    \end{semilogyaxis}
    \end{tikzpicture}
    \caption{Varying $\rg{t}$'s \textit{raw} audience size with its longest path-length.}
    \label{fig:pathlength-rawaudience}
\end{subfigure}
\caption{Comparison of the relationships between a $\rg{t}$'s distinct and raw audience size and its maximum path-length $\forall t \in T'$, where $T'$ is the set of analysed Tweets}
\end{figure}

Figure \ref{fig:pathlength-audience} illustrates, initially, that which might be expected; that the distinct audience size of a Tweet, $t$, is mostly proportional to the maximum path length of $\rg{t}$. However, as the maximum path-length of retweet groups exceeds 5, then a \textit{decline} in the distinct audience size is observed. This particular behaviour has an unclear cause, but it is felt that this could be to do with a saturation in the proportionate overhead's ratio at this stage - in particular, that retweet groups attracting many retweets are circulated more within communities than outside and between communuties. At this stage, the overhead becomes so large, causing this reduction in audience size. This is significant in that the distribution of the non-distict over the increasing path-lengths demonstrates, mostly, a continuous positive correlation (Figure \ref{fig:pathlength-rawaudience}).

 

The \textit{largest} overhead was of a size over six times greater than the group's distinct audience size itself, indicating a massive overlap between the followers of the author of the original Tweet and the authors of its retweets. Whilst the audience overhead was only found to be greater than the distinct audience size in around 3\% of observed retweet groups, it is still clear that the potential for overlap in the followers of retweet group members can be very large in more closely-knit communities Figures \ref{fig:pathlength-largestoverhead} and \ref{fig:pathlength-largestproportionateoverhead} show that the largest overheads observed diminishes in groups with greater maximum path-lengths, which helps illustrate this concept.

Inversely, Figures \ref{fig:pathlength-meanoverhead} and \ref{fig:pathlength-meanproportionateoverhead} respectively show that the \textit{mean} overhead and mean proportionate overhead increase in retweet groups with greater maximum path-lengths. It is assumed that with larger retweet groups there is a greater chance for overlap between the followers of the authors of the members due to there being a greater audience size. Since it is known that increases in the sizes of groups can be indicated by increases in the groups' maximum path-lengths, then this suggests that, on average, the overhead should increase with maximum path-length. 

The diminishing behaviour observed in the other two previously-discussed subplots suggest that these groups with smaller maximum path-lengths exhibiting greater overheads are those that do not fit the trends across retweet group size and maximum path-length observed earlier in this chapter. As such, it is likely that these groups are actually large, with representative trees that are shallow and very \textit{wide}. This illustrates that the Tweets have not propagated far from the original author, yet have circulated thoroughly through a local community. Indeed, three of the largest five overheads in the set of analysed Tweets occur in retweet groups which have a maximum path-length of one. 


\begin{figure}[h]
\begin{subfigure}{.5\textwidth}
    \centering
    \begin{tikzpicture}[scale=0.8]
    \begin{axis}[
            ylabel=Mean overhead of $\rg{t}$,
            xlabel=Maximum path-length of $\rg{t}$,
            grid=major,
            xmin=1
            ]
       \addplot[mark=+,only marks,blue] plot coordinates{
            (1,159) (2,336) (3,532) (4,590) (5,1148) (6,1321) (7,1020) (8,1258)
        };    % overhead-pathlength.csv
        \addplot[mark=none,red] table[mark=none,blue,row sep=\\,
            y={create col/linear regression={y=Y}}]{
            X Y\\
            1 159\\
            2 336\\
            3 532\\
            4 590\\
            5 1148\\
            6 1321\\
            7 1020\\
            8 1258\\
        };
    \end{axis}
    \end{tikzpicture}
    \caption{Varying mean overhead with maximum path-length.}
    \label{fig:pathlength-meanoverhead}
\end{subfigure}
\quad
\begin{subfigure}{.5\textwidth}
    \centering
    \begin{tikzpicture}[scale=0.8]
    \begin{axis}[
            ylabel=Largest overhead of $\rg{t}$,
            xlabel=Maximum path-length of $\rg{t}$,
            grid=major,
            xmin=1
           ]
       \addplot[mark=+,only marks,blue] plot coordinates{
            (1,48868) (2,23988) (3,13840) (4,7688) (5,29267) (6,7122) (7,3974) (8,3827)
        };    % largestoverhead-pathlength.csv
    \end{axis}
    \end{tikzpicture}
    \caption{Varying \textit{largest} observed overhead with maximum path-length.}
    \label{fig:pathlength-largestoverhead}
\end{subfigure}

\begin{subfigure}{.5\textwidth}
    \centering
    \begin{tikzpicture}[scale=0.8]
    \begin{axis}[
            ylabel=Mean proportionate overhead of $\rg{t}$,
            xlabel=Maximum path-length of $\rg{t}$,
            grid=major,
            xmin=1
            ]
       \addplot[mark=+,only marks,blue] plot coordinates{
            (1,0.113) (2,0.191) (3,0.315) (4,0.300) (5,0.227) (6,0.381) (7,0.288) (8,0.511)
        };    % overheadproportion-pathlength.csv
        \addplot[mark=none,red] table[mark=none,blue,row sep=\\,
            y={create col/linear regression={y=Y}}]{
            X Y\\
            1 0.113\\
            2 0.191\\
            3 0.315\\
            4 0.300\\
            5 0.227\\
            6 0.381\\
            7 0.288\\
            8 0.511\\
        };
    \end{axis}
    \end{tikzpicture}
    \caption{Varying mean overhead \textit{proportion} with maximum path-length.}
    \label{fig:pathlength-meanproportionateoverhead}
\end{subfigure}
\quad
\begin{subfigure}{.5\textwidth}
    \centering
    \begin{tikzpicture}[scale=0.8]
    \begin{axis}[
            ylabel=Largest proportionate overhead of $\rg{t}$,
            xlabel=Maximum path-length of $\rg{t}$,
            grid=major,
            xmin=1
           ]
       \addplot[mark=+,only marks,blue] plot coordinates{
            (1,25.7433) (2,4.8584) (3,8.8644) (4,1.876) (5,2.228) (6,2.176) (7,1.51) (8,2.39)
        };    % largestoverheadproportion-pathlength.csv
    \end{axis}
    \end{tikzpicture}
    \caption{Varying \textit{largest} observed overhead \textit{proportion} with maximum path-length.}
    \label{fig:pathlength-largestproportionateoverhead}
\end{subfigure}
\caption{Relationships between $\rg{t}$'s audience overhead properties and its maximum path-length $\forall t \in T'$, where $T'$ is the set of analysed Tweets}
\end{figure}

The power of the retweet phenomeon in terms of how it affects the potential audience reach of a particular Tweet is discussed in further detail by the authors of \cite{kwak10}, in which they find that a retweeted Tweet of sufficient interest can reach a very large number of users even if the original author has only a few followers. The same paper more specifically mentions that the audience size of a retweeted Tweet reaches, on average, at least 1,000 users, no matter the number of followers of the original author. This result agrees with the results in Figure \ref{fig:pathlength-audience} in that even Tweets with a short maximum path-length still, on average, have a relatively large audience size.


\subsection{Retweet Groups on the Social Graph}
\label{section:retweets_graph}
Now that an understanding has been achieved in the behaviours and properties of retweets and retweet groups, it is important that the social ties between users in groups is studied. This will provide a grounding for the research in the following chapter, in which the social structure and its role in facilitating propagation, are discussed in more detail.

It has already been mentioned that the probability of a retweeting author following the original author in unit-length retweet chains was found to be around 90\%. However, in retweet groups with longer chains, a decrease in the likelihood of the final retweeter (the user at the bottom of the retweet tree) following the original author was observed. Indeed, on average across all retweet groups, the final retweeter follows the \textit{previous} retweeter in a particular retweet chain in around 67\% of cases. The final retweeter is defined as the author of the chronologically final retweet in the group, and is not necessarily the user at the leaf node of the group's longest retweet chain.

%\begin{figure}[h]
%\centering
%\includegraphics[scale=0.6]{3.Chapter1/Media/following-possibility.png} 
%\caption{\textit{Relationship between the chance of followship between the final retweeter and original author in groups and varying maximum path-length}}
%\label{fig:followingchance_pathlength}
%\end{figure}

It is interesting that this value should be about 20\% lower than in unit-length maximum path-length groups, and it suggests that users have a greater chance of `stumbling over' retweets found on non-friends' timelines whilst browsing through other users. Since it has been shown that with an increase in maximum path-length an increase in the audience size is also observed, then this demonstrates the increased chance of discovery of the Tweet through users searching through others' profiles. In cases where the maximum path-length of $\rg{t}$ is equal to one, then the audience size is far smaller and thus there is a lower chance of users who aren't followers of $\aut{t}{O}$ or $\left\{\aut{r}{R} \forall r \in \rt{t}\right\}$ finding the Tweet.

\begin{figure}[h]
\centering
\includegraphics[scale=0.7]{3.Chapter1/Media/final_following_upstream.png} 
\caption{Effect of the final retweeter following the upstream user on the follower count of the upstream user}
\label{fig:final_following_upstream}
\end{figure}

In addition, there is some evidence of user influence playing a role in the analyses of these data. In particular, in the 67\% of retweet groups in which the final retweeter \textit{does} follow the author of the retweet (or original Tweet) directly `upstream', the latter user has, on average, around 950 followers. Inversely, in the remaining 33\% of groups (in which the author of the final retweet does \textit{not} follow the preceding author), the preceding author has an average of 600 followers. This is illustrated by an example in Figure \ref{fig:final_following_upstream} and it implies that there is a significant difference in the retweet potential with varying author influence levels.

\begin{figure}[h]
\centering
\includegraphics[scale=0.7]{3.Chapter1/Media/final_following_original.png} 
\caption{Effect of the final retweeter following the original author on the follower count of the original author}
\label{fig:final_following_original}
\end{figure}

This is further accentuated when one studies the follower connections of $\aut{t}{O}$ in groups where the maximum path-length is greater than one. Whilst it was found earlier that the likelihood of a $\aut{r}{R}$ following $\aut{t}{O}$ is around 40\%, the average follower count of $\aut{t}{O}$ has a four-fold increase (from about 550 to 2,000) when he/she is also followed by the final retweeter. Figure \ref{fig:final_following_original} demonstrates an example of this in a retweet chain of two retweets; that when the author of $r_3$ follows the author of $t$, then $\foc{\aut{t}{O}}$ is significantly greater. 

In fact, in groups of \textit{all} maximum path-lengths, $\aut{t}{O}$ had a consistently higher follower count when followed also by the final retweeter of $\rg{t}$ than when not followed.

This particular behaviour also helps illustrate that a user is more likely to be retweeted when having more followers - in this case, having four times the follower count increases the correlation dramatically (40\% to 90\%). The follower count can, therefore, be directly related in this way to the discussions of user influence in \cite{cha10}, and also of users using retweeted Tweets to passively `advertise' themselves.

\begin{figure}[h]
\centering
    \begin{tikzpicture}
    \begin{axis}[
            ylabel=Mean $\textrm{deg}^+(t.\textrm{author}_O$),
            xlabel=Maximum path-length of $\rg{t}$,
            grid=major,
            xmin=1,
            width=8cm]
       \addplot[mark=+,only marks,blue] plot coordinates{
            (1,805) (2,621) (3,291) (4,205) (5,195)
        };    
    \end{axis}
    \end{tikzpicture}
    \caption{Analysis of variance in $\foc{\aut{t}{O}}$ as $\rg{t}$'s maximum path-length increases}
\label{fig:originalfollowers-pathlength}
\end{figure}


%\begin{figure}[h]
%\centering
%\includegraphics[scale=0.6]{3.Chapter1/Media/originalfollowers-pathlength-distribution.png} 
%\caption{\textit{Relationship between number of followers (and respective distribution) of the original tweeter as the path-length increases}}
%\label{fig:originalfollowers-pathlength}
%\end{figure}

Strangely, it was found that increases in maximum path-length of retweet groups caused the follower count of the original author to diminish, indicating further penetrative depth of propagation when the original author has \textit{fewer} followers. The collected retweet groups that contained longer retweet chains often also contained retweet chains that were much shorter. For example, a group containing chains with path-lengths of five, or more, are also likely to contain many more chains with path-lengths of one and two (as is implied in the distribution in Figure \ref{fig:pathlength-distribution}). There are, therefore, various possible explanations for this property, including the argument that users with many followers are generally likely to be part of a large community of users, from which retweets are not transmitted. Users that are part of several communities, and are therefore less involved with any given one, may find that their Tweets have the potential to be retweeted a further distance.

Additionally, and more interestingly, it is possible that users possess some awareness of their local networks and the users within them. A user, who is part of a large community with lots of obvious follower overlaps occurring between the members, may decide \textit{not} to retweet a particular Tweet if he/she feels that many of their own followers may have already seen the Tweet due to them also having a high chance of following the original author.

A final analysis on the social ties between users in retweet chains is carried out on the followship pattern of authors throughout the chain. Let $h$ be the number of hops (or edges in the retweet tree) between two users in a retweet chain. It was illustrated in earlier sections that, when $h = 1$, the likelihood of the later retweeting author following the upstream author is around 67\%. However, as $h$ is increased, then the followship likelihood mostly consistently decreases (see Figure \ref{fig:following-possibility}), as might be expected. This illustrates how longer retweet chains do indeed increase both the likelihood of the Tweet reaching further through the social structure and the chance of achieving a smaller proportionate overhead.

Further to this, of the 67\% of retweeters who \textit{do} follow an upstream author at $h = 1$, only 19\% follow also the upstream author at $h = 2$. In these cases, the latter has an observed average of around 3,000 followers. In the 81\% of cases when the user at $h = 2$ \textit{isn't} also followed, then the upstream author has a much lower average follower count of about 520.

\begin{figure}[h]
\centering
    \begin{tikzpicture}
    \begin{axis}[
            ylabel=Chance of $\aut{r}{R} \in t.\textrm{author}_O.E_i$,
            xlabel=$h$,
            grid=major,
            xmin=1,
            width=8cm]
       \addplot[mark=+,only marks,blue] plot coordinates{
            (1,0.67) (2,0.35) (3,0.25) (4,0.19) (5,0.25)
        };    
    \end{axis}
    \end{tikzpicture}
    \caption{Relationship between the followship chance of $\aut{r}{R}$ (where $r \in \rt{t}$) to $\aut{t}{O}$ and the increase in `distance' between $r$ and $t$ given by $h$}
\label{fig:following-possibility}
\end{figure}


%\begin{figure}[h]
%\centering
%\includegraphics[scale=0.55]{3.Chapter1/Media/following-possibility.png} 
%\caption{\textit{Proportion of final retweeters following upstream users at varying distances along the chain.}}
%\label{fig:following-possibility}
%\end{figure}

It is, therefore, sensible to assume from these analyses that Tweets are forwarded more between groups of less-connected users, highlighting the notions of social network awareness and of community-hopping. If retweets were usually circulated around more closely-knit communities of users, then the followship likelihoods would be generally greater, more uniform, and consistent throughout the retweet chain. Users would have as much of a chance of following their immediate upstream neighbour author in the retweet chain as they would an author further upstream.

As mentioned near the start of this chapter, the author of the original Tweet should be cited by the \texttt{RT @<username>} sequence observed \textit{closest} to the retweet body, where the \texttt{<username>} is the username of the original author user. Rather than specifically looking for the author's Tweet appearing in this location, Tweets were examined to check for the existence of the author's username being mentioned \textit{anywhere} in the Tweet content, and was found to exist in about 68\% of Tweets.\\
This frequency did not vary with any consistent correlation upon changes to the maximum path-length or retweet group size, and so it is assumed that users do feel the need to credit the original author more so than not. 

 
\subsection{The Temporal Properties of Retweets}
The final set of analyses in this chapter relate to time's influence on retweet propagation. This provides insights into how quickly information can spread and, when combined with the knowledge of the social structure and audience, how this can relate to the rate of information dissemination and consumption.

%\begin{figure}[h]
%\centering
%    \begin{tikzpicture}
%    \begin{axis}[
%            ylabel=Time elapsement (secs.),
%            xlabel=Maximum path-length of $\rg{t}$,
%            grid=major,
%            xmin=1,
%            width=8cm]
%       \addplot[mark=+,blue] plot coordinates{
%            (1,21766) (2,33466) (3,53582) (4,80658) (5,41600) (6,31360) (7,23989) (8,63102) (9,133611) 
%        };    
%    \end{axis}
%    \end{tikzpicture}
%    \caption{Comparison between the average time elapsement from $t$ to the final $r \in 
%\rt{t}$ and the maximum path-length of $\rg{t} \forall t \in T'$, where $T'$ is the set of analysed Tweets}
%\label{fig:timedelay-pathlength}
%\end{figure}

%\begin{figure}[h]
%\centering
%\includegraphics[scale=0.35]{3.Chapter1/Media/pathlength-timedelay.jpg} 
%\caption{\textit{Average time (in seconds) between first post and final retweet of a retweet group varying with the group's maximum path-length}}
%\label{fig:timedelay-pathlength}
%\end{figure}

Generally, it was found that the elapsement of time between the original Tweet and the final retweet in retweet groups increased with the groups' maximum path-lengths, indicating that if there are more hops for a Tweet to travel down between users then it takes longer to do so. However, this correlation is only really applicable to shorter retweet chains, which more uniformly increase in temporal elapsement with increases in maximum path-length in a linear fashion roughly proportional to $ v=\frac{s}{t} $, where the distance, \textit{s}, is the hypothetical distance given by the number of hops between users, thus indicating that the speed, $v$, of propagation remains relatively constant. 

Retweet groups exhibiting longer maximum path-lengths are less consistent in terms of the groups' propagation speeds. Whilst this is likely attributed to smaller samples, there are conflicting arguments for patterns observed in this propagation speed, which rely on various intervening factors. 

As mentioned, the time taken for a Tweet to reach a specific path-length could be a function of the path-length itself, where as the path-length increases, then so does the time taken for the Tweet to be retweeted to the end of the chain. Inversely, Tweets that are especially popular, possibly as a result of being particularly topical (such as in the disaster cases mentioned in the Introduction), may be retweeted more quickly by users so that the information is spread more quickly. In these cases, retweet groups with longer retweet chains may complete their trees more quickly than those groups with much shallower retweet trees.

Similarly, user influence could play a role in dissemination speed; if a Tweet is retweeted by a user with many followers, then there is an increased likelihood of propagation through this user. Whilst this could, in addition to the previous argument, cause longer retweet trees to be completed more quickly than groups with shorter trees, it could also facilitate `faster branches', in which particular long branches grow faster and reach their leaves more quickly than shorter ones in the same retweet tree if the other branches consist of less-influential author users.

There is not enough evidence provided in this analysis to make any inferences towards a generic pattern of retweet group growth speed, and it is believed that this growth is governed by many more factors than the Tweet itself or the social structure alone. As such, there is no predefined rule for predicting the spread of dissemination in this way, since the retweet path is an unknown feature, with too many variables and conflicting arguments.

The temporality of retweets has been the focus of some researchers, including the authors of \cite{kwak10}, who also used retweet trees as an illustration of the propagation pattern produced by Tweets. They found that, generally, half of all retweet action on a Tweet occurs \textit{within an hour} of the Tweet being posted, and that by the end of the first day, 75\% of all retweets of the Tweet will have been carried out. The authors also conducted an analysis on the elapsed time of a Tweet's travel between hops as it is retweeted. Although they observe a flatter time elapsement initially, indicating that Tweets travelling over the first few hops are retweeted almost concurrently, they also found there to be a general incline in time elapsement over the shorter path-lengths. After this point, as seen and discussed in the analysis earlier, the elapsement becomes more `noisy'. 

An interesting notion that is not directly addressed in this thesis is that the time a particular Tweet is authored may have some effect on its propagation speed. Just as `prime-time' television achieves higher audience ratings as it is at a time of the day when many people are at home and relaxing, Twitter may also have a time window in which its users are more active. For example, if a user posts a Tweet at a time when many of his/her followers are asleep, then the immediate audience size of the Tweet can be significantly reduced.

If there are fewer initial users viewing the Tweet, then the likelihood of retweet, as a function of this, is also reduced. This could have an effect on the perceived popularity of the Tweet, although since, by definition, there are fewer active users on Twitter at this time, then the number of Tweets sent during this period will be much smaller. We therefore do not take this factor into account during our experimentation in later chapters.


\section{Summary}
In this chapter, a set of initial exploratory analyses have been undertaken into the behaviour of retweets and retweet activity in Twitter, the properties of retweet groups, the relationships between the propagation graph and the social graph, and briefly into the effects of time on Tweet dissemination.

The analyses were found to support and complement the findings of other research in the area, including the notions of message cascading \cite{galuba10} and the relationships of this to the interconnection of users on the social graph through communities \cite{java07}. Trees representing the retweet groups were found to grow in a variety of ways, from thost illustrating long retweet chains, indicating a high level of inter-community dissemination, to shorter and wider trees, in which propagation can still be widespread but not as likely to disseminate to other communities.

User influence, in terms of an author's follower count, was observed as being an important factor in facilitating information spread, implying that popular users also produce popular information, since these users are more likely to achieve more retweets.

These inferences have helped to describe the multi-dimensional principles of retweet groups in terms of the features governing their spread over the social graph, and the quickness with which many users can be exposed to a Tweet. Although it is important to have an understanding of user psychology, and the thought processes behind the retweet decision, of most interest in this chapter is the analysis of the social structure.


\section{Taking the Investigative Research Further}
Twitter's social structure has been found to have a large effect on Tweet propagation, since it combines the features observed around user influence (in the na{\"i}ve form of a user's follower count) with that of communities and sub-graphs of dense and sparse user interconnections.

In the following chapter, interests are focused on the topiological structure of user followships by investigating further into the flow of information between users as they are arranged in different ways in order to develop a method to infer information interestingness taking into account these information flow properties and user influence. It is clear that different Tweets can have a different level of \textit{quality} in that Tweets that are retweeted have a greater chance of being interesting, but does the way in which the social structure of users is formed also have a quality in terms of supporting retweeting?

\chapter{Analysis of Twitter's Social Structure}


Stuff to add to this section:
\begin{itemize}
\item Change tweet features for each simulation and make comparison on these differences
\item Observe differences in patterns when network generation parameters are altered
\item Link up section to previous section (i.e. how did the previous research help and how does this build on that work?)
\item Explain how this section becomes the basis for work in 'main chapter 3'.
\item (e.g. Issues with current method (too long, requires network, inaccurate due to having to choose users with fewer followers), so need a quicker, more accessible and online approach).
\item Explain the Mechanical Turk questions in more detail, with examples.
\item Discuss about the machine learning approach used (logistic regression and how it works)
\item Link 'retweet volume' to 'retweet group size'
\end{itemize}


In the previous chapter, a series of studies were conducted into Twitter with respect to message propagation through retweeting. In particular, research was done to provide an understanding of the patterns produced through retweets and how their properties relate to the Twitter users that the Tweets pass through.

Of particular interest, however, is the social graph underlying Twitter, which describes how the users are interconnected and dictates the information flow between them. It has been discussed that users with a higher follower count are more likely to have their Tweets retweeted and that some users can have their Tweets forwarded through many hops indeed, so that information may be passed between different communities of users.

In addition to the effects of user influence,  several other factors also govern an individual retweet decision of a given user for a particular Tweet. These include properties of the Tweet, such as whether, or not, the Tweet contains a URL, whether it mentions a particular user, whether the user even has an opportunity to view the Tweet, and so on.\\
These factors account for the individual user's retweet decision and the almagamation of every user's retweet decision on the Tweet describes the Tweet's overall retweetability and, which determines how far the Tweet can propagate.

However, it is believed that the topology of the network, below the level of user influence, can play an important role in facilitating (or inhibiting) Tweet propagation by opening and closing available retweet pathways between users and groups of users.\\
Whilst retweet decisions based on Tweet features alone (such as the actual content of the Tweet or a URL the Tweet points to) may imply a level of interest in the Tweet, the influence of users has a very large impact on how many retweets a certain Tweet receives. Thus, abstracting the concepts away from user influence may help in discovering methods for inferring which information is interesting.

Twitter's social structure has previously been described as being built from users creating edges between themselves through the act of following. A followship defines the direction of travel of information from the follower to the friend, and this illustrates how users with many followers immediately have their Tweets made available to lots of users before any retweeting even takes place.\\
As more edges are constructed between users, then the global initial spread of Tweets is increased, and with the addition of retweets, this has an even bigger effect. Although other infleuncing factors have been mentioned earlier, such as the notion of a user's network awareness and of user influence, the organisation of users on the graph and the differences in observed propagation pattern is an interesting avenue for research towards uncovering interestingness.

In this chapter, various user network types are used to simulate retweet behaviour between users on Twitter. The behaviours are studied in order to research the propagation patterns observed in different network structure types. Non-realistic and realistic networks are used to highlight the low-level propagation in these networks and the similarities between more realistic simulated networks and Twitter's own social graph.\\
This research is then used to generate the methodology for estimating Tweet interestingness based on an \textit{expected} Tweet popularity, as is discussed further later in the chapter.


\section{Observing Differences in Propagation Patterns Between Different Network Structures}
In this section, simulations are carried out in three different network topologies - a path (or `linear') network, a random network, and a scale-free network. In the experiments, individual user \textit{decisions} are used as the bases for demonstrating Tweeting and retweeting.  

The simulation algorithm and ideas behind the model used for generating the simulated users' retweet decisions are adapted from the work carried out in \cite{zhu11} and \cite{peng11}, which introduces methodologies for illustrating Tweet spread through a given network of users.\\
From the analyses of the simulation experiments, of interest is whether, and how, changing the network structure does affect retweet propagation patterns, and whether a simulation can mimic Twitter's own bhevaiour in terms of retweet spread.

Measuring retweet behaviour is carried out through studying the distribution of retweet group sizes that result from running the experiments, as is described in later sections.


\subsection{Overview of the Simulation Algorithm}
The algorithm covers the simulation of Tweet propagation through a given set of connected users by emulating retweet decisions of each user who receives the Tweet. The retweet decision is made based on a logistic regression, as is described later.

\cite{zhu11} developed a simulation algorithm which carried this out, which was found to work accurately. This algorithm was modified to fit the purposes of the experiments in this section.\\
In essence, the simulation initially requires a graph of connected users, $U$, and a Tweet, $t$, to be retweeted between them. It begins by initialising a set of users, $S$, to contain the followers of a particular $u_s \in U$, which represents the user $\aut{t}{O}$. As such, users in $S$ are the current set of users to have $t$ on their home timeline.\\
The procedure then iterates over timesteps, at each stage checking the retweet probability of each $u \in S$. If $u$'s retweet probability is suffuciently great for $t$, then $u$ retweets $t$ by being removed from $S$ and then added to $\rt{t}$, which represents the set of users who have retweeted $t$. The followers of $u$ are then added to $S$.\\
A threshold value, $TH$, is used to emulate the idea of Tweet `decay'. The reasoning behind this is that as time goes by, more and more Tweets arrive onto the home timeline, pushing the previous Tweets further down, whether they are interesting or not. Tweets may be ignored and not retweeted if the user has not viewed their home timeline for a while or if the user decides the Tweet is not of a sufficient quality to retweet it. If a Tweet is pushed down to the extent that is out of view, or out of the current page, then the chance of that user retweeting that Tweet is reduced. Thus, if a user is in $S$ for more iterations than specified by $TH$, then the user is removed from $S$, meaning they cannot retweet the Tweet.\\
Users who have retweeted $t$, or are unable to do so (either by having previously retweeted it or by exceeding $TH$) are prevented from being (re-)added to $S$.

The algorithm terminates either when the timesteps exceed the maximum allowed, $T$, or when $S$ becomes empty. This results in the retweet group, $\rg{t}$, which comprises the final set, $\rt{t}$, along with the initial $u_s$. As in the previous chapter, $\rc{t} = |\rt{t}|$.

Therfore, the additional necessary components to run the simulation are the functionality to build the user graph, a constructed Tweet, and functionality for generating a retweet probability for each user who receives the Tweet.

\newfloat{algorithm}{H}{lop}
\begin{algorithm}
\caption{Simulation of retweet decisions in a given network of users}
\begin{algorithmic}[1]
\Procedure{simulate}{graph of users $U$, tweet $t$}
    \State $RT\gets$ empty set \Comment{To hold users who retweet $t$}
    \State $T\gets$ number of timesteps allowed
    \State $TH\gets$ maximum timesteps \Comment{Emulate $t$ `slipping' down timeline}
    \State $us\gets$ source User selected from $U$
    \State $S\gets$ initialise to followers of $us$
    \Statex % new line
    \ForAll{$ti$ in range $(0,T)$}
        \ForAll{$u \in S$}
            \State $P\gets$ retweet probability of $u$ on $t$ in range $(0,1)$
            \State $r\gets$ random number in range $(0,1)$
            \If{$P > r$}
                \State Remove $u$ from $S$
                \State Add $u$ to $RT$
                \State Add followers of $u$ to $S$
            \Else
                \State increment $u.\textrm{TIME\_HELD}$
                \If{$u.\textrm{TIME\_HELD} > TH$}
                    \State remove $u$ from $S$ \Comment{$u$ has held $t$ for too long in timeline}
                \EndIf
            \EndIf
        \EndFor
        \If{$|S| = 0$}
            \State return $RT$ \Comment{No more users can retweet $t$}
        \EndIf
    \EndFor
    \State return $RT$
\EndProcedure
\end{algorithmic}
%\caption{Algorithm to simulate retweet decisions in a given network of users.}
\label{algo1}
\end{algorithm}


\subsection{Generating a User's Retweet Probability} 
As prevoiusly mentioned, \cite{zhu11} used a predictive model for retweet decisions based on a logistic regression, which was demonstrated to be capable of accurately predicting a user's retweet chance on a given Tweet at a given time. The regression was trained on a set of user, tweet and context features in order to classify a likelihood on the binary decision: retweet or no retweet, such that if $P = 1$ then the retweet will definitely occur.


\subsubsection{Machine Learning}
Machine learning is the term given to the family of techniques that allow a program to make predictions for the outcome of unseen instances based on an observed and known history of occurrences. There are many types of machine learning classifiers that are suitable for different purposes, such as for predicting an expected outcome from a set of nominal categories, for predicting a value from a continuous range, or for predicting the \textit{probability} of a binary outcome.

Most machine learning techniques involve the training of a predictive model, which contains the information on known outcomes for a set of features. The model is then used to estimate an unknown outcome, usually with a probability on the \textit{confidence} of the classification, for new sets of instances.

For example, consider three attribute variables, $A$, $B$, and $C$, each of which can be equal to one of two nominal values; \textsc{True} or \textsc{False}. A particular machine learning algorithm trains a model based on its knowledge that;
\begin{itemize}
    \item $A\gets$ \textsc{True}, $B\gets$ \textsc{False} $\Longrightarrow$ $C\gets$ \textsc{True}
    \item $A\gets$ \textsc{False}, $B\gets$ \textsc{False} $\Longrightarrow$ $C\gets$ \textsc{False}
\end{itemize}
Although training of predictive models nearly always involves using more than two instances, the history of these example instances indicate that $C$ is more strongly associated with $A$ than with $B$. As more instances are added showing similar behaviours, then the association becomes stronger, to the extent that the technique will predict $C\gets$ \textsc{True} in instances where $A\gets$ \textsc{True} (and vice versa) with higher confidence.

In this case, $A$, $B$, and $C$ are known as the `features', and a set of such features form the `instance'. Once a trained model has been constructed, the machine learning algorithm will only be able to make predictions using instance features it has knowledge of. For example, if the example technique was now given an instance containing a feature $D$, then the example technique will not know how changes in $D$ will affect $C$'s outcome.

If there is not a strong correlation between the features in a dataset, then the confidence on classification of a particular feature will be weaker. Although this example has focussed on boolean (nominal) data types, many machine learning classifiers are able to work with features that are higher dimensional nominal values, contunuous reals, and so on, and will apply weights to the different features based on their level of influence to other features in the instance.


\subsubsection{The Logistic Regression}
Logistic regression analysis can be used as a machine learning technique for working with binary outcomes based on a set of predictor variables (or features) \cite{hosmer13}, which makes it an appropriate approach for predicting the binary retweet decision. As mentioned previously, logistic regressions have been frequently used in retweet analysis \cite{castillo11} \cite{zhu11} \cite{peng11} \cite{naveed11} \cite{hong11}.

An implementation of the logistic regression algorithm was written in the Python programming language, which formed the basis of calculating the value for $P$ mentioned in the algorithm overview earlier based on a set of feautures of the Tweet and author user.


\subsection{Summary of Training Features}
\cite{zhu11} used the approach in order to accurately model retweet decisions in Twitter. A set of around 50 different features were used to train the logistic regression, with the retweet outcome (\textsc{True} or \textsc{False}) being the predicted classification in each case. These features included tweet-related features (such as content analysis, inclusion of URLs, etc.), and network and user features (followships, mentions, etc.).

Since the network structures themselves, and the propagation \textit{patterns}, are what are of interest in this section, the simulation is significantly simplified by using far fewer features, yet which are features that have been shown to have a strong influence on the retweet decision. As long as a consistent set of feature groups and values are used, the properties of the retweet groups observed should demonstrate the varying behaviours across the different user structures.

As such, each instance comprised the following four features associated with each Tweet, $t$, and where $u$ is the user currently making the retweet decision, \textsc{retweet};

\begin{table}[h]\footnotesize
\begin{center}
\begin{tabular}{ l | c | l }
	Feature & Data type & Description\\
	\hline
	\hline 
	\textsc{follows}    & \{\textsc{True, False}\} & \textsc{True} if $u \in \fos{aut{t}{O}}$\\
    \textsc{followed}   & \{\textsc{True, False}\} & \textsc{True} if $u \in \frs{aut{t}{O}}$\\
    \textsc{mentioned}  & \{\textsc{True, False}\} & \textsc{True} if $u$ is mentioned in $t$'s content\\
    \textsc{url}        & \{\textsc{True, False}\} & \textsc{True} if \texttt{http://} or \texttt{https://} in $t$'s content\\
    \hline 
    \textsc{retweet}    & \{\textsc{True, False}\} & \textsc{True} if $u \in \rt{t}$\\ 
    \hline
\end{tabular}
\end{center}
\caption{Training features for the logistic regression.}
\label{table:logisticregressionfeatures}
\end{table}

The \textsc{url} feature has, in the literature, often been found as a large impacting feature on retweets in Twitter, especially in \cite{alonso10}, who use it as their basis for determining which Tweets are interesting.


\subsection{Training the Model}
In order to train the logistic regression model, data was required from Twitter so that the instance sets of features could be built. 

Data collection for these experiments again utilised Twitter's REST API, which was queried between March and June 2012 to collect a set of around 12,000 Tweets and retweets. Since this time was before the mandatory switch-over to v1.1 of the REST API, the public timeline could again be used to collect the data without the necessity of crawling through the social graph.\\
In this case, it was particularly necessary that non-retweets were also collected in order to provide the negative case when training the regression model, so that there were instances where the \textsc{retweet} feature could be \textsc{False}.

In cases where the collected Tweet was a retweet, further calls were made to the API to determine the relationships between the retweet's author and the original Tweet's author in order to satisfy the required \textsc{follows} and \textsc{followed} features.
Where the collected Tweet was not an instance of retweet, there is no original author to examine the relationships between. In these cases, further Tweets were retrieved for the user in order to find their retweet rate in terms of the ratio of retweets to Tweets on their user timeline and an analysis of the relationship between these and the original authors. This was used in conjunction with the user's follower and friend count to determine a probability of the `faux' followships.

After storage, the regression model was trained using features extracted from the raw data, which the simulation algorithm could then use to generate the required retweet probability, $P$.


\subsection{Running the Simulations}
Once the model had been trained, the simulations could be run. In each simulation experiment, a network of users was generated, as described in the next section, and a Tweet object was created.\\
This Tweet object contained information on whether or not it contained a URL and if it mentioned one of the users in the generated network. 

Various parameters - such as the $TH$, the size of the user network, $U$ to be generated, $t$'s particular parameter for the simulation, and any weightings on the decision probability prediction generator - could be altered to affect the strength or correlation of the patterns produced by the different network structure types.


\subsection{Network Analyses}
In this section, three network structures are assessed in terms of the differences in the patterns of propagation each expresses. Each generated graph is a \textit{directed} graph in order to illustrate the followships between the user nodes, and to support the use of the \textsc{follows} and \textsc{followed} features required in the decision probability calculation.

In each case, the same set of generated Tweets were used, but different structures required the various parameters to be set slightly differently. As such, each network structure will present with different proportionate retweet group size distributions; of interest to this work is the difference in \textit{pattern} of the distribution.


\subsubsection{Path Network}
The first assessed structure was to illustrate the pattern on a non-realistic social network structure; a path network.

Path networks are one of the simplest type of graph, and a linear directional path network consists of the graph of users, $U$, of size $n$, in which each user $U_i \forall 0 \leq i < n$ is followed by user $U_{i+1}$. As a result, each $u \in U$ has precisely one follower and one friend, except the users $U_n$ and $U_1$ respectively.\\
$n$ is the only parameter necessary in the construction of this user graph.

\begin{figure}[h]
\centering
\includegraphics[scale=0.8]{4.Chapter2/Media/path_network.png} 
\caption{Example of a path network.}
\label{fig:path_network}
\end{figure}

In this graph, the size of the retweet group is, by definition, equal to the depth of penetration, as there is only one path (or retweet chain) available for propagation to occur down. As such, in each case, the retweet tree representing a resultant retweet group formed in this type of network will have the same structure as the graph itself, yet with a size dependent on the collective retweet decisions of the users.

Since each internal user has only one follower, the likelihood of a retweet decision being positive at each timestep is somewhat progressively reduced, and thus is much more likely to tail off sooner than in graphs with more propagation avenues. This is also due to the fact that each retweet can only reach an audience of size 1 at each time step, and thus the survival of the retweet cannot rely on a summation of many users' retweet decisions. 

\begin{figure}[h]
\centering
\begin{tikzpicture}
 \begin{semilogyaxis}[
        xlabel=$|\rg{t}|$,
        ylabel=Frequency of occurrence,
        width=7cm,
        grid = major]
    \addplot[only marks,mark=+,blue] plot coordinates {
        (1,540)  (2,118)  (3,53) (4,27) (5,13) (6,5) (7,4) (8,2) (9,1) (10,0)
    };
\end{semilogyaxis}
\end{tikzpicture}
\caption{Frequency distribution of retweet group sizes in path network simulations}
\label{fig:linear}
\end{figure}

The likelihood of a particular user achieving the opportunity to receive the Tweet, in order to then retweet it, becomes the product of the probability function the further it travels through the graph, in which user $U_i$ requires each user from $U_{i-1}$ to first make a positive retweet decision. For example, if each user has probability $p$ of retweeting the Tweet, then each user's chance of retweeting the Tweet is $\frac{1}{p^i}$, where $i$ is the position of the user in the graph.

Therefore, as might be expected, the frequency distribution of retweet group sizes shows a half-life type behaviour demonstrating a logarithmic pattern with many small retweet groups followed by a series of exponentially smaller groups.\\
This user structure illustrates well how some users that might find the Tweet interesting, and who may decide to retweet it, do not even get the chance to view it in order to make the decision. Although this is accentuated in this structure, the same principle applies to any non-complete social graph, and demonstrates how the way users are connected can have a large impact on the retweetability of a particular Tweet.


\subsubsection{Random Network}

\begin{figure}[h]
\centering
\includegraphics[scale=0.8]{4.Chapter2/Media/random_network.png} 
\caption{Example of a random network where $n = 5$ and $p \sim 0.5$.}
\label{fig:path_network}
\end{figure}


The random network was the next user structure to be analysed. Although it is certainly more similar to a real-life social graph than a path network, it is much more basic and uniform and does not consider user communities and clusters or different levels of influence in users in terms of their follower and friend counts.

A random social network is defined as the case in which the graph of user nodes, $U$, and where $n = |U|$, consists of each user, $u$, having probability $p$ of following each other $u_i \in U \forall 0 \leq i \leq n$ and where $u_i \neq u$. Thus, as $p$ is increased, then the likelihood of $u$ following a $u_i$ increases, causing the overall network edge density to increase. In general, therefore, the average number of followers and friends of a user is proportional to $p.n$.\\
The only parameters needed for constructing such a graph are $n$ and $p$.

\begin{figure}[h]
\centering
\begin{tikzpicture}
 \begin{semilogyaxis}[
        xlabel=$|\rg{t}|$,
        ylabel=Frequency of occurrence,
        width=7cm,
        grid = major]
    \addplot[only marks,mark=+,blue]
       file {4.Chapter2/data/random.dat};    
\end{semilogyaxis}
\end{tikzpicture}
\caption{Frequency distribution of retweet group sizes in random network simulations}
\label{fig:random}
\end{figure}

The frequency distribution demonstrates a very large proportion of mid-range values for $\rg{t}$, indicating that Tweets tend to have a consistent spread amongst the network, as might be expected. There are few smaller groups since there are no users that have disproportionately smaller spheres of influence, and each user has many incoming edges and a proportionately similar number of outgoing edges. As such, there are more mid-range retweet group sizes than smaller ones.\\
However, as in any distribution so far examined, the distribution of retweet group sizes must eventually tail off due to the natural eventual reduction in retweet decisions being successfully made as retweet chains increase in length.


\subsubsection{Scale-Free Network}
The final network structure examined in this section is the scale-free network. Also known as `small world', scale-free graphs are generally throught to be representative of the general structure of online social networks \cite{mislove07} and, indeed, they are used to describe the interconnections of real-world properties, such as friendship groups and food webs \cite{guido07} \cite{hein06}.\\
Essentially, scale-free networks dictate that there are a small number of nodes with a high degree and many nodes with a low degree and are usually generated through some form of preferential attachment algorithm. Thus, this type of network supports the notions of user communities and influential users in terms of those demonstrating a disproportionately large follower count. The other user structures studied do not have the scope for emulating this property of interconnection between the user nodes.

Scale-free networks are constructed such that the distribution of the degree of the graph's nodes follow a power-law-type distribution in that the distribution of this degree is lograthimic. \\
For these analyses, NetworkX\footnote{http://networkx.lanl.gov}, a Python graph and networking package, was used to generate directed scale-free graphs of users, which essentially accepts a network size, $n$, and edge density $d$ as the graph construction parameters. 

\begin{figure}[h]
\centering
\begin{tikzpicture}
 \begin{loglogaxis}[
        xlabel=$|\rg{t}|$,
        ylabel=Frequency of occurrence,
        grid = major,
        width = 7cm,
        legend entries={Twitter data, Generated scale-free data}]
    \addplot[only marks,mark=*,red]
       file {4.Chapter2/data/comparison-real.dat};
    \addplot[only marks,mark=*,blue]
       file {4.Chapter2/data/comparison-scale-free.dat};
\end{loglogaxis}
\end{tikzpicture}
\caption{Comparing the retweet volumes distribution from scale-free graph simulation to data from Twitter's graph}
\label{fig:real-scalefree}
\end{figure}

From simulations of the algorithm through these scale-free networks, a logarithmic trend is observed similar to that demonstrated from the `real' Twitter data analysed in the previous chapter and published in \cite{webberley11}, and the similarities in the distribution pattern is illustrated by Figure \ref{fig:real-scalefree}.


\subsection{General Comparison of Propagation Characteristics across Different Graph Structures}
In this section, three different network structures have been compared, and whilst the path network is very unrealistic in terms of a representation of a social network, the differences in propagation behaviour presented by each do show how the interconnection of users on the graph can have a large effect on the spread of a Tweet. A small set of features to govern retweet features were used in order to accentuate the difference made by the user structures themseleves.

This has demonstrated that, in addition to the processes behind a user's individual retweet decision, the eventual spread of a Tweet also depends somewhat on how the original author's local network is arranged. Thus, the retweet decision of each involved user along with the available information pathways provided by the underlying social structure both contribute to the overall retweetability of a Tweet. 

If there are many edges in the network, such as in the case of the random network, then there are many more routes for peopagation to occur down to and from each user, due to the relatively large in- and out-degree of each user node on the graph. This increases the number of users who end up receiving the Tweet and then have the chance to make a retweet decision. This resulted in there being a larger distribution of larger retweet group sizes than smaller ones, before naturally diminishing again. Despite this high throughput of retweets, which provides a high level of information \textit{recall} for the users, the random graph structure is likely to demonstrate a low \textit{precision} in terms of the interestingness of the received Tweets.\\
This is due to the large number of users having the opportunity to retweet the Tweet, increasing the chance that the `noisy' information will be filtered through.

The path network demonstrated very poor propagation, and required that its simulation parameters were altered to facilitate retweet behaviour significantly more than in the other graph structures in order to produce any observable pattern. The results showed that propagation down a single allowed chain cannot be an effective way to spread Tweets, as it required each user in the chain to retweet it so that the successive users can have a chance to view it.

Whilst the scale-free network does not have the same general propagation throughput as the random network, it does demonstrate retweet patterns similar to those observed in data from Twitter's own social graph. This complements the findings of \cite{mislove07} and \cite{hein06} in terms of online social networks emulating real-life social networks having scale-free properties.\\
This type of structure supports areas of the graph with denser communities, as is shown to exist by \cite{java07}, and have the potential for facilitating very large numbers of retweets if influencial users are involved, but illustrate how Tweets `travelling' through less dense areas (and less-influencial) users will not be as demonstrably popular.



\section{Using the Social Graph as a Method for Inferring Interestingness}
The graph analyses in the previous section have demonstrated a method for generating a $\rg{t}$ for a given Tweet, $t$. Since $\rc{t} = |\rg{t}| - 1$, then the simulation algorithm used can be used to estimate a retweet count for a given Tweet. Although it has been discussed that although an individual's retweet decision does generally imply that user's interest in the Tweet, the overall retweet count of a Tweet is capable of denoting that Tweet's popularity. However, this value could be used in tandem with the expected estimated retweet count of a Tweet in order to determine if the Tweet is, in fact, interesting.

This notion is based on the idea that if a Tweet is more popular than expected, then there is something about that Tweet that makes it more \textit{interesting} than similar Tweets that are less popular, such as some breaking news or a link to a controversial article.\\
For example, consider the case of two Tweets, written by the same author, and both containing the same instances of feature values, such as the inclusion of a URL or a mention. If one of these Tweets achieves significantly more retweets than the other, then there must be some non-trivial feature of the more popular Tweet that makes it stand out to the audience, and thus allows it to be perceived as more \textit{interesting}. This is because the features taken into account are very static, and do not take into account any depth of the actual content of the Tweet.\\
Similarly, if most Tweets of a user achieve between one and two retweets, then the expected retweet count for this user's future Tweets is likely to be similar. If, however, the author posts a Tweet which achieves an observed total of 10 retweets, then this is more than what was expected. If a Tweet achieves one or zero retweets, then this is as expected or less than expected, and is therefore not interesting.

As such, a method is proposed based on the following two points;
\begin{itemize}
    \item observed $\rc{t} >$ expected $\rc{t} \Longrightarrow t$ is interesting
    \item observed $\rc{t} \leq$ expected $\rc{t} \Longrightarrow t$ is non-interesting
\end{itemize}

Although it was found, in the previous chapter and in other relevant literature, that pseudo-generated scale-free networks can be representative of Twitter's own social structure, a user's actual own local social network would more accurately portray the links between the users surrounding the original author of a Tweet. By constructing a network based on a user's own local network, then the method would effectively be simulating the Tweets' propagation through the edges representing the followships of the actual users in Twitter's social graph.

Thus, in the simulation algorithm, the user in question is $u_s$, and the intiial value of $S = \fos{u}$. At each timestep, each user in $S$ would have the opportunity to retweet the Tweet, and therefore, by running the simulation, an estimation on the \textit{expected} value for $\rg{t}$ can be obtained, where $\aut{t}{O} = u_s$.\\
In particular, this method follows these steps;
\begin{enumerate}
    \item Select a user
    \item Collect that user's local follower network 
    \item Collect a set of that user's recent Tweets
    \item Construct a network based on the users and edges in the collected network
    \item Simulate the collected Tweets through the constructed network using the simulation algorithm
\end{enumerate}

This procedure would provide an estimated retweet group size for each Tweet, which could then be compared to the actual observed retweet count of the Tweet on Twitter.


\subsection{Data Collection}
Due to the scaling properties of breadth-first traversal of Twitter's social graph, it became infeasible to collect a user's local network containing users more than two edge `hops' away from the source user under the rate limitations of Twitter's REST API.\\
As mentioned, v1 of the REST API allowed 350 calls to the API each hour for each authenticated Twitter account. One call, for example, was required to obtain a list of up to 5,000 IDs representing the follower users of a particular user - the users one hop from the source user. An additional call would then be required to collect each of these user's own followers in order to provide the 2-hop representation of the local network from the source user.

For a user with a follower count of 700, a total of 701 API calls would be required to collect the user's local network within two hops - the one to retrieve the source user's immediate followers, and then one further call for each of the 700 followers. This would take over two hours of collection as it is, and to collect the third hop would require another exponential number of API calls.\\
If each of the 700 followers of the source user has, on average, 200 followers, then this would require  a further $700 \times 200 = 140,000$ API calls, which, in total, equates to over 402 hours of data collection time. Although some follower overlap is likely to be present among the users two hops away, when one considers that this is simply the time taken to collect the local network for one user, then it becomes clear that this must still be an impractical approach.

In the previous chapter it was found that the vast majority of retweets do actually occur \textit{within} two hops of the source user, in that retweet groups produced have a maximum path-length of less than three. In addition, as mentioned, online social networks are `closer' than real-life social networks, and was found to have a value of around four degrees of separation in Facebook. These points help to justify the decision made to class a user's local network as those users and edges existing within two hops from the source user.

In June 2012, the Twitter REST API was used in order to conduct a random walk through Twitter's social graph. Starting at selecting an initial user, an edge expressing the followship of a random follower was chosen in order to select the next user. This continued for each of the selected users in turn and, for each user selected, the most recent 300 Tweets and surrounding information was collected along with that user's local follower network within two hops. The friend network (i.e. the outward edges from each user) was ignored, as only the directional outward flow of information from the source user was useful in this experiment.\\
If, at any stage, the currently selected user did not have any followers, the collection algorithm backtraced to the previous user and another follower was selected instead. The crawler continued until the rate limit for the current request window was met, at which time the current data state was stored, and then waited until the rate-limit was reset before continuing.

The data collection resulted in a set of 33 Twitter users, each with a full local network collected and a set of up to 300 Tweets. In total, around 10,000 Tweets were collected in order to carry out the simulations. It was decided that the previously trained regression model would be re-used as part of the retweet decision engine in this experiment also, and so no further training data was required for collection.



\subsection{Validating Results}



Needed to validate results using human input. Machines themselves are generally unable to express human interests, so results need to be properly evaluated.

\subsubsection{Crowdsourcing}
Discuss crowdsourcing, its uses, how it is useful in this area. Talk about its history (with any references), and then about mechanical turk.
\\
Mention mechanics of mechanical turk, how it is US only (but we used crowdflower - which autmatically handles submission to MT and several other crowd-sourcing services.

\subsubsection{What We Wanted to Assess}
Used MK, etc.

\subsubsection{Constructing the Questions}
Set up questions (i.e. 5 tweets - choose most interesting and least interesting), give example of this.


In order to validate our prediction results, we ran a pilot user study in order to obtain some human input on the interestingness of each tweet. We compiled the tweet data into a set of questions which were submitted to Amazon's Mechanical Turk. Each question consisted of five tweets from our dataset and each Mechanical Turk Worker (MTW) undertook five questions. Each question asked the MTWs to select which tweet was the most interesting of the five, and which was the least interesting.
\\
For consistency we ensured that at least three MTWs had answered each question. When selecting tweets to include in the Mechanical Turk questions, we excluded those which are `@-replies' - i.e. tweets which begin with another user's screen-name and typically form part of a conversation between two or more users. This meant that there were around 4,500 tweets in total in the questions.

Through using the model and simulating each user's tweets through their individual local networks we achieved around 86\% accuracy in correctly predicting the number of times each tweet was retweeted. 
\\
The precision in predicting the \textit{interestingness} of each tweet was around 30\%. While this value is low, it does mean that in 30\% of cases, a tweet that we predicted to be interesting was verified to be interesting by at least three MTWs all selecting one tweet from a set of five. In addition, when simulating the questions by randomly choosing the most `interesting' tweet of the five in each case, the performance was unable to near our precision even after several thousand iterations.

\subsection{Improving This}
idea is there - need better idea for getting expected retweet value! problem is two-fold, but related - need too much data, and can only evaluate users with sparser local networks.
\\
Need offline methodology.
\\
One route for this would be to try and infer a user's local network from a set of their immediate parameters, drawing on our earlier work suggesting that the Twitter network has the properties of a scale-free small-world graph. Through studying graph patterns, it is possible to make sensible inferences on the edges and nodes of a user's local network based on their follower count. From this, a graph edge density can be calculated, $ d = \frac{|E|}{|N|(|N|-1)} $, for use in generating a scale-free network.
\\
Since, for these preliminary experiments, we were only able to collect data from users with a more modest local network, the real and predicted retweet values were both relatively low, allowing more room for error. When simulating much larger local networks involving many more real retweets for each tweet, predicting interestingness, with some threshold value, may become more accurate and thus help improve the precision. The reason for this is that the retweet count of tweets that naturally get retweeted many tens, hundreds, or more times is likely to vary more with interestingness than those that are naturally only retweeted very few times.

\section{Future Work}
There is much further research that could be carried out based on the results in this chapter. Now that the foundation has been laid for simple retweet prediction based on network analysis, research could begin to look at ways in which, as mentioned, networks could be generated based on a few environmental features surrounding users.
\\
This would allow for quick generation of user networks (bypassing the need for data collection) and would also support the same calculations for more highly influential users (users with more followers and more retweets per Tweet).
\\ \\
For this research, the notion of the network will continue and form the basis of the environmental features in the next chapter. Since we now know that the network plays an important role in dictating the way in which information can propagate

\section{Summary}
In this chapter we aimed to carry out a study on the behaviour of propagation through different types of social graph structures and to introduce our ongoing work into predicting the interestingness of tweets from their retweet patterns.
\\
Using a set of tweet and user features, we trained a regression model which we used to simulate a number of tweets through different network types. We produced a distribution of retweet volumes for each network type and confirmed that, with the same tweet features, different network configurations do indeed facilitate different retweet behaviours in terms of propagation spread. We were also able to compare our results to data from Twitter to verify that Twitter's own social graph most closely resembles a scale-free small world graph.
\\
We then finished by discussing how we used the trained model to simulate real networks from Twitter, along with the tweets that were passed through these networks, in order to try to predict how interesting a tweet is based on its retweet patterns. While we were able to often correctly predict the retweet outcome of a tweet, we found that more work would be required to improve the performance of predicting whether or not these tweets are truly interesting to users.

\chapter{Inferring Interestingness of Tweets based on Information Flow Through the Network}


Mention:
\begin{itemize}
\item How this chapter builds upon network stuff in previous chapter
\item We hope to compare and contrast two better ways of predicting retweet volume \emph{and} interestness
\item What needs to be improved (speed, usability - more users with more followers etc.)
\item Why do improvements need to be made?
\item How is this useful, and how does first chapter relate to work done here?
\end{itemize}

Discuss:
\begin{itemize}
\item Does not use network to simulate tweets - instead uses a set of user features
\item Previous chapter shown how basic features can be used to generate a scale-free network, which is what twitter is
\item Use these features as input attributes of a new machine learning technique model.
\item This method does not use a network or model individual user decisions
\item Trained on a set of that particular user's tweets with the retweet outcome of integer type
\item A new tweet modelled with the regression outputs a retweet volume prediction without having to simulate the Tweet's travels through the network. 
\item Discuss about the machine learning approach used (logistic regression and how it works)
\item Talk about the 'binning' of retweet outcome volumes and its approaches (distribution dependent / independent, tables of precisions, etc.)
\item Link 'retweet volume' to 'retweet group size'
\end{itemize}

Research in the previous chapter focussed on researching the effect of the social graph on retweet propagation characteristics. From this, a methodology, displaying a range of various shortcomings, emerged based on the models and simulations utilised in the graph analyses. In this chapter, the methodology is modified with the aim of improving its performance and increasing the range of use-cases it is appropriate for. Since the social structure was found to play an important role in propagation, many network and user features are taken into account throughout the improvements.\\
In addition, modifications are made in order to provide an indication of \textit{how} interesting a piece of information is estimated to be, and more about this particular component is discussed in later sections.

The proposed methodologies also relate to the differences highlighted between a Tweet's raw popularity, as indicated by its retweet count, and hos interesting the Tweet actually is to those who read it. It has been shown that making retweet predictions against models trained with a large number of features can be accurate \cite{zhu11}, but in this work, the focus is more on the Tweet's content and beyond the static features.\\
That is, that when comparing Tweet popularity, then there may be some content, either within the Tweet itself or perhaps in a resource indicated by a URL contained in the Tweet, that makes the Tweet stand out to its recipients and to cause the aforementioned notion of \textit{affective stimulation} \cite{xu07} to its viewers.

Of course, this brings about the notion of information \textit{relevance}, and the fact that the same Tweet could be very boring or irrelevant to one user, and very interesting to another. In this work we focus on \textit{global} (or `average') interest, where interestingness inferences are made for the general case. It is considered that Tweets that are retweeted more than expected within their authors' local networks, relative to the usual retweet count of the authors' other Tweets, are also likely to be of interest to a wider audience, especially since they are now more likely to penetrate through the social graph enough to be received by users in different communities..
 
As such, the focus of the work in this chapter is that of adapting the inference methodology in order to develop a technique for accurately \textit{quantifying} the interestingness of tweets. This is concerning universal relevance in terms of highlighting interesting Tweets from the noise. In particular, there are two main improvements of the previous methodology to be made;
\begin{itemize}
    \item Improve method for generating the \textit{expected} retweet count of a Tweet (in terms of accuracy and range of application)
    \item Expand the binary retweet interesting inference into a more useful scale in order to support \textit{ranking} of interesting information.
\end{itemize}


\section{User Influence}
As has previously been posited, of importance to this work is the difference between the retweet count of a Tweet and the interestingness of the Tweet. An example in the Background chapter was provided, which related to the case of Justin Bieber. His account, \texttt{@justinbieber}, is one of the most influential on Twitter, with nearly 50 million followers at the time of writing. His Tweets receive an average of around 50-120 thousand retweets per Tweets, and they rarely receive fewer than 40,000 retweets.\\
Since an average Twitter user would generally attract a maximum of a few hundred followers, and would normally receive very few, if any, retweets per Tweet. A particularly interesting Tweet from such a user may be retweeted, for example, between 5-20 times. It is therefore apparent that, in the general case, an uninteresting Tweet from an influential user may receive 50,000 retweets, and an exceptionally interesting Tweet from a less-influential user may be retweeted 30 times. It is therefore clear that user influence dictates that this value cannot alone be indicative of Tweet interest.

However, since interestingness \textit{does} have an effect on an a user's individual retweet decision, this absolute retweet count can be used as part of the method for generating an interestingness \textit{score} for a Tweet.


\section{Interestingness Scores}
To address the notion of interest quantification, a scoring scheme is hereby introduced, allowing certain interesting Tweets to be ranked as `more interesting' than other interesting Tweets. This, in itself, is an improvement over the previous method, which allowed only for Tweets to be labelled as `interesting' or `non-interesting'.

Similar to the previous method's \textit{comparison} between the observed and expected retweet counts, the new scoring technique is based now on the \textit{difference} between the two counts. The general idea and potential use-case for this is that if a score is known for a set of Tweets, then these can be used as a basis for ordering information as part of information retrieval or an information delivery system, where Tweets can be displayed to users in a more useful way and where interesting Tweets could be brought forward to users who don't follow the source user or a retweeter, and thus deliver information to an interested user, yet without him or her having to know about it first.

Essentially, the notion scoring stems from the following scenario. Consider two Tweets, $A$ and $B$, which have the following properties;
\begin{itemize}
    \item $\ec{A} = 3000$ and $\rc{A} = 3010$
    \item $\ec{B} = 5$ and $\rc{B} = 15$
\end{itemize}
Where $\ec{A}$ represents the expected retweet count of $A$.

In this case, both Tweets would have been flagged as `interesting' under the previous scheme (although, in reality, the method would not be able to model users who are typically expected to achieve 3,000 retweets). However, it is clear that, despite the \textit{difference} between the counts being equal, Tweet $B$'s observed retweet count is actually much more significantly proportionately greater than what was expected, and is therefore likely to be more significantly interesting.

Since the proportionate difference is the key to this, the interestingness score, $\score{t}$, for Tweet $t$ is hereby simply given by;
\[
\begin{array}{cc}
 \score{t} = \frac{\ec{t}}{\rc{t}}
\end{array}
\]

This provides a positive score where;

\[
\score{t}
	\begin{cases}
		> 1		&	\text{indicates } t	\text{ is interesting} \\
		\leq 1	&	\text{indicates } t	\text{ is non-interesting}
  \end{cases}
\]

And where $\score{A} > \score{B}$ implies that $A$ is more interesting than $B$.

Since this methodology relies on data collection from Twitter in order to obtain the observed retweet counts, it involves extracting a snapshot of the state of the evaluated Tweets at one stage during their lifetime. Since Tweets are not removed over time, unless they are deleted by their author, they can be discovered and retweeted at any time after their composition and posting.\\
The work in this chapter assumes that the most significant portion of retweet activity for a specific Tweet has already occurred by the time the information on the Tweet has been collected. Indeed, the authors of \cite{kwak10} carried out investigative analyses into various temporal retweet brhaviours, and discovered that, on average, that a Tweet receives around 75\% of its retweets within the first day of being posted. 50\% of the retweets of a Tweet take place within the first \textit{hour} of the Tweet being posted.\\
Due to this, and to ensure that the retweet count collected is mostly representative of the Tweet's exraploated `final' retweet count, only Tweets that had been posted at least one day ago were considered for experimentation.


\section{Further Adaptations of the Inference Methodology}
In the previous chapter, it was noted how it was necessary to improve the method used for producing a Tweet, $t$'s, expected retweet count, $\ec{t}$. Problems with the previous method dictated that the method could only work under certain restrictions. In particular, that the user must have a small enough local network (in practice, a follower count of more than 500 or so made the method very unsuitable), and that, due to this, Tweets only attracting very few retweets could effectively be simulated. In addition, the interestingness inferences made were not significantly accurate, although this is likely due to a combination of the aforementioned issue providing much less room for error and the fact that the interestingness decision was only binary.

A new method is hereby proposed for carrying out the prediction for the value of $\ec{t}$. This method is immediately more superior to the previous, as only a very small amount of data (if any) is required to be collected from Twitter. This means that inferences on Tweet interestingness could be made on demand\footnote{Not `live' due to retweet action relies on time to occur.}.\\
Essentially, the method involves creating a classifier model capable of producing a base-line expected retweet count for a given Tweet and its relationship with its author. In this case, the classifier would be trained with the Tweet's actual retweet \textit{count} instead of the binary retweet decision used previously, and it would not require the simulations of the user's local network. Many more features regarding the Tweet, and its conetent, and its author are used to represent the particular user-Tweet information required for generating the predictions.

Since the graph structure clearly has an impact on message propagation, then it was felt that a significant consideration should be made towards including features relating to the interconnection of users, such as follower counts, Tweet rate, and information on a sample of friends and followers. More detail on the features used is provided in later sections.

In general, the newly proposed methodology follows these basic steps:
\begin{enumerate}
    \item Collect sufficent data from Twitter to train a classifier with an appropriate set of features. The trained model is known as the `global' model;
    \item Obtain a Tweet, $t$, and extract its own features as well as information about its author and its author's network properties;
    \item Classify the Tweet's features against the trained classifier to produce a retweet count prediction, $\ec{t}$ for this feature instance;
    \item Calculate \score{t} through using this $\ec{t}$ value and the known $\rc{t}$.
\end{enumerate}

In addition to this `global' model, a `user' model was proposed to be built for each user being evaluated. This user model would be much smaller, as it would only contain information on that user's historical Tweets, but would be capable of providing a second value for $\ec{t}$. With two such values, two scores could be generated by comparing the static $\rc{t}$ to each in turn.

As such, the two scoring mechanisms work as follows;
\[
    \gscore{t} = \frac{\rc{t}}{\ec{t}_1}
\]
\[
    \uscore{t} = \frac{\rc{t}}{\ec{t}_2}
\]


\section{Retweet Volumes as Nominal Attributes}
Most machine learning classifiers are not useful in accurately predicting the outcome of a feature of a large-ranging and continuous data type. Instead, the performance can be greatly improved when predicting from a limited range of discrete ranges, or `nominal' data.

Thus, in order to help improve the accuracy of $\ec{t}$ predictions, it was decided to convert the retweet count feature into a nominal data type for the purposes of training the model and making classifications. By `binning' the retweet counts into categories representing interval ranges, there would be fewer outcome possibilities, and thus the \textit{confidence} of classifcation could be greater.

The values for $\score{t}$ would then be determined through the ratio of $\rc{t}$ to the upper-bound of the nominal range category containing $\ec{t}$.


\subsection{Binning the Retweet Counts}
Since a trained classifier is only generally able to make predictions on features and values it has prior knowledge of, the bin ranges for each category must be equal in both the training feature data and the testing feature data. If the available nominal values for an instance feature representing a Tweet has a different set of category ranges to that in the trained classifier model, then it is likely that a prediction cannot be generated for this instance. It was therefore necessary to consider this when determining a method for binning the retweet counts.

There are various ways in which the counts could be binned, and all begin with a decision on the number of bins to use. The varying performance of this factor is considered later.

Initially, retweets were binned in a \textit{linear} fashion. That is, that the full range of retweet counts in the training set was calculated and then split into bins such that each category had an equal a range as possible. If there were no cases where $\rc{t} = 0$, then a category representing $[0,l)$, where $l$ is the minimum value for the lowest range, was pre-pended to the set of available nominal categories. Similarly, in all cases, the interval $[m+1,\infty)$ was appended to the set of categories, where $m$ is the maximum value in the highest range. This dictates that no Tweet in the training set can have a value for $\rc{t}$ in this category, and thus this allows any Tweet to potentially have $\score{t} > 1$. For example, if a training set of Tweets had a total range of values for $\rc{t}$ being between 1 and 20 was binned into four ranges, then the following interval categories would be applicable:
\[
    [0,1) [1,6) [6,11) [11,16) [16,21) [21,\infty)
\]

Since the distribution of retweet counts (expressed through retweet group sizes) is known \cite{webberley11}, then it is clear that this binning methodology would produce bins containing a very non-uniform distribution of Tweets, where the lower bin ranges would contain many Tweets and the cardinality of each category would decrease exponentially as the ranges become higher. This means that there would be fewer feature instances representing Tweets with larger retweet counts.\\
 Indeed, when training classifiers and running cross-validations on these, this binning scheme demonstrated a high accuracy of predictions on Tweets with lower values for $\rc{t}$ and a low accuracy for Tweets with higher counts. It would be more appropriate, and better address the desire for more universal use-cases expressed earlier in this and the previous chapter, if the accuracy of predictions could be more uniform across the bin ranges.

Various other methodologies were implemented, which eventually evolved into a histogram-based responsive  binning algorithm. Essentially, this algorithm involved is based around the initial calculation for the projected size of each bin, which is based on the total number of Tweets to be categorised and the target number of bins. Each bin was then filled according to the interval range specifying the bounds of that bin, and in such a way such that each retweet count frequency would only be present in one bin. For example, all of the retweets achieving one retweet would be placed in the single bin encompassing this value.\\
As such, after the intervals representing the bin bounds have been produced, then these represent the nominal categories for the retweet count feature in each instance for training and testing against the classifier.



\newfloat{algorithm}{H}{lop}
\begin{algorithm}
\caption{Algorithm for producing intervals for bin categories for $\rc{t}$ values.}
\begin{algorithmic}[1]
\Procedure{generate\_intervals}{set of Tweets $T$, number of bins $B$}
    \State $C\gets$ empty list \Comment{To hold ordered retweet counts}
    \State $I\gets$ empty list \Comment{To hold bin range intervals}
    \ForAll{$t \in T$}
        \State Add $\rc{t}$ to $C$
    \EndFor
    \State Sort $C$ into ascending order 
    \State $M\gets\max(C)$ \Comment{Highest instance of $\rc{t}$}
    \State $T\textrm{Sum}\gets\frac{|C|}{B}$ \Comment{Number of Tweets in each bin}
    \State $H\gets$ empty dictionary \Comment{Histogram of retweet count distribution}
    \Statex
    \ForAll{$c \in C$}
        \If{$c \in H$}
            \State Increment $H_c$
        \Else
            \State $H_c\gets0$
        \EndIf
    \EndFor
    \ForAll{$i$ in range $M+1$}
        \If{$i \in H$}
            \State $s = s + H_i$
        \EndIf
        \If{$s \geq T\textrm{Sum}$}
            \State Add $i$ to $I$
        \EndIf
    \EndFor
    \State Return $I$
\EndProcedure
\end{algorithmic}
\label{algo2}
\end{algorithm}
 
This new responsive method more readily supports more uniform bin sizes, and copes with this by expressing exponentially larger bin \textit{ranges}. As such, the distribution of bin sizes is generally described by a distribution simular to that shown in Figure \ref{fig:bin-hist}. As with the linear method, the interval $[0,l)$ is pre-pended, where necessary, and $[m+1,\infty)$ is always appended in addition to the intervals produced by the algorithm.

\begin{figure}[h]
\centering
\begin{tikzpicture}
\begin{semilogyaxis}[
    symbolic x coords={[0-1), [1-2), [2-3), [3-4), [4-5), [5-100)},
        ylabel=Cardinality of bin,
		xlabel=Bins,
        ybar,
        bar width=7pt,
        yticklabels={,,},
        xticklabels={,,}
        ]
   \addplot[plot 0,bar group size={0}{1}]
        coordinates {([0-1),100) ([1-2),50)  ([2-3),50) ([3-4), 50) ([4-5), 50) ([5-100), 25)};
        
\end{semilogyaxis}
\end{tikzpicture}
\caption{Example distribution of retweet count bin cardinalities under the responsive binning algorithm}
\label{fig:bin-hist}
\end{figure}

This method is responsive in that the bin ranges adapt to the variety and number of retweet counts available, and always attempts to produce a similar number of bins to what is requested. However, due to the disproportinately large number of small retweet groups, the bin sizes cannot be entirely uniform and means that the number of intervals returned will be smaller than the number requested.\\
This also stems from the fact that a single retweet count cannot exist in more than one bin concurrently; for example, if the interval $[0,2)$ existed in a scenario, and the number of Tweets with retweet count equal to 0 or 1 is greater than the value for $T\textrm{Sum}$, which is often the case, then this will result in a larger bin. Without this particular feature, a Tweet may be categorised into more than one bin, causing the prediction accuracy to be reduced. 

Due to this dynamicity, the bin ranges and cardinalities produced by the algorithm vary across different datasets. As a result, the nominal bin categories generated for producing the value for $\uscore{t}$ from the user model trained from the complete set of collected Tweets posted by $\aut{t}{O}$ would be different from those categories generated for a different user. The intervals in each bin category are therefore reflective of the various different number of retweets that each author's Tweets are likely to receive. 


\section{`Twitter is a Memepool'}
In 1976, Richard Dawkins coined the term `meme' to be defined as a ``unit of cultural transmission'' \cite{dawkins76}. The general idea behind memetics is as an analogy to biological genetics except, unlike genes, memes are entirely non-physical and represent a cultural idea or aspect or another human-based behaviour. The rise of social networks on the Internet has allowed the spread of memes to grow to the extent that they are sometimes now even \textit{represented} by physical constructs, such as images.

In genetics, a gene is a physical entity containing information and instructions. It is a unit of genetic inheritance, in that they are passed from parent to offspring through the act of reproduction, and the result of an organism having a gene will express the features represented by that particular gene. These genes contain instructions that make up the features of an individual, such as physical characteristics like eye colour and height, and non-physical characteristics, including various aspects of personality.\\
Organisms exist in an evironment that also has features, such as humidity, altitude, temperature, relations to other organisms, and so on. If the genes of an organism are such that they cause the individual to be well-suited to its environment, then that organism has a better chance of survival and, therefore, a better chance of achieving reproduction.

Memes are similar in that they are effectively made up of a set of features, or `memome', such as the wordings of a particular phrase, or their relevance to other cultural aspects. These enable the meme to be less or more likely to be replicated in different environments, which is made up of the humans exposed to it and the interactions between them. For example, an Internet meme relating to the Star Wars movies would likely have a greater chance of being reproduction, through discussion and reposting, in an environment comprising a set of science-fiction fans than when amongst more mixed-interest groups.

The meme is also a useful analogy in this thesis when describing the way in which Tweets undergo replication within Twitter. Like a meme, a Tweet has a specific set of features, such as the text it contains, the inclusion of any mentions or a URL, and so on, and it exists within an environment consisting of a set of interconnected users on the Twitter social graph.\\
A particular Tweet would generally have a greater chance of `surviving' and being replicated, through the act of retweeting, amongst certain users intereconnected in a particular way than in other environments.

As such, the Tweet features are analogous to the \textit{genes} of a genome, and the arrangement and type of users on the social graph that receive the Tweet and have an opportunity to assist in its propagation comprise the Tweet's \textit{environment}. Since the environment has previously been found to have a large effect on propagation, then these features are useful aspects to include as part of the improved methodology covered in this chapter.


\section{Generating Values for $\ec{t}$}
In order to generate the estimated retweet counts, a trained machine learning classifier is needed to make predictions on a set of feature instances. This section covers an overview of the classifier used for this purpose including a justification in terms of an analysis of its performance.


\subsection{Machine Learning Classifier}
An overview of machine learning classifiers and their processes was provided in the previous chapter. In that case, a logistic regression was used to generate a prediction on a binary retweet decision based on a small number of features. If the retweet count for the Tweet being trained or tested was greater than zero, then the retweet decision would be positive (\textsc{True}). Otherwise, the decision was negative (\textsc{False}).

Presently, the new methodology involves the prediction of a retweet count category from a set of nominal values of greater cardinality than two. As mentioned, the instances of a particular Tweet and its evironment are categorised based on the value of the retweet count of the Tweet. Although this means that a degree of accuracy is sacrificed when training the classifier, it does mean that there are fewer categories for predictions on test Tweet feature instances, providing a higher confidence in each prediction made.

The Bayesian network machine learning classifier was elected for use for the purposes required in this chapter. Use of this classifier in the social media domain is more rare than other classifiers, such as those involving a regression or a decision tree, but was selected for the various reasons highlighted later in this section.

The Bayesian network is an unsupervised classifier since its learning algorithms do not simply determine the class of the outcome, the retweet count, from the attribute features alone \cite{friedman97}. Instead, a probabilistic graph is constructed based on the dependencies between the variables. The variable attributes form the nodes of the graph and edges between the nodes denote the dependencies (or lack thereof) between them.

Thus, in the case of this research, the various Tweet and envirnomental features, including the nominal retweet count, form the nodes in the Bayesian Network. When forming the graph through training, the dependencies and their probabilistic weightings are adjusted so that an expected value for the retweet count can then be `predicted' from the values of all the other variable attributes.


\subsection{Classification Performance}
When selecting classifiers, the Weka\footnote{http://www.cs.waikato.ac.nz/ml/weka} machine learning and data mining toolkit was used to evaluate the relative performance of various types of appropriate classifiers for the task. The classifiers were selected to cover a sample of the range of available classifier categories. Whilst some types may work inefficiently in this scenario, it is likely that they are more efficient when employed in different use-cases.\\
Although the accuracy of prediction was important, it would also be useful for the classifier to be \textit{efficient} in training its model and when testing future instances against it. This is so that this method could be used to produce interestingness inferences on demand and to further improve on the methodologies used in the previous chapter.

\begin{table}[h]\footnotesize
\begin{center}
\begin{tabular}{ l | c | c | c }
	Classifier	& Precision & Accuracy (recall) &  training time (secs.) \\
	\hline
	\hline 
	Simple logistic & 52\% &  56\% & 528\\
    Logistic        & 62\% &  56\% & 18\\
    SMO             & 51\% &  55\% & 1384\\
    Na\"{i}ve Bayesian & 50\% & 44\% & 0.13\\
    Bayesian network & 62\%&  64\% & 0.54\\
    \hline  
\end{tabular}
\end{center}
\caption{The training performance of different machine learning classifiers.}
\label{table:classifierperformance}
\end{table}

The Bayesian network was found to be accurate and time efficient when evaluating the performance of the set of classifiers. A dataset, obtained as part of the general data collection (please see the relevant section below) and which contains a set of over 57,000 Tweets, was identically used in each of the analyses. For each Tweet instance the same retweet count binning scheme was used, and each classifier performed the same number of cross-validations against the same dataset in order to obtain the precision and recall values.

Although the dataset used in this analysis is not the complete set used in practice, the cardinality of the dataset was sufficient to cause the outputs to be indicative of the Bayesian network's relative advantages over the other assessed classifiers.


\subsection{Varying the Cardinality of Nominally-Categorised Retweet Counts}
Applying the continuous retweet count values to produce a set of nominal categories representing interval ranges of the retweet counts requires a certain balance.\\
By reducing the number of target category bins then the classification accuracy increases, but the level of applicability of the eventual interestingness score for the wide range of retweet counts observed would be reduced. Inversely, with too many bins, the classification accuracy reduces, as there would be fewer instances in each category, yet the scores would be apliccable to a wider range of retweet counts.

\begin{table}[h]\footnotesize
\begin{center}
\begin{tabular}{ c | c | c | c }
	Target category count	& Resultant category count & Precision &  Accuracy (recall) \\
	\hline
	\hline 
    1 & 1 & 100\% & 100\% \\
    2 & 2 & 89.3\% & 89.3\% \\
    5 & 4 & 78.8\%  & 74.5\% \\
    10 & 7 & 68.6\% & 65.7\% \\
    15 & 10 & 61.2\% & 56.4\% \\
    20 & 12 & 59.1\% & 52.9\% \\
    25 & 15 & 51.4\% & 47.5\% \\
    30 & 18 & 49.3\% & 45.3\% \\
    35 & 21 & 47.2\% & 43.2\% \\
    40 & 23 & 46.2\% & 42.5\% \\
    \hline  
\end{tabular}
\end{center}
\caption{The effect of varying the number of nominal categories representing retweet counts on the classification performance using a Bayesian network classifier.}
\label{table:binperformance}
\end{table}

Clearly, by increasing the number of nominal categories used, then the relative number of feature instances in each eventual interval decreases. These bins represent the nominal categories that each feature instance is classified as in relation to the predicted retweet count of the instance. Table \ref{table:binperformance} outlines the decrease in classification accuracy observed with increases in target bin count. The dataset used consisted of nearly 67,000 Tweets, also as part of the ongoing general data collection discussed in later sections, and the classifier used to conduct the analysis was a Bayesian network through cross validation.

It has already been discussed that the resultant bin count is usually likely to be less than that requested of the binning algorithm. This is due to the long-tail distribution of retweet counts referred to previously.\\
In the coming experimentations the algorithm tried to produce, where possible, about 10 nominal ranges for use with training and testing against the general global dataset for the purposes of generating the global expected retweet count. Since each user's own retweet count ranges were different, the number of categories were calculated individually for producing the user-centric expected retweet counts as part of calculating values for $\uscore{t}$.


\section{Training and Testing Against the Bayesian Network Classifier}
This section discusses the processes used behind the calculation of interestingness scores for Tweets through the generation of expected retweet counts using the methodologies outlined in the previous sections. Particular focus is lended to the data collection and the features extracted through the resultant data corpora.


\subsection{Collecting the Training and Testing Data}
In order to train the model on a set of Tweets and then use it to make predictions, data was required for collection from Twitter. This data could then be divided up when required, as described below.

Since, in this case, it was necessary to collect the Tweet data along with each Tweet's numeric retweet count, rather than the binary nominal yes/no required in the previous chapter, only the retweets of a particular Tweet that had been created using the button method could be considered. This is because a Tweet's retweets executed using the manual copy and paste method do not contribute to the Tweet's official, and observable, retweet count that is returned from Twitter's API. This is not considered to be a limitation, however, since this factor is used consistently through the training and later evaluation of the trained model.

In March 2013, a random walk was conducted through Twitter's social graph using v1.1 of Twitter's REST API. Although this date was before the mandatory transfer to this version of the API, the crawler method was used in preference over collecting from the public API, which was deprecated and removed in v1.1, so that user data could be collected, as described, for the environmental training features.

The walk originated with one Twitter user, and each stage consisted of focusing on one user before selecting another one from the followers of the currently focused user. As such, the crawler is very similar to that used in the latter sections of the previous chapter.\\
At each step of the crawl, a set of the most recent Tweets authored by the current user were collected. The number of Tweets obtained for each user had various dependencies, such as the user's Tweet-posting frequency and the number of Tweets in total posted by the user. Usually, several hundredf Tweets from each user were yielded. In addition to the Tweet data, information on the user itself was collected as well as on a sample subset of up to 100, if they exist, of its friends and followers.\\
A sample subset of friends and followers was used instead of the complete set for the purposes of efficiency and to address the associated limitation in the previous interestingness inference methodology, yet it still provides an example snapshot of up to an additional 200 users in the author user's local network in order to provide some idea of the activity within the local network both upstream and downstream from the author user. Around ten API calls were required to obtain this information for each user, giving it immediate advantages over the older method. 

The data collection crawl resulted in a dataset containing around 241,000 Tweets authored by 370 unique Twitter users. Of those Tweets, around 90,000 were cases in which the retweet count was greater than zero. The partial datasets as subsets of the complete set, obtained up until various intermediate points of the entire collection, were those used by the classifier and binning performance analyses in the previous section.\\
Importantly, Tweets from many different types of user were collected; from less-active users with very few followers and friends to influential users and celebrities with millions of followers and achieving many thousand retweets. The collection of this range of users will help demonstrate if this new methodology is able to correctly assess a wider range of users and Tweets.


\subsection{Data Corpora}
After collection, the complete data was divided into two datasets; a training set, consisting of 90\% of the entire set, and a testing dataset, consisting of the remaining 10\%. The original set was divided in such a way as to ensure that all of the Tweets authored by one particular user existed in only one of the two resultant datasets. After being used to train the Bayesian network model, the larger dataset was then discarded from use for the rest of the experimentation.

As has been previously mentioned, of interest is the generation of \textit{two} interestingness scores for each Tweet, $t$; one based on a comparison between $\rc{t}$ and an $\ec{t}$ produced from the global model, and the other between $\rc{t}$ and an $\ec{t}$ produced from $\aut{t}{O}$'s user model. As such, each Tweet requires the two models in order to provide the predicted values for $\ec{t}$.\\
To assist with this, individual user datasets were extracted from the testing dataset, each containing information only on that particular user and its local network and its Tweets, from which individual user Bayesian network models could be trained.

The complete testing dataset is referred to as the `global' corpus of Tweets, and each individual user dataset is known as a `user' corpus.



\subsection{Features}
Producing the instances used for testing and training the Bayesian network models involved the extraction of various features from the global and user datasets. Generally, each feature falls into one of three categories; the network features (`environment'), the Tweet features (`genome'), and the author features (representing the author of the current Tweet). The nominalised retweet count is categorised as a Tweet feature.

Generally, the Tweet features follow the same notions as those used in the previous chapter in that they are static and generally binary features describing various aspects of the Tweet's content and metadata. The network features are more variable and describe the ways in which the author's local network is constructed and the activity within it.

Each Tweet is represented by an instance of a complete set of features relating to that Tweet, its author, and its author's local network. As a result, feature instances representing Tweets authored by the same user will share the same values for their network and author features.


\subsubsection{Features for the global corpus model}
The global corpus model is the Bayesian network model representing the classifier trained from the complete training dataset. In this case, a total of 31 features, outlined in Table \ref{table:globalfeatures}, were used to train the classifier.\\
As such, there were around 217,000 Tweet instances using this feature scheme used for traning the global classifier.

\begin{table}[h]\footnotesize
\begin{center}
\begin{tabular}{ c | l | c }
	 Feature category	& Feature & Feature data type \\
	 \hline
	 \hline 
	& mention & \{True, False\}\\
    & Tweet length & real (numeric)\\
    & url & \{True, False\}\\
  	& hashtag & \{True, False\}\\
  	Tweet & positive emoticon & \{True, False\}\\
  	(`genome')& negative emoticon & \{True, False\}\\
  	& exclamation mark & \{True, False\}\\
  	& question mark & \{True, False\}\\
  	& starts with `RT' & \{True, False\}\\
  	& is an @-reply & \{True, False\}\\
    & \textbf{retweet count} & \textbf{[dynamic nominal]}\\
  \hline                        
	& follower count & real (numeric)\\
    & friend count  & real (numeric)\\
	Author & verified account & \{True, False\}\\
	& status count & real (numeric)\\
	& listed count & real (numeric)\\
    \hline
  	&  max. follower count & real (numeric)\\
	&  min. follower count & real (numeric)\\
	& avg. follower count & real (numeric)\\
    Network & max. friend count & real (numeric)\\
	(`environment') & min. friend count & real (numeric)\\
	& avg. friend count & real (numeric)\\  
	& avg. status count & real (numeric)\\  
  	& proportion verified & real (numeric)\\  
  \hline  
\end{tabular}
\end{center}
\caption{Features used to train the model from the global data corpus.}
\label{table:globalfeatures}
\end{table}

The network features listed apply to both samples of the followers and friends retrieved for each author user during the data collection. For example, the first feature of this category, `max. follower count', represents two features referring to the maximum follower count observed across the sample of the user's followers and the sample of the user's friends respectively.

It should be noted that although the Tweet features, aside from the retweet count as has already been discussed, are permanent after the Tweet has been created and posted, the author and network features are more dynamic due to the continuous mutations in the social graph as edges representing followships are constantly being formed and broken between the user nodes. In this thesis, it is assumed that changes to the features representing these factors were not significant over the period of posted Tweets for each user, and the effect is minimised through consideration only of the recent Tweets of each author user.


\subsubsection{Features for individual user models}
Since the author and network features have identical values in the instances representing all of the Tweets from one particular user, then these features were not considered when training and testing using the user models.\\
As such, the 10 Tweet features were those used in the feature instances in training, and testing against, each user model.

 

\subsection{Testing Against the Trained Classifier}
Once the feature extraction was completed and the instances were built for the global training dataset and each individual user set, the models were trained as described above.

In order to produce the expected \textit{global} retweet counts, each Tweet $t \in T$, where $T$ represents the entire testing dataset, had its features extracted, less the retweet count nominal category, and was evaluated against the global model. This classified each Tweet into one of the categories given by that Tweet's predicted retweet count, and the top interval of the expected retweet outcome category became the \textit{expected} retweet count for that particular Tweet.\\
Similarly, the expected \textit{user} retweet counts were produced in the same way, but instead each Tweet was classified by the user model associated with that Tweet's author.

In each case, the two interestingness scores for each Tweet could be calculated, based on the process described earier, using the ratio between the Tweet's two expected retweet counts and its \textit{observed} retweet count stored as part of the data collection from Twitter. This meant that each Tweet $t \in T$ had two numeric scores, $\uscore{t}$ and $\gscore{t}$, assigned to it. 



\section{Initial Validations of the Scoring Methodologies}
Similar to in the previous chapter, tests are required in order to ensure the validity of the interestingness scores applied to each of the Tweets in the testing dataset. By running these validations, the relative performance of the scoring mechanism can be assessed, and the comparative performance of the two scores, $\uscore{t}$ and $\gscore{t}$ can be evaluated.


\subsection{Planning the Validations}
It was decided that crowdsourcing through Crowdflower and Mechanical Turk, again, would be used to validate the new scoring mechanism, as this would facilitate interestingness evaluations from a wider range of human input. The MTWs taking part would not be associated with the collected Tweets in any way, and thus this assists in the identification of the non-noisy Tweets that are `globally' interesting and are those that the scores have theoretically determined as `interesting'.

Certain Tweets and users were removed, at this stage, from the dataset of Tweets to be assessd by the MTWs. Since the Tweet data was collected through a random crawl through the Twitter and no checks were placed on the crawler at that stage, there was no governance over the content of the text in each Tweet of the data. Thefore, users who frequently used offensive phrases or wrote Tweets in non-English had their Tweets removed. The reasoning behind the latter is based on the fact that the Mechanical Turk microtasks were submitted to be completed by people living in the USA.\\
As before, individual Tweets that were `@-replies' were also stripped so that only Tweets intended to be broadcasts were included in the final MTW test set.


\subsection{Validating the Methodology Outcomes}
In the context of this validation scheme, the MTWs were random account-holders on Mechanical Turk and had no connection to the Tweets they were evaluating. By not determining the humans to make the assessments, a more diverse opinion on the interestingness can be achieved, as the different users will have varying considerations on what contstitutes `noise' and will therefore form a more diverse opinion and further reinforce a decision when multiple MTWs form agreements on what is interesting. 

The validations were carried out such that the MTWs were presented with a series of questions, each of which consisting of five different Tweets from \textit{one} specific author. As such, Tweets were assessed against others than had been posted by the same user. In each question, the MTWs were asked to select the Tweets that they consider to be the most interesting of the group, and that they must select at least one Tweet for each question. For each judgment, where a judgment is one question answered with one or more Tweets selected, MTWs were paid \$0.05.

The test was conducted under the condtitions of a randomised controlled trial. To this end, each Tweet was assessed in three different contexts, in that it would appear in three different questions alongisde four other randomly chosen Tweets, and that each question would then be judged by three different MTWs.

From the stripped testing dataset, 750 Tweets were selected, at random, to be filtered by the author user into the questions to be assessed on Mechanical Turk. Since each Tweet was to appear in three different questions and since each question consisted of five unique Tweets, then this resulted in a total of 450 different questions. Each of these questions was then judged by three different MTWs.


\subsection{Outcomes From the Validations}
The validation test involved contributions from 91 different MTWs, demonstrating the wide diversity of human input attainable validations employing crowdsourcing in this way. From these MTWs, 325 of the 450 questions in total asked had responses where a Tweet was selected with a confidence of two-thirds or greater. Since the MTWs had the opportunity to select more than one Tweet of each question to be the most interesting, there were 349 Tweets of the original 750 Tweets, denoted as $T' : T' \subset T$, that were selected as sufficiently interesting by the MTWs. Tweets selected from individual questions that did not have sufficient confidence were discarded.

The remainder of this section analyses the validation data in various ways to demonstrate the strengths and weakenesses of the interestingness score inferences.\\
Of immediate notice was the comparitive difference between the two different scoring mechanisms for each Tweet $t$; $\gscore{t}$ and $\uscore{t}$. The inference validation results are not significant between the use of the two scores in any of the analyses conducted. As such, the following analyses concern only the use of $\gscore{t} \forall t \in T'$.


\subsubsection{General Performance}

\begin{figure}[h]
\centering
\begin{tikzpicture}
\begin{semilogyaxis}[
    symbolic x coords={{[0,1)}, {[1,2)}, {[2,3)},{[3,4)}, {[4,5)}, {[5,100)}}, % inside braces to support commas in intervals
        ylabel=Proportionate frequency,
		xlabel=$\gscore{t}$,
        ymin=1,
        legend pos=north east,
        legend style={nodes=right},
        ybar,
        bar width=7pt,
        legend entries={ Chosen Tweets ($T'$),  All Tested Tweets ($T$)}
        ]
   \addplot[plot 0,bar group size={0}{2}]
        coordinates {({[0,1)},76.30057803) ({[1,2)},7.514450867)  ({[2,3)},4.335260116) ({[3,4)}, 1.445086705) ({[4,5)}, 2.023121387) ({[5,100)}, 6.936416185)};
        \addplot[plot 1,bar group size={1}{2}]
        coordinates {({[0,1)},80.94365552) ({[1,2)},6.596426935)  ({[2,3)},3.710490151) ({[3,4)}, 1.099404489) ({[4,5)}, 0.961978928) ({[5,100)}, 4.634448007)};
        
\end{semilogyaxis}
\end{tikzpicture}
\caption{Proportionate frequency distribution of $\gscore{t} \forall t \in T$ compared to only those $\gscore{t} \forall t \in T'$}
\label{fig:hist}
\end{figure}

Of the subset $T'$, the scoring mechanism found 140 of the Tweets to have a value of $\gscore{t} > 1$, and thus inferred as interesting. Of these, 65\% were agreed on as interesting by the MTWs. The performance of the $\uscore{t}$ was worse in providing a 55\% agreement, resulting in a general of 60\% agreement on the mean of the two scoring schemes.\\
It is also demonstrable that the proportionate frequency of Tweets with higher values of $\gscore{t}$ is greater in the subset $T'$ than in $T$. This implies that, on average, the MTWs were selecting and agreeing on Tweets being interesting that had a higher score than those that were \textit{not} selected. Further to this, there is a greater proportion of Tweets with $\gscore{t} < 1$ in $T$ than in $T'$, and a greater proportion of higher values for $\gscore{t}$ in $T'$ than in $T$.

This means that, in general, the humans were marking a greater number of Tweets as interesting that were inferred as interesting by the scores than ones that \textit{weren't} inferred as interesting. Although this demonstrates a clear advantage on the binary inference of interestingness over the methodologies in the previous chapter, this analysis does not consider how well the scheme is able to \textit{rank} Tweets in order of interestingness.


\subsubsection{Per-Question Performance}

In order to assess the ability of the scores to effectively rank Tweets in order of inferred interest \textit{level}, the Tweets were studied on a per-question basis.

\begin{figure}[h]
\centering
\begin{tikzpicture}
 \begin{axis}[
        xlabel=Chosen Tweet in top \textit{n} of ranked Tweets in question,
        ylabel=Likelihood of occurrence (\%),
        grid = major,
        legend entries={where $\gscore{t^q_1} \geq 0$, where $\gscore{t^q_1} \geq 1$, where $\gscore{t^q_1} \geq 2$, where $\gscore{t^q_1} \geq 6$, random performance},
		legend style={at={(2,0.5)}},
		xmin=0, xmax=5,
		ymin=0, ymax=100	,
		width=8cm
		]
	\addplot[mark=+,black] plot coordinates {
        (0,0) (1,50.8) (2,85.3) (3,94.3) (4,98) (5,100)
        };
        
	\addplot[mark=*,cyan] plot coordinates {
        (0,0) (1,32.5) (2,65.2) (3,86.95) (4,95.4) (5,100)
        };
    \addplot[mark=*,orange] plot coordinates {
       (0,0) (1,30.4) (2,56.3) (3,81.1) (4,94.15) (5,100)
    };
    \addplot[mark=*,red] plot coordinates {
        (0,0) (1,30.9) (2,53.2) (3,75.7) (4,88.6) (5,100)
    };
    \addplot[gray] plot coordinates {
        (0,0) (1,20) (2,40) (3,60) (4,80) (5,100)
    };
\end{axis}
\end{tikzpicture}
\caption{The probability of a selected Tweet's $\score{t}$ being in the top \textit{n} of \textit{ranked} Tweets for that question. Also illustrating the effect of raising the minimum allowed $\gscore{t^q_1} \forall q \in Q$}
\label{fig:score-dist}
\end{figure}

These questions are those that were assessed by the MTWs and where a particular question, $q \in Q$, where $Q$ is the set of all 450 questions, is comprised of a set of Tweets such that $|q| = 5 \forall q \in Q$.\\
In order to conduct this analysis, each question $q$ had its five Tweets $t \in q$ ranked in order of ascending $\gscore{t}$, such that;
\[
    q = (t^q_1, t^q_2, t^q_3, t^q_4, t^q_5)
\]
and where; 
\[
    \gscore{t^q_1} \geq \gscore{t^q_2} \geq ... \geq \gscore{t^q_5}
\]
In questions where $\sum\limits_{i=1}^5 \gscore{t^q_i} = 0$, the question was discarded from the analysis.



For conducting these question-based analyses, each question's five Tweets were ranked in order of ascending mean interestingness score such that $TS_{Avg}(t^q_1) \geq TS_{Avg}(t^q_2) \geq ... \geq TS_{Avg}(t^q_5)$. We then calculated the number of times the MTWs chose a Tweet that appeared in the top $n$ of the ranked list of Tweets, as shown in Figure \ref{fig:score-dist}.\\
In this figure, we vary the minimum allowed value of $TS_{Avg}(t^q_1)$ (the highest Tweet score in question $q$) to show how detecting more interesting Tweets was more accurate when the range of scores in each question is more disparate. We show, for cases in which $TS_{Avg}(q_1) \geq 1$, that the likelihood of the MTWs choosing one of our two most highly ranked Tweets of the question using the interestingness predictions is around 66\% and the chance that they choose one of the top three ranked Tweets is 87\%.

\subsubsection{Probability of Selection}

\begin{figure}[h]
\centering{
\begin{tikzpicture}
\pgfplotsset{every axis plot/.style={line width=2pt}}
 \begin{axis}[
        xlabel=$ x $,
        ylabel=$P(\text{t chosen} : TS_G(t) > x)$ ,
        grid = major,
        height=5.2cm, width=8cm,
        xmin=0, xmax=4
       ]
	\addplot+[smooth, mark=none]  table {5.Chapter3/data/cum-dist-score.dat};
\end{axis}
\end{tikzpicture}
}
\caption{The probability that Tweet $t$ is chosen given that $TS_G(t)$ is greater than a given value, $x$.}
\label{fig:score-cum-dist}
\end{figure}

Finally, we show that the probability of MTWs deciding a Tweet $t$ is interesting becomes higher as the value of $TS_G(t)$ increases. \\
In Figure \ref{fig:score-cum-dist}, although we exclude cases of Tweets with $TS_G(t) > 4$ to reduce noise (due to fewer samples), a significant increase in probability is observed, particularly in the interval 0-1 representing the range of Tweets that are uninteresting (fewer observed retweets than predicted) to `as expected' (observed retweet volume is equal to predicted). It is also illustrated that, from this initial work, Tweets with a predicted interestingness score of 3 or more are not significantly different from one another in terms of their `real', human-judged, interestingness. However, more work will be carried out towards this in the future so that more accurate research can be done on Tweets with scores of greater then 3 in this context.


\subsection{Further Analyses}
The interestingness scores have been validated in terms of there being recognition that they can signify interesting Tweets. This section will now continue onto some deeper analyses of the results in order to show \textit{how} it is able to work.

\subsubsection{Discerning Interesting Information from Noise}
In this subsection, the human-selected interestingness selection will be assessed. Of particular interest is the likelihood of humans agreeing on an interesting piece of information and the properties of the Tweet scores in questions when agreements \textit{are} made. \\
For this, analyses were made into the \textit{disparity} of scores for Tweets. That is, the range of scores of Tweets in a particular question and the effect this has on human decision in deducing interesting information.

\begin{table}[h]\footnotesize
\begin{center}
\begin{tabular}{ c | c | c | c }
	 Num. confident answers in $q$& min. $d^{Avg}(q)$ & max. $d^{Avg}(q)$ & avg. $d^{Avg}(q)$ \\
	 \hline
	0 & 0 & 846 & \textbf{17.6} \\
	> 0 & 0 & 1445 & 32.1 \\
	1 & 0 & 1445 & \textbf{34.3} \\
	> 1 & 0 & 4 & 0.647 \\
	> 2 & 0 & 0.55 & 0.204
\end{tabular}
\end{center}
\caption{Absolute $TS_{Avg}(t)$ disparity of questions with varying number of confident answers made. Entries in \textbf{bold} are used to highlight interesting values.}
\label{table:score_disparities}
\end{table}

The absolute Tweet score disparity for a question, $q$, is defined as $d(q)$. For example, for a disparity of average scores, this is calculated thus;

\[ d^{Avg}(q) = \max(TS_{Avg}(q)) - \min(TS_{Avg}(q)) \]

Table \ref{table:score_disparities} shows how the values for $TS(q)$ vary for questions with differing numbers of confident answers. A confident answer, as mentioned previously, is one where at least two MTWs have agreed on an interesting Tweet.\\
The data shows that the average $TS(q)$ is around double in cases where a question is answered with precisely one confident answers than in cases where there are no confident answers made. This indicates that a wider scale of interestingness in a question is useful for humans for picking out the content they'd prefer to read. If several pieces of content are more similarly interesting (or, as the case may be, uninteresting), then it becomes more difficult for an agreement to be made on which information is the \textit{most} interesting.\\
In addition, the average score disparities in cases where multiple confident answers were selected are very low. This helps to reinforce the notion that pieces of information that are very similar in terms of interest level make it hard for users to decide on the \textit{most} interesting. Indeed, in questions where this is the case, MTWs have selected, and agreed on, multiple Tweets.

 \begin{figure}[h]
\centering{
\begin{tikzpicture}
\pgfplotsset{every axis plot/.style={line width=2pt}}
 \begin{axis}[
        xlabel=Average $d^{Avg}(q)$,
        ylabel=Cumulative probability of a confident selection being made,
        grid = major,
        height=7.5cm, width=13cm,
       ]
	\addplot+[mark=none]  table {5.Chapter3/data/cum-question-disparity.dat};
\end{axis}
\end{tikzpicture}
}
\caption{The probability of a confident selection being made for question $q$ with varying $d(q)$.}
\label{fig:cum-question-disparity}
\end{figure}

To take this further, it is demonstrable that the probability of a confident selection being made for a particular question, $q$, increases as $TS(q)$ also increases (Figure \ref{fig:cum-question-disparity}.  Thus, this reinforces the notion that people find it easier to discern interesting information when compared to \textit{un}-interesting information. This, too, is highlighted in Table \ref{table:score_disparities_2}, in which it is shown that amongst \textit{all} questions (i.e. not only questions that have been confidently-answered) the score disparity is much smaller between Tweets that were selected than the score disparity for the entire question. \\
This is particularly the case in which there are a few Tweets which have similarly high scores amongst Tweets which collectively have \textit{lower} scores. Therefore, selecting confidently from the few Tweets with the similar scores become difficult, but it is demonstrated that these at least are \textit{more} interesting than the ones that weren't selected.\\

\begin{table}[h]\footnotesize
\begin{center}
\begin{tabular}{ l || c | c | c }
	   & $TS_G(t)$ &  $TS_U(t)$ &  $TS_{Avg}(t)$\\
	 \hline
	$TS_d(q)$ & 62.4 & 4.7 & 33.3\\
	$TS_{d_C}(q)$ & 35.3 & 3.1 & 19.0\\
	\hline
	Ratio & 57\% & 66\% & 58\%
\end{tabular}
\end{center}
\caption{Score disparity comparison between selected Tweets of question $q$ and \textit{all} Tweets in $q$ when using the three different Tweet scores as metrics}
\label{table:score_disparities_2}
\end{table}

For example, this data shows that, on average, the global score disparity for selected Tweets of a question, $q$, was only around 57\% that of the entire disparity of $q$.



\section{Addressing Information Relevance}
- Promising results from the controlled trial - how does it work with people assessing Tweets from their \textit{own} local network
Twinterest stuff



Mis-calculation errors when validating the interestingness predictions:
 - caused by non-authorisation to collect the data from Twitter (i.e. user has a protected account). Therefore we cannot build the test (or train data) successfully for these tweets
 - if an experimenter selected one of these Tweets, then we have to discard that timeline from our analyses.


\chapter{Assessment and Conclusions}
This chapter includes an overview and assessment of the work conducted in this thesis, bringing together the ideas from the initial research and how these have helped in developing the methodologies introduced in later chapters. The validations from the methods are further assessed, followed by how the research forming them may be taken further in potential future projects. Finally, an overview of the thesis in terms of its contributions is described.


\section{Analysis of Research and Results}
Following is an analysis of the research carried out over the stages described in the main chapters of this thesis, from the initial research into retweeting and the social graph through to the interestingness inference methodologies explained at later stages.

\subsection{Summary \& Analysis of Initial Research}
Initial research was conducted into the act of retweeting and Twitter in general for the purposes of providing a background and foundation for the later work. In particular, research was carried out into retweet \textit{groups} and the properties and behaviours they demonstrate.

It was found that, agreeing with other research in the area, retweet groups, which represent the set of retweets (and their authors) of a particular Tweet, can have widely ranging sizes and depth. The path-length of a retweet group's branch was defined as the number of retweet hops between the author of the initial Tweet and that of the final retweet of the chain. This phenomenal takes into account that retweets can, themselves, be retweeted, and that retweet groups do \textit{not} consider the followships between the set of user's they represent. Retweet groups were found to present an average \textit{maximum} path-length of around two, and the longest maximum path found in the dataset collected from retweets on the public timeline was of length nine. This demonstrates a significant penetration through the social graph, especially considering the `real' world's six degrees of separation, and that social networks often exhibit a social graph even more closely connected than this.

This, and other retweet group analyses, led more onto more focused research on Twitter's social graph, which began by studying the properties of the \textit{audience} size attainable through retweet groups, and the overhead generated through shared followships, and ties between users involved in retweet groups and their relationships on the social graph. It was found that the chance of a retweeter retweeting a Tweet was significantly greater to occur in cases where the retweeter follows the author of the original Tweet, but that as retweet pathways become longer, the chances of the final retweeter following the original author diminishes over the distance, demonstrating strong correlations between the edges between users on the retweet graph and those on the social graph. These experiments were conducted using a trained logistic regression to predict a retweet outcome decision for each user who received the Tweet during simulations of each structure type.

Since, at this stage, it was demonstrable that the social structure of Twitter clearly affects the propagation of retweets, and that this property could provide a way of inferring interestingness. Research then focused on examining the differences in propagation patterns in order to demonstrate that each structure type can present very different retweet propagation patterns. Because the propagation pattern difference at this structural level was so large, it was decided that this could be a basis for an interestingness inference methodology. This method utilised the same research and algorithms behind those used in the graph structure analysis to predict a retweet count for a given Tweet within a graph of connected users, and worked through a simple comparison between this predicted value and the \textit{observed} retweet count of the Tweet. This method was not shown to perform particularly well in the validation tests conducted, and thus improvements were necessary before any further analyses were made.


\subsection{Summary of the Improved Methodology}
Improvements over the previous methodology were based around the introduction of interestingness \textit{scores}, with which Tweets could be ranked according to the ratio of their observed and expected popularities, and where if the observed popularity is proportionately larger than the expected popularity, the score for that Tweet would also be proportionately greater. This in itself provides many benefits over the previous system, which was unable to provide any indication over \textit{how} interesting a Tweet is.

The prediction of the esitmated retweet count was altered so that they could be generated directly through the use of a Bayesian network machine learning classifier, which made predictions based on a larger set of Tweet and environmental network features. These features could be collected much more efficiently from Twitter's REST API, illustrating another advantage in terms of the ease with which predictions (and thus score assignations) can be made. 

Furthermore, the efficiency stretches to providing a more universal approach, allowing Tweets from most users on Twitter to be evaluated equally and on the same scale, since the complexity of any part of the assignation process is not affected by the influence or other properties of the author user. 


\subsection{Analysis of Validation Results}
Producing the scores partially relies on the initial accuracy of making retweet predictions in the \textit{general case}, using cross-validation tests on the Bayesian network classifier and the binning policy of retweet counts explained in the prevoius chapter. The performance of each factor is highlighted in Tables \ref{table:classifierperformance} and \ref{table:binperformance} respectively, which demonstrated that a decent precision and recall could be obtained in cases using up to between 15 and 20 nominal bins. Although the binning algorithm was responsive in terms of the range and distribution of retweet counts in the set being analysed, it was aimed to achieve a projected bin count of 10 in the case of producing the global model. When generating the user-specific classifier models, the number of bins ranged significantly depending on the relative influence of each user, as described previously.

The mention of the ``general case'' is important, since the methodology is designed to discover Tweets which do \textit{not} fit this case, as these would be Tweets which have a greater (or smaller) retweet count than expected, and would therefore be the Tweets which would contribute negatively to the aforementioned performance analyses of the prediction method. As such, if all Tweets fit their general cases as given by their features and the features of their authors, then the general performance of the cross-validations on the classifier could be greater.

Two main human validation tests were conducted into the performance of the scoring mechanisms provided by the improved methods; one based on interestingness decisions from non-related participants, and another based on decisions from Twitter users to whom the Tweets assessed are more relevant, as denoted by the followships of the author users. These validations expressed a good performance of the scoring scheme in a variety of ways, from the ranking of Tweets in order of interestingness through to analyses into the motivation of Tweet selection from the \textit{disparity} of Tweet scores in the timeline.

The background chapter of this thesis described other similar research in this area along with the strengths and weaknesses of each. Whilst this included a fair amount of research into retweet decision and count predictions, they are often quite similar to one another, and these goals are not the primary focus of the work in this thesis. Instead, research into information interestingness with regards to Twitter will now be evaluated against the methods outlined in this thesis.

\cite{gransee12} introduced a system for scoring Tweets based on a na{\"i}ve Bayesian classifier. The authors' learner was concerned only with textual cues for producing a score. The learner was trained using a set of Tweets from a particular author, with each Tweet being assigned a score based, similar to the work in this thesis, on the distance between the observed retweet count of the Tweet and the single \textit{baseline} retweet count for the Tweet's author at that particular time. Words in unseen Tweets are then scored individually according to the scores of Tweets the words have previously been seen in, which, when averaged, generates a score for the unseen Tweet.

The scoring methods in this thesis are superior to this methodology in a number of ways. 

- limited to a set of 175 users with no indication of its performance in the general case 
- requires a dictionaruy of words to be built for each user based on all of their previous tweets (ours can be used on-demand through comparisons to global model)
- cannot be used on-demand scoring of tweets
- maintain a baseline retweet count at granular time intervals



Background described other similar research in the area - (e.g. if URL = interesting, etc.) - how is this better?



\section{Further and Future Work}
How can this research be taken further in the future?

\begin{itemize}
\item Use previous results to predict how far a tweet is likely to be retweeted (for advertising purposes)
\item Useful for detecting the kind of messages that are likely to travel further
\item As well as providing an interest level, the systems also predict sensible estimations on retweet volumes.
\item Perhaps useful for measuring the spread of rumours.
\end{itemize}

% this stuff:
One route for this would be to try and infer a user's local network from a set of their immediate parameters, drawing on our earlier work suggesting that the Twitter network has th    e properties of a scale-free small-world graph. Through studying graph patterns, it is possible to make sensible inferences on the edges and nodes of a user's local network based o    n their follower count. From this, a graph edge density can be calculated, $ d = \frac{|E|}{|N|(|N|-1)} $, for use in generating a scale-free network.

- remove links between users, do still receive the information - (future work?)


\section{Conclusions}
\subsection{Summary}
Summarise events and processes covered, reiterate what the point of the work was and how each part of the work covered relates to that.

\subsection{Contributions}
Restate the original contributions (from Introduction section). Explain the ways in which the work done relates to the projected contributions, that it is novel and useful.
contributions:
- a survey into relevant literature of the area
- analysis of current interestingness inference techniques
- thorough research into retweet properties and its ties to users on the social graph
- a method for suitably predicting estimated retweet counts, leading to...
- ... a method for suitable inferring the level of interest of a particular Tweet.


%\chapter*{Appendices}
%Source code, further diagrams, ideas, etc.

%%%%% BACK STUFF 
\backmatter

\def\baselinestretch{1.24}\normalfont

\bibliographystyle{plain}
\bibliography{includes/library}

\end{document}
