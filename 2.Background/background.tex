\section{Information Propagation through Retweeting}

Introduction to Twitter, act of Tweeting and retweeting information. Ideas:
\begin{itemize}
\item endorsement of info - retweets / likes / etc.
\item Retweeing act of forwarding on of information to those who follow you
\end{itemize}

\section{Domain Context}
Go into more information regarding information propagation in Twitter and about the mechanics of Twitter. Ideas:
\begin{itemize}
\item Retweet volumes
\item communities and how networks form
\item `search bubbles'
\item propagation
\item social structure (and its effect on propagation)
\item information quality, relevance, retrieval, filtration
\item information interestingness
\item categorise above into subsections (possibilities: the network and communities, information retrieval and relevance, information retrieval and bubbles, and retweeting as a form of propagation)
\end{itemize}

\section{Research Motivation}
Link to `The Problem' in introduction chapter. Talk about wanting to allow people to be exposed to information that they are \textit{likely} to find interesting based on the interestingness of the tweet, but without them having to search for the information or follow the users responsible for sourcing or forwarding the information. \\ \\
Further work would be done on refining this based on a per-user basis (i.e. that user's particular interests as a relevance metric for the interestingness of the Tweet).