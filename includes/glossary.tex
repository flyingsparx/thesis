\chapter*{Glossary} 

{\bf Follower}
A type of user. A user, \textit{x}, is a follower of user \textit{y} if user \textit{x} follows user \textit{y}. Other users who follow a particular user will receive all of the user's Tweets and retweets to their home timeline. A user can elect to follow another user.

{\bf Friend}
The inverse of follower. User \textit{x} is a friend to user \textit{y} if \textit{y} follows \textit{x}.

{\bf Path-length}
The penetration of a Tweet - i.e. the number of times a Tweet is retweeted down one chain. The final retweeter in the chain indicates the number of hops the Tweet has taken from its author.
			
{\bf Retweet}
\textit{n.} - A replica of a Tweet, which has been forwarded on by a user (who is not the Tweet's original author) to their own followers. \\
\textit{v.} - The act of replicating a Tweet. A user who finds a Tweet interesting may retweet it so that it gets more exposure.
									
{\bf Retweet Group}
Set of Twitter users responsible for the propagation of a Tweet. Comprises the original author of the Tweet and the users which have since retweeted it.

{\bf Retweet Count}
The number of times a particular Tweet has been retweeted.

{\bf Timeline}
A collection of Tweets in Twitter in reverse-chronological order. A user timeline consists of that user's Tweets. A user's home timeline consists of the Tweets of each friend of the user.

{\bf Tweet}
\textit{n.} - A piece of information in Twitter; a piece of text, less than 240 characters long, which is written by a user. When sent, the Tweet is sent to the home timelines of each of the followers of the Tweet's author.\\
\textit{v.} - The act of writing and sending a Tweet.

{\bf User}
An account on Twitter. Each user (usually representing a real-life person or organisation) can Tweet, retweet, follow other users and be followed by other users. In this thesis, sometimes the terms \textit{user} and \textit{person} are used interchangeably.