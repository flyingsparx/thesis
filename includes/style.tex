\documentclass[a4paper,oneside,onecolumn,openright,12pt]{book}


\makeatletter

%
\usepackage{caption}
\usepackage{subcaption}
\usepackage{footmisc}
\usepackage{paralist}
\usepackage{amsthm}
\usepackage{subfig}

% Fonts, encoding, etc.
\usepackage{type1cm}
\usepackage[latin1]{inputenc}
\usepackage[british]{babel}
\usepackage[T1]{fontenc}
\usepackage{times}
\usepackage{hyperref}
\usepackage[all]{hypcap}
\usepackage{longtable}
\usepackage{algorithm}
\usepackage{algpseudocode}
\usepackage{multirow}
\usepackage{float}


% Line spacing
\def\baselinestretch{1.5}
\parindent0cm
\parskip1.5ex\@plus.7ex\@minus.1ex\relax

%Acronyms
 \usepackage{acronym}
 %Inline lists
 \usepackage{paralist}

%Glossary
\usepackage[toc]{glossaries}

% Page dimensions
\usepackage{vmargin}
\setpapersize{A4}
\setmarginsrb{40mm}{20mm}{25mm}{30mm}{14.5pt}{8mm}{0pt}{11mm}

% Footer and header
\usepackage{afterpage}
\usepackage{fancyhdr}
\pagestyle{fancy}
\fancyhead{}
\fancyhead[LE,RO]{\thepage}
\fancyhead[LO,RE]{\slshape \leftmark}
\fancyfoot{}
\renewcommand{\chaptermark}[1]{}
\renewcommand{\sectionmark}[1]%
             {\markboth{\thesection\ #1}{\thesection\ #1}}
\renewcommand{\subsectionmark}[1]{}
\fancypagestyle{plain}{%
  \fancyhead{}
  \fancyhead[LE,RO]{\thepage}
  \fancyfoot{}
  \renewcommand{\headrulewidth}{.6pt}
}

% Chapter
\def\@makechapterhead#1{%
  \ \\[-35.5pt]\hbox to \textwidth {%
    \hfill {\vbox{\hbox{\rule[5pt]{140pt}{4pt}}%
        \hbox to 140pt {\hfill\huge\bfseries\slshape \@chapapp\space\thechapter\/}}}}%
  \vskip55\p@%
  {\parindent \z@ \raggedright \normalfont%
    \interlinepenalty\@M%
    \Huge \bfseries #1\par\nobreak%
    \vskip 45\p@%
  }}
\def\@chapter[#1]#2{\ifnum \c@secnumdepth >\m@ne
                       \if@mainmatter
                         \refstepcounter{chapter}%
                         \typeout{\@chapapp\space\thechapter.}%
                         \addcontentsline{toc}{chapter}%
                                   {\protect\numberline{\thechapter}#1}%
                       \else
                         \addcontentsline{toc}{chapter}{#1}%
                       \fi
                    \else
                      \addcontentsline{toc}{chapter}{#1}%
                    \fi
                    \chaptermark{#1}%
                    \addtocontents{lof}{\protect\addvspace{10\p@}}%
                    \addtocontents{lot}{\protect\addvspace{10\p@}}%
                    \addtocontents{loa}{\protect\addvspace{10\p@}}%
                    \if@twocolumn
                      \@topnewpage[\@makechapterhead{#2}]%
                    \else
                      \@makechapterhead{#2}%
                      \@afterheading
                    \fi}
\def\@schapter#1{\addcontentsline{toc}{chapter}{#1}%
                 \markboth{#1}{#1}%
                 \addtocontents{lof}{\protect\addvspace{10\p@}}%
                 \addtocontents{lot}{\protect\addvspace{10\p@}}%
                 \addtocontents{loa}{\protect\addvspace{10\p@}}%
                 \if@twocolumn%
                    \@topnewpage[\@makeschapterhead{#1}]%
                 \else%
                    \@makeschapterhead{#1}%
                    \@afterheading%
                 \fi}
\def\@makeschapterhead#1{%
  \ \\[-35.5pt]\hbox to \textwidth {%
    \hfill {\vbox{\hbox{\rule[5pt]{140pt}{0pt}\rule[5pt]{0pt}{4pt}}%
        \hbox to 140pt {\hfill\huge\bfseries\slshape \ \/}}}}%
  \vskip55\p@%
  {\parindent \z@ \raggedright \normalfont%
    \interlinepenalty\@M%
    \Huge \bfseries  #1\par\nobreak%
    \vskip 45\p@%
  }}

% Table of contents
\def\contentsname{Contents}
\renewcommand\tableofcontents{%
    \if@twocolumn%
      \@restonecoltrue\onecolumn%
    \else%
      \@restonecolfalse%
    \fi%
    \chapter*{\contentsname}%
    \@starttoc{toc}%
    \if@restonecol\twocolumn\fi%
    }

% Bibliography
\renewenvironment{thebibliography}[1]
     {\chapter*{\bibname}%
      \list{\@biblabel{\@arabic\c@enumiv}}%
           {\settowidth\labelwidth{\@biblabel{#1}}%
            \leftmargin\labelwidth
            \advance\leftmargin\labelsep
            \@openbib@code
            \usecounter{enumiv}%
            \let\p@enumiv\@empty
            \renewcommand\theenumiv{\@arabic\c@enumiv}}%
      \sloppy
      \clubpenalty4000
      \@clubpenalty \clubpenalty
      \widowpenalty4000%
      \sfcode`\.\@m}
     {\def\@noitemerr
       {\@latex@warning{Empty `thebibliography' environment}}%
      \endlist}

% Floats
\long\def\@makecaption#1#2{%
  \vskip\abovecaptionskip
  \sbox\@tempboxa{\textbf{#1: #2}}%
  \ifdim \wd\@tempboxa >\hsize
    \textbf{#1: #2.}\par
  \else
    \global \@minipagefalse
    \hb@xt@\hsize{\hfil\box\@tempboxa\hfil}%
  \fi
  \vskip\belowcaptionskip}
\renewcommand{\topfraction}{0.9}
\renewcommand{\textfraction}{0.1}
\renewcommand{\floatpagefraction}{0.9}

% Tables
\usepackage{dcolumn}
\usepackage{hhline}

% Graphics
\usepackage[dvips]{graphicx}
\usepackage[usenames,dvipsnames]{color}
\usepackage{rotating}
\usepackage{psfrag}
\usepackage{epic}
\usepackage{eepic}


% List of figures and tables
\renewcommand\listoffigures{%
    \if@twocolumn
      \@restonecoltrue\onecolumn
    \else
      \@restonecolfalse
    \fi
    \chapter*{\listfigurename}%
      \@mkboth{\listfigurename}{\listfigurename}%
    \@starttoc{lof}%
    \if@restonecol\twocolumn\fi
    }
\renewcommand\listoftables{%
    \if@twocolumn
      \@restonecoltrue\onecolumn
    \else
      \@restonecolfalse
    \fi
    \chapter*{\listtablename}%
      \@mkboth{\listtablename}{\listtablename}%
    \@starttoc{lot}%
    \if@restonecol\twocolumn\fi
    }

% Math symbols, fonts, etc.
\usepackage{amsmath}
\usepackage{amsfonts}
\usepackage{amssymb}

\newcommand{\N}{\mathbb{N}}
\newcommand{\Z}{\mathbb{Z}}
\newcommand{\Q}{\mathbb{Q}}
\newcommand{\R}{\mathbb{R}}
\newcommand{\C}{\mathbb{C}}
\renewcommand{\S}{\mathbb{S}}
\renewcommand{\P}{\mathbb{P}}
\newcommand{\E}{\mathbb{E}}

\newcommand{\Cf}{\mathfrak{C}}
\newcommand{\Pf}{\mathfrak{P}}

\DeclareMathOperator{\sign}{sign}
\DeclareMathOperator{\avg}{avg}
\DeclareMathOperator{\floor}{floor}
\DeclareMathOperator{\ceil}{ceil}
\DeclareMathOperator{\round}{round}

\providecommand{\abs}[1]{\lvert#1\rvert}
\providecommand{\absd}[1]{\left\lvert#1\right\rvert}
\providecommand{\card}[1]{\lvert#1\rvert}
\providecommand{\norm}[1]{\lVert#1\rVert}

% URLs
\usepackage{url}
%% Define a new 'leo' style for the package that will use a smaller font.
\makeatletter
\def\url@leostyle{%
  \@ifundefined{selectfont}{\def\UrlFont{\sf}}{\def\UrlFont{\footnotesize\ttfamily}}}
\makeatother
%% Now actually use the newly defined style.
\urlstyle{leo}

\usepackage{thmbox}
\newtheorem[S,leftmargin=18pt,thickness=0.9pt,bodystyle=\noindent]{mydefinition}{Definition}[chapter]
\newtheorem[S,leftmargin=18pt,thickness=0.9pt,bodystyle=\noindent]{mytheorem}{Theorem}[chapter]


% STUFF I ADDED:

\usepackage{amssymb}
\usepackage{graphicx}
\usepackage{float}
\usepackage{pgfplots}
\usepackage{pgfplotstable}
\pgfplotsset{
    % #1: index in the group(0,1,2,...)
    % #2: number of plots of that group
    bar group size/.style 2 args={
        /pgf/bar shift={%
                % total width = n*w + (n-1)*skip
                % -> subtract half for centering
                -0.5*(#2*\pgfplotbarwidth + (#2-1)*\pgfkeysvalueof{/pgfplots/bar group skip})  + 
                % the '0.5*w' is for centering
                (.5+#1)*\pgfplotbarwidth + #1*\pgfkeysvalueof{/pgfplots/bar group skip}},%
    },
    bar group skip/.initial=2pt,
    plot 0/.style={blue,fill=blue!30!white,mark=none},%
    plot 1/.style={red,fill=red!30!white,mark=none},%
    plot 2/.style={brown!60!black,fill=brown!30!white,mark=none},%
}

\usepackage{listings}
\usepackage{color}
\lstset{ %
  language=Octave,                % the language of the code
  basicstyle=\footnotesize\ttfamily,           % the size of the fonts that are used for the code
  numbers=left,                   % where to put the line-numbers
  numberstyle=\tiny\color{black},  % the style that is used for the line-numbers
  stepnumber=2,                   % the step between two line-numbers. If it's 1, each line 
                                  % will be numbered
  numbersep=5pt,                  % how far the line-numbers are from the code
  backgroundcolor=\color{white},      % choose the background color. You must add \usepackage{color}
  showspaces=false,               % show spaces adding particular underscores
  showstringspaces=false,         % underline spaces within strings
  showtabs=false,                 % show tabs within strings adding particular underscores
  rulecolor=\color{black},        % if not set, the frame-color may be changed on line-breaks within not-black text (e.g. commens (green here))
  tabsize=3,                      % sets default tabsize to 3 spaces
  captionpos=b,                   % sets the caption-position to bottom
  breaklines=true,                % sets automatic line breaking
  breakatwhitespace=false,        % sets if automatic breaks should only happen at whitespace
                                  % also try caption instead of title
  keywordstyle=\color{blue},          % keyword style
  commentstyle=\color{dkgreen},       % comment style
  stringstyle=\color{mauve}      % string literal style
}


% FUNCTIONS:

% retweet group
\newcommand{\rg}[1]{RG(#1)}

% retweets 
\newcommand{\rt}[1]{RT(#1)}

% retweet count
\newcommand{\rc}[1]{#1.\textrm{count}_R}

% expected retweet count
\newcommand{\ec}[1]{#1.\textrm{count}_E}
\newcommand{\ecg}[1]{#1.\textrm{count}^G_E}
\newcommand{\ecu}[1]{#1.\textrm{count}^U_E}

% follower count
\newcommand{\foc}[1]{\textrm{deg}^+(#1)}

% friend count
\newcommand{\frc}[1]{\textrm{deg}^-(#1)}

% author
\newcommand{\aut}[2]{#1.\textrm{author}_#2}

% set of followers
\newcommand{\fos}[1]{N^+(#1)}

% set of friends
\newcommand{\frs}[1]{N^-(#1)}

% immediate audience
\newcommand{\ima}[1]{\textrm{audience}(#1)}

% distinct audience
\newcommand{\dia}[1]{\textrm{distinct audienct}(#1)}

% interestingness score
\newcommand{\score}[1]{s(#1)}
\newcommand{\gscore}[1]{s_G(#1)}
\newcommand{\uscore}[1]{s_U(#1)}
\newcommand{\ascore}[1]{s_{\textrm{avg}}(#1)}

% score disparity
\newcommand{\disparity}[1]{d(#1)}
\newcommand{\gdisparity}[1]{d_G(#1)}
\newcommand{\udisparity}[1]{d_U(#1)}
\newcommand{\sdisparity}[1]{d_{\textrm{sel}}(#1)} % disparity between SELECTED tweets in question #1

% END OF STUFF I ADDED

\makeatother
