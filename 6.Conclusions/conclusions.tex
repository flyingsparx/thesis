\chapter{Assessment and Conclusions}
This chapter includes an overview and assessment of the work conducted in this thesis, bringing together the ideas from the initial research and how these have helped in developing the methodologies introduced in later chapters. The validations from the methods are further assessed, followed by how the research forming them may be taken further in potential future projects. Finally, an overview of the thesis in terms of its contributions are described.


\section{Analysis of Research and Results}
Following is an analysis of the research carried out over the stages described in the main chapters of this thesis, from the initial research into retweeting and the social graph through to the interestingness inference methodologies explained at later stages.

\subsection{Summary \& Analysis of Initial Research}
Initial research was conducted into the act of retweeting and Twitter in general for the purposes of providing a background and foundation for the later work. In particular, research was carried out into retweet \textit{groups} and the properties and behaviours they demonstrate.

It was found that, agreeing with other research in the area, retweet groups, which represent the set of retweets (and their authors) of a particular Tweet, can have widely ranging sizes and depth. The path-length of a retweet group's branch was defined as the number of retweet hops between the author of the initial Tweet and that of the final retweet of the chain. This phenomenal takes into account that retweets can, themselves, be retweeted, and that retweet groups do \textit{not} consider the followships between the set of user's they represent. Retweet groups were found to present an average \textit{maximum} path-length of around two, and the longest maximum path found in the dataset collected from retweets on the public timeline was of length nine. This demonstrates a significant penetration through the social graph, especially considering the `real' world's six degrees of separation, and that social networks often exhibit a social graph even more closely connected than this.

This, and other retweet group analyses, led more onto more focused research on Twitter's social graph, which began by studying the properties of the \textit{audience} size attainable through retweet groups, and the overhead generated through shared followships, and ties between users involved in retweet groups and their relationships on the social graph. It was found that the chance of a retweeter retweeting a Tweet was significantly greater to occur in cases where the retweeter follows the author of the original Tweet, but that as retweet pathways become longer, the chances of the final retweeter following the original author diminishes over the distance, demonstrating strong correlations between the edges between users on the retweet graph and those on the social graph.

Since, at this stage, it was demonstrable that the social structure of Twitter clearly affects the propagation of retweets, and that this property could provide a way of inferring interestingness. Research then focused on examining the differences in propagation patterns in order to demonstrate that each structure type can present very different retweet propagation patterns. Because the propagation pattern difference at this structural level was so large, it was decided that this could be a basis for an interestingness inference methodology. This method utilised the same research and algorithms behind those used in the graph structure analysis to predict a retweet count for a given Tweet within a graph of connected users, and worked through a simple comparison between this predicted value and the \textit{observed} retweet count of the Tweet. This method was not shown to perform particularly well in the validation tests conducted, and thus improvements were necessary before any further analyses were made.


\subsection{Summary of the Improved Methodology}



\subsection{Analysis of Methodology Validation Results}

\begin{itemize}
\item Have methods been able to sensibly predict retweet volumes?
\item Have methods sensibly inferred Tweet interestingness?
\item What might have worked better?
\item Which parts were useless?
\item Which parts helped develop other areas of research which may have provided further avenues of research ideas?
\end{itemize}



\section{Further and Future Work}
How can this research be taken further in the future?

\begin{itemize}
\item Use previous results to predict how far a tweet is likely to be retweeted (for advertising purposes)
\item Useful for detecting the kind of messages that are likely to travel further
\item As well as providing an interest level, the systems also predict sensible estimations on retweet volumes.
\item Perhaps useful for measuring the spread of rumours.
\end{itemize}

% this stuff:
One route for this would be to try and infer a user's local network from a set of their immediate parameters, drawing on our earlier work suggesting that the Twitter network has th    e properties of a scale-free small-world graph. Through studying graph patterns, it is possible to make sensible inferences on the edges and nodes of a user's local network based o    n their follower count. From this, a graph edge density can be calculated, $ d = \frac{|E|}{|N|(|N|-1)} $, for use in generating a scale-free network.

- remove links between users, do still receive the information - (future work?)


\section{Conclusions}
\subsection{Summary}
Summarise events and processes covered, reiterate what the point of the work was and how each part of the work covered relates to that.

\subsection{Contributions}
Restate the original contributions (from Introduction section). Explain the ways in which the work done relates to the projected contributions, that it is novel and useful.
