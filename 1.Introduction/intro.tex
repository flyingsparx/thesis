\chapter{Introduction}

Online social networks have exploded into the lives of millions of people worldwide over the last decade, and their use has facilitated the interconnection of the world in ways never before perceived possible.

These social networks have many characteristics that are also exhibited by `real'-world social networks. Although most such platforms each provide a different service to collaboratively satisfy an array of different use-cases, they tend to all be based around the idea of `friendships' (i.e. links between the user nodes in the social graph) and the sharing of information amongst friends.

Social networks like these have been available for around ten years now (with MySpace\footnote{http://myspace.com} launching in 2003 and Bebo\footnote{http://bebo.com} in 2005), but it wasn't really until the worldwide launch of Facebook\footnote{http://facebook.com} in 2006 that social networks became the staple, ubiquitous norm that they are today. More recently, we have seen the introductions of Google's social network grown from its Buzz service, Google Plus\footnote{http://plus.google.com}, Pinterest\footnote{http://pinterest.com}, App.net\footnote{http://app.net}, and many more. They make up a large part of and contribute heavily towards the ideas behind Web 2.0, which describes the web as being primarily formed from user-generated content and encourages the sharing of such content.

Another component that helped in the dawn of Web 2.0 was the rise of \textit{blogging}. A blog (`web-log') is a time-based series of posts consisting of continuous pieces of text, photos, or other media, and is generally contributed to by a single author. Blogs are often based around one or a set of topics and are usually public - meaning that they are written with the intention of being read by others. Despite this, they are often a way in which the author can look back at their history of posts, acting more as a diary recording snapshots of the author's life.\\
Various blogging services exist on the web today, such as Medium\footnote{http://medium.com}, Wordpress\footnote{http://wordpress.com}, and Tumblr\footnote{http://tumblr.com}.


\section{Twitter as a Social Network}
Twitter\footnote{http://twitter.com} is an online social network, which launched in the summer of 2006 \cite{krishnamurthy08}. Since then, it has rapidly gained in popularity amongst several different user groups - teens and young people, casual users, celebrities, reporters, and so on - and within eight months had around 94,000 registered users \cite{java07}. Although Twitter has never been a direct competitor with Facebook, users tend to use the two sites concurrently for different purposes: whilst Facebook's focus is on providing many services at once (such as photo-sharing, commenting/endorsing of information, messaging, pages for businesses, groups, events, etc.), Twitter's is more on simplicity.

More specifically than just being an online social network, Twitter is a microblogging website. Whilst a blog, as mentioned, typically contains long posts, Twitter only allows its users to post short pieces of text, up to 140 characters in length \cite{krishnamurthy08} \cite{huberman08}, called `Tweets'. Thus, Twitter is a hybrid social network and blogging service and whilst each Tweet may only realistically be able to hold a couple of sentences, this system facilitates quick, timely, and `real-time' \textit{live} information-sharing amongst its millions of users \cite{zhao09}. Its idea is that short pieces of news will `travel' faster and will be seen by more people more quickly than traditional news stories.

Although Tweets are limited to 140 characters in length, the inclusion of URLs is allowed. This enables further extension of Tweets through external websites, and supports the inclusion of links to images and videos. Twitter has encouraged this use-case by providing `share' buttons for developers to embed in websites, and direct support for photo and video applications, such as TwitPic\footnote{http://twitpic.com} and Vine\footnote{http://vine.com}.

Its simplicity has also helped its growth into the mobile domain, in which smartphone users are able to very quickly post updates about their lives, a piece of information they want to share, or a photo or video, and be able to post it \textit{as it happens} directly from the news source or geographical location \cite{castillo11}. This has been especially useful in emergency situations worldwide, including the Haiti earthquake in 2010 \cite{muralidharan11}, and 2011's Egyptian protests \cite{wilson11} and Thai flood \cite{kongthon12}.\\
Indeed, \cite{sakaki10} used Twitter to build an earthquake-reporting system for Japan that outperforms the Japan Meteorological Agency in terms of its promptness of notification.

Use of Twitter is based around `timelines' of Tweets, to which new Tweets are pre-pended as they are posted by users. The \textit{home} timeline is the default view, in which Tweets from all of a person's subscribed-to users are placed. Timelines of an individual user contain only Tweets from that user, and are known as a `user' timeline. Customisation of timelines is also possible through the use of Twitter lists, in which different users can be placed to categorise streams of Tweets from different sets of users.


\section{Twitter's Social Graph and Information Subscription}
The structure of Twitter lies within the users and their connectivity within its social graph. However, unlike Facebook, whose social structure is made up of bi-directional `friendships' between users, Twitter's primary social graph is made up more of mono-directional links between its users \cite{edwards13}. A person using Twitter can elect to \textit{follow} another user, which subscribes the person to receive all of that user's Tweets to their home timeline. The set of users that follow a person are known as that person's \textit{followers}, and the set of users that the person follows are the person's \textit{friends}. Therefore, if two users both mutually follow each other, then the link between them is bi-directional.

Whilst bi-directional links are common amongst communities of similar interests, friends, colleagues, and so on, mono-directional links are found more in situations in which less-influential users follow more-influential users, such as celebrities.


\section{The Problem}
A person who follows a set of other users can generally be said to find that set of users to produce more interesting information than those that the person does not follow. However, despite that, not \textit{all} information produced by an `interesting' user is likely to be interesting to the person, and yet \textit{all} information produced by a Twitter friend will be received onto the home timeline.

Noise is a common problem in Twitter, and is the uninteresting information one might receive that conveys little interest. It is likely that most of the information received on Twitter \textit{is} uninteresting \cite{alonso10}, and this makes it very hard to distinguish the interesting information from the uninteresting.

Since people tend to use Twitter most in short sporadic moments, looking for a quick news fix, they do not have time to filter out noisy information. Thus, the presence of noise can dampen the experience of the user, making it much more difficult to find interesting information.

In addition, Twitter users typically exist within an information `bubble'. This is similar to the notion of the Google search bubble, in which the search engine uses previous results and search terms to only return information to a user based on what \textit{it thinks} the user would find the most interesting and useful. This results in the users not knowing which information exists beyond the confines of their bubble, and if they do not know it exists, they cannot know if it is of interest to them. Similarly, a Twitter user cannot follow all of the users he/she may find interesting, since he/she will not \textit{know} of all the interesting users existing on the social graph.

Although not directly answered in this thesis, the key question and motivation behind the work in the thesis is;\\
\centerline{\textbf{How can users be exposed to \textit{interesting} and \textit{relevant} information,}}
\centerline{\textbf{but without them having to know about it or look for it first?}}

\section{Main Contributions}
This thesis focuses on understanding information propagation, and how this combined with knowledge of the social structure of Twitter can assist towards solving the problem of identifying interesting and relevant information and determining it from the noise on Twitter. Whilst other work in the area has also looked into the notions of relevance and interest in online social networks, and Twitter in particular, none has addressed the problem with such rigor or have attempted to validate findings so thoroughly.

The thesis provides an in-depth survey of relevant research in the area, following onto a thorough understanding of retweet behaviour and its ties to the properties of the underlying Twitter social graph. 

Part of the outcome of this research are methods for effectively inferring interesting information and, indeed, paving ways to support the of ranking information in a more useful way than simply by time. The methods are validated in various ways to help highlight their strengths and weaknesses in performing inferences and in appropriate use-cases. The work addresses the problem area in that it helps towards solving the goal of identifying \textit{globally} interesting information in Twitter, and the ways in which users are able to identify interesting information from noise. In addition, certain measures are taken in an attempt to address the idea of information relevance in terms of validations being made on the methods by users assessing Tweets from users they have a previously-declared interest in.


\section{Thesis Structure}
The rest of this thesis is structured as follows.

A background is provided as an introduction to some of the ideas behind the main research, which immediately follows this chapter, and includes a review of relevant literature across the range of topics addressed in the thesis.\\
Following this are chapters that contain research on Twitter's information propagation characteristics and its interesting and useful behaviours, the social structure of Twitter and the ways in which this is important for understanding the spread of information, and then on the research of the methodologies themselves, including validation and analysis of the results of this work.

The thesis ends with a general analysis and conclusion, and a discussion of potential future work in this area and leading on from this research.
