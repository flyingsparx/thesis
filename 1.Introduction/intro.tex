Online social networks have exploded into the lives of millions of people worldwide over the last decade, and their use has dominated the communication highways and facilitated the interconnection of the world in ways never before perceived possible.

These social networks imitate real-world social networks. Although most such platforms each provide a different service to collaboratively satisfy an array of different use-cases, they tend to all be based around the idea of `friendships' (i.e. links between the user nodes in the social graph) and the sharing of information amongst friends.

Social networks like these have been available for around ten years now (with MySpace\footnote{http://myspace.com} launching in 2003 and Bebo\footnote{http://bebo.com} in 2005), but it wasn't really until Facebook's\footnote{http://facebook.com} worldwide launch in 2006 that social networks became the staple, ubiquitous norm that they are today. More recently, we have seen the introductions of Google's social network grown from its Buzz service, Google Plus\footnote{http://plus.google.com}, Pinterest\footnote{http://pinterest.com}, App.net\footnote{http://app.net}, and many more. They make up a large part of the basis and meaning behind the ideas of Web 2.0, which describes the web as being primarily formed from user-generated content and encourages the sharing of such content.


\section{Twitter as a Social Network}
Twitter\footnote{http://twitter.com} is an online social network, which launched in the summer of 2006. Since then, it has rapidly gained in popularity amongst several different user groups - teens and young people, casual users, celebrities, reporters, and so on. Although Twitter has never been a direct competitor with Facebook, users tend to use the two sites concurrently for different purposes: whilst Facebook's focus is on providing many services at once (such as photo-sharing, commenting/endorsing of information, messaging, pages for businesses, groups, events, etc.), Twitter's is more on simplicity.

Further to being an online social network, Twitter is a microblogging website. Whilst a blog (`web-log') typically contains posts containing long pieces of text or other media focusing on a single or set of topics, such as  a diary, Twitter only allows its users to post short pieces of text, up to 140 characters in length, called Tweets. Whilst each Tweet may only realistically be able to hold a couple of sentences, this system facilitates quick, timely, and \textit{live} information-sharing amongst its millions of users. Its idea is that short pieces of news will `travel' faster and will be seen by more people more quickly than traditional news stories.

Although Tweets are limited to 140 characters in length, the inclusion of URLs is allowed. This enables further extension of Tweets through external websites, and supports the inclusion of links to images and videos. Twitter has encouraged this use-case by providing `share' buttons for developers to embed in websites, and direct support for photo and video applications, such as TwitPic\footnote{http://twitpic.com} and Vine\footnote{http://vine.com}.
Its simplicity has also helped its growth into the mobile domain, in which smartphone users are able to very quickly post updates about their lives, a piece information they want to share, or a photo or video, and be able to post it \textit{as it happens}. This has been especially useful in emergency situations worldwide, including the Haiti earthquake in 2010 and the protests in Egypt in 2011.

Use of Twitter is based around `timelines' of Tweets, to which new Tweets are pre-pended as they arrive. The \textit{home} timeline is the default view, in which Tweets from all of a user's subscribed users are placed. Timelines of an individual user contain only Tweets from that user, and are known as a `user' timeline. Customisation of timelines is also possible through the usage of Twitter lists, which can have users placed within to categorise different streams of Tweets.


\section{Twitter's Social Graph and Information Subscription}
As with many social networks, the structure of Twitter lies within the users and their connectivity within its social graph. However, unlike Facebook, Twitter's primary social graph is made up of \textit{directional} links between its users. A person using Twitter can elect to \textit{follow} another user, which \textit{subscribes} the person to receive all of that user's Tweets to their home timeline. The set of users that follow a person are known as that person's \textit{followers}, and the set of users that the person follows are the person's \textit{friends}.

Whilst bi-directional links are common amongst communities of similar interests, friends, colleagues, etc., single-directional links are found more in situations in which less-influential users follow more-influential users, such as celebrities.


\section{The Problem}
A user who follows a set of other users can \textit{generally} be said to find that set of users to produce more interesting information than those users that the user does not follow. However, despite that, not \textit{all} information produced by an `interesting' user is likely to be interesting.
Noise is a common problem in Twitter, and is the uninteresting information one might receive that conveys little interest. It is likely that most of the information received on Twitter \textit{is} noise, and this makes it very hard to distinguish the interesting information from the uninteresting.
Since people tend to use Twitter for short sporadic moments and do not have time to filter out noisy information, the presence of noise can dampen the experience of the user, making it much more difficult to find interesting information.

In addition, Twitter users typically exist within an information `bubble'. This is similar to the notion of the Google search bubble, in which Google uses previous results and searches to only return information to a user based on what \textit{it thinks} the user would find the most interesting and useful. This results in the users not knowing which information exists beyond the confines of their bubble, and if they do not know it exists, they cannot know if it is of interest to them. Similarly, a Twitter user cannot follow all of the users he/she may find interesting, since he/she will not \textit{know} of all the interesting users existing on the social graph.

How can users be exposed to \textit{interesting} and \textit{relevant} information, but without them having to know about it or look for it first?


\section{Contributions}
This thesis focuses on understanding information propagation, and how this combined with knowledge of the social structure of Twitter can assist towards solving the problem of identifying interesting and relevant information on Twitter. 
Whilst other work in the area has also looked into the notions of relevance and interest in online social networks, and Twitter in particular, none has addressed the problem in such a way as this.

Part of the outcome of this research are methods for effectively inferring interesting information and, indeed, ranking information by interestingness. The methods are validated in various ways to help highlight their strengths and weaknesses in detecting interestingness in different ways.

The work addresses the problem area in that it helps towards solving the goal of identifying \textit{globally} interesting information in Twitter. In addition, certain measures are taken in an attempt to address the idea of information relevance, which denotes how information interestingness is subjective, and thus different from user to user.


\section{Thesis Structure}
The rest of this thesis is structured as follows.

