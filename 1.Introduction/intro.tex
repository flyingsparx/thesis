Online social networks have exploded into the lives of millions of people worldwide over the last decade, and their use has dominated the communication highways and facilitated the interconnection of the world in ways never before perceived possible.

These social networks imitate real-world social networks. Although most such platforms each provide a different service to collaboratively satisfy an array of different use-cases, they tend to all be based around the idea of `friendships' (i.e. links between the user nodes in the social graph) and the sharing of information amongst friends.

Social networks like these have been available for around ten years now (with MySpace \footnote{http://myspace.com} launching in 2003 and Bebo \footnote{http://bebo.com} in 2005), but it wasn't really until Facebook's \footnote{http://facebook.com} worldwide launch in 2006 that social networks became the staple, ubiquitous norm that they are today. More recently, we have seen the introductions of Google's social network grown from its Buzz service, Google Plus, Pinterest, App.net, and many more. They make up a large part of the basis and meaning behind the ideas of Web 2.0, which describes the web as being primarily formed from user-generated content and encourages the sharing of such content.


\section{Twitter as a Social Network}
Twitter is an online social network, which launched in the summer of 2006. Since then, it has rapidly gained in popularity amongst several different user groups - teens and young people, casual users, celebrities, reporters, and so on. Although Twitter has never been a direct competitor with Facebook, users tend to use the two sites concurrently for different purposes: whilst Facebook's focus is on providing many services at once (such as photo-sharing, commenting/endorsing of information, messaging, pages for businesses, groups, events, etc.), Twitter's is more on simplicity.

Further to being an online social network, Twitter is a microblogging website. Whilst a blog (`web-log') typically contains posts containing long pieces of text or other media focusing on a single or set of topics, such as  a diary, Twitter only allows its users to post short pieces of text, up to 140 characters in length, called Tweets. Whilst each Tweet may only realistically be able to hold a couple of sentences, this system facilitates quick, timely, and \textit{live} information-sharing amongst its millions of users. Its idea is that short pieces of news will `travel' faster and will be seen by more people more quickly than traditional news stories.

Although Tweets are limited to 140 characters in length, the inclusion of URLs is allowed. This enables further extension of Tweets through external websites, and supports the inclusion of links to images and videos. Twitter has encouraged this use-case by providing `share' buttons for developers to embed in websites, and direct support for photo and video applications, such as TwitPic and Vine.
Its simplicity has also helped its growth into the mobile domain, in which smartphone users are able to very quickly post updates about their lives, a piece information they want to share, or a photo or video, and be able to post it \textit{as it happens}. This has been especially useful in emergency situations worldwide, including the Haiti earthquake in 2010 and the protests in Egypt in 2011.

Use of Twitter is based around `timelines' of Tweets, to which new Tweets are appended as they arrive. The \textit{home} timeline is the default view, in which all of a user's

\section{Twitter's Social Graph and Information Subscription}
As with many social networks, the structure of Twitter lies within the users and their connectivity within its social graph. However, unlike Facebook, Twitter's primary social graph is made up of \textit{directional} links between its users. A person using Twitter can elect to \textit{follow} another user, which \textit{subscribes} the person to receive all of that user's Tweets to their home timeline. The set of users that follow a person are known as that person's \textit{followers}, and the set of users that the person follows are the person's \textit{friends}.

Whilst bi-directional links are common amongst communities of similar interests, friends, colleagues, etc., single-directional links are found more in situations in which less-influential users follow more-influential users, such as celebrities.

\section{Information Propagation through Retweeting}

Introduction to Twitter, act of Tweeting and retweeting information. Ideas:
\begin{itemize}
\item Twitter is an online social network
\item Microblogging - short, quick updates - useful for growing mobile domain
\item links in tweets allow extension
\item how the social graph is built up
\item Used for quick information consumption in spare moments
\item endorsement of info - retweets / likes / etc.
\item Retweeing act of forwarding on of information to those who follow you
\end{itemize}

\section{The Problem}
Explain about the problem of `noise' in Twitter:
\begin{itemize}
\item Not all information is interesting
\item Noisy tweets can dampen experience - becomes harder to find interesting information
\item Users cannot find what they do not know exists (similar to Google search bubble)
\item etc.
\end{itemize}

\section{Contributions}
\begin{itemize}
\item How is this work novel?
\item What benefits does this research provide?
\item Does the work solve The Problem? 
\item etc.
\end{itemize}
\section{Thesis Structure}
Break down the structure of thesis (i.e. refer to contents page, but also a general overview of the order of sections and what is discussed).

